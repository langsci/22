%8
% examples done
% bib done
% tables done
% crossrefs  

\section{Introduction}
\label{sec:8:Introduction}
\hypertarget{Toc376958929}{}Numeral systems are more endangered than languages. Cultural or commercial superiority of one group over another often results in borrowing\is{borrowing} of numerals, or replacements of parts or all of a numeral system, even in a language that itself is not endangered \citep{Comrie2005numsys}. In the Alor-Pantar (AP) context, the national language, Indonesian\il{Indonesian}, plays a dominant role in education and commerce, and this will inevitably lead to the replacement of the numerals and the arithmetic expressions with Indonesian\il{Indonesian} equivalents. It is therefore crucial to keep a record of the forms and patterns as they are currently used for future reference, and this chapter aims to be such a documentary record. 


The patterns described in this chapter fall into two broad classes, pertaining to two distinct linguistic levels: the word ({\S}~\ref{sec:8:Operations}) and the clause or sentence ({\S}~\ref{sec:8:Structures}). At the word level we describe how numeral words are created, discussing the structure of cardinals\is{cardinal numeral(s)} ({\S}~\ref{sec:8:Cardinal}), ordinals\is{ordinal numeral(s)} ({\S}~\ref{sec:8:Ordinal}) and distributives\ist{distributive numerals} ({\S}~\ref{sec:8:Distributive}). At the clause and sentence level, we describe the constructions that contain numerals and function to express the arithmetic of addition (\ref{sec:8:Addition}), subtraction ({\S}~\ref{sec:8:Subtraction}), multiplication ({\S}~\ref{sec:8:Multiplication}), division ({\S}~\ref{sec:8:Division}), and fractions ({\S}~\ref{sec:8:Fractions}). {\S}~\ref{sec:8:Summary} presents a summary and conclusions. 

Details on the data on which this chapter is based are given in the  Sources section at the end of this chapter. Adang\il{Adang}, Blagar\il{Blagar}, Kamang\il{Kamang} and Abui\il{Abui} are each very diverse internally. The data presented in this chapter are from the Dolap dialect of Blagar\il{Blagar}, the Takailubui dialect of Kamang\il{Kamang}, the Lawahing dialect of Adang\il{Adang}, and the Takalelang dialect of Abui\il{Abui}. These dialect names refer to the place where the variety is spoken.


\section{Operations to create numeral words} 
\label{sec:8:Operations}

Most of the cardinals\is{cardinal numeral(s)} in AP languages are historically morphologically complex forms. Within and across the languages we find variation in choice of numeral base, the type of operations invoked for the interpretation of the composite elements, and the ways in which these operations are expressed ({\S}~\ref{sec:8:Cardinal}). Ordinals\is{ordinal numeral(s)} in AP languages are possessive\is{possession} constructions that are derived from cardinals\is{cardinal numeral(s)}, where the ordered entity is the grammatical possessor\is{possession} of the cardinal\is{cardinal numeral(s)} ({\S}~\ref{sec:8:Ordinal}). Distributive numerals\is{distributive numerals} in AP languages are also derived from the cardinal\is{cardinal numeral(s)}, by reduplicating\is{reduplication} it partially or fully. When the cardinal\is{cardinal numeral(s)} contains more than one morpheme, generally only the right-most formative is reduplicated\is{reduplication} (section \ref{sec:8:Irregulardistributive}).  In all cases, the numeral words follow the noun they quantify. Cardinals\is{cardinal numeral(s)} may be preceded by a classifier\is{numeral classifier}, if the language has them. 

\subsection{Cardinal numerals}
\label{sec:8:Cardinal}
By cardinal\is{cardinal numeral(s)} numerals, we understand the set of numerals used in attributive quantification of nouns (e.g., `three dogs'). In enumeration, the numeral follows the noun in all AP languages (N NUM), as in Teiwa\il{Teiwa} \textit{yaf haraq} `house two' {\textgreater} `two houses'. If a language uses a sortal or mensural classifier\is{numeral classifier}, the classifier\is{numeral classifier} occurs between the noun and the numeral (N CLF NUM). The same cardinals\is{cardinal numeral(s)} that are used in enumeration are also used for non-referential counting (\textit{one, two, three, four, five, etc.}), and all the AP languages use the same numeral forms to count small animates (ants, flies, bees, or house lizards), large animates\is{animacy} (children, dogs, or pigs), and inanimates\is{animacy} (houses, rocks, stars, or coconut trees).

In all the AP languages we surveyed, the cardinal\is{cardinal numeral(s)} numbers `one' to `five' are morphologically simple forms, as illustrated in Table~\ref{tab:8:1}. The composition of `six' varies. Most of the AP languages have a monomorphemic `six', an example is Teiwa\il{Teiwa} \textit{tiaam}. Bi-morphemic forms for `six' are composed of (reflexes) of `five' and `one', e.g., Kula\il{Kula} \textit{yawaten sona}. The cardinals\is{cardinal numeral(s)} `seven' and higher consist of minimally two formatives in all AP languages. Often, these forms involve reflexes of `five', `one', `two', `three' and `four', as illustrated in Table~\ref{tab:8:1}, though other patterns are also attested \citep{SchapperEtAlTVnumeralsystems}. 

\begin{table}
\caption{`One' through `nine' in Teiwa\ilt{Teiwa} (Pantar) and Kula\ilt{Kula} (East Alor)}
\label{tab:8:1}


\begin{tabularx}{\textwidth}{XXXX}
\lsptoprule
\textbf{Cardinal\ist{cardinal numeral(s)}} & \textbf{Analysis} & \textbf{Teiwa\ilt{Teiwa}} & \textbf{Kula\ilt{Kula}}\\
\midrule
1 & 1 & \textit{nuk} & \textit{sona}\\
2 & 2 & \textit{(ha)raq} & \textit{yakwu}\\
3 & 3 & \textit{yerig} & \textit{tu}\\
4 & 4 & \textit{ut} & \textit{arasiku}\\
5 & 5 & \textit{yusan} & \textit{yawatena}\\
6 & 5  1 &  & \textit{yawaten sona}\\
6 & 6 & \textit{tiaam} & \\
7 & 5  2 & \textit{yes raq} & \textit{yawaten yakwu}\\
8 & 5  3 & \textit{yes nerig} & \textit{yawaten tu}\\
9 & 5  4 & \textit{yes na'ut} & \textit{yawaten arasiku}\\
\lspbottomrule
\end{tabularx}
\end{table}

From the above it can be inferred that the AP languages have at most six mono-morphemic numerals. This number is significantly fewer than the number we find in many European languages. Present-day English, for example, has twelve mono-morphemic cardinal\is{cardinal numeral(s)} numerals \citep[26]{VonMengden2010}. 

\largerpage Both within and across the AP languages we find variation in the way cardinals\is{cardinal numeral(s)} are composed \citep[cf.][]{Stump2010}. First, in choice of numeral base: in all systems both quinary (`base-five') and decimal (`base-ten') bases are used. Table~\ref{tab:8:1} includes examples of numerals with a quinary base (\textit{yes} in Teiwa\il{Teiwa}, \textit{yawaten} in Kula\il{Kula}). A decimal base is used in numerals `ten' and above; an illustration is Teiwa\il{Teiwa} \textit{qaar} in \textit{qaar nuk} `ten' and \textit{qaar raq} `twenty'. 

Second, the type of operations invoked for the interpretation of the morphemes that make up the compound numerals vary between addition (Abui\il{Abui} \textit{yeting} \textit{buti} `nine'{\textless} \textit{yeting} `five' + \textit{buti} `four'), subtraction (Adang\il{Adang}  \textit{ti}\textit{{\textglotstop}}\textit{i} \textit{nu} `nine' {\textless} \textit{ti}\textit{{\textglotstop}}\textit{i} (semantically opaque), \textit{nu} `one' {\textless} `minus one') and multiplication (Western Pantar\il{Western Pantar} \textit{ke atiga} `thirty' {\textless} \textit{ke} `ten' x \textit{atiga} `three'). Of these, subtraction is the least frequent.

Third, different types of operations are involved in the derivation of cardinals\is{cardinal numeral(s)}: typically they involve simple juxtaposition of bases (e.g., Abui\il{Abui} \textit{kar nuku} `ten' {\textless} \textit{kar} `ten', \textit{nuku} `one'), but in some cases, a lexeme is added that expresses the operation (e.g., the operator \textit{wal} signifying addition in numerals 11-19, e.g., Abui\il{Abui} \textit{kar nuku} \textbf{\textit{wal}} \textit{nuku} `eleven'. 

The number compounds in AP languages are all exocentric, that is, they lack a morphological head. In this respect they contrast with nominal compounds, which are typically endocentric (e.g., Teiwa\il{Teiwa} \textit{xam yir} `milk' {\textless} \textit{xam} `breast', \textit{yir} `water', where the rightmost element is the head). As both nominal and numeral compounds have stress on their final member, we can analyse both types of compounds as prosodically right-headed across the board in Alor-Pantar. 

In the Pantar languages in particular, the synchronic morphological make-up of numeral compounds can be rather obscure. For instance, Teiwa\il{Teiwa} \textit{yesnerig} `eight' is not a transparent compound of synchronic \textit{yusan} `five' + \textit{yerig} `three'. In contrast, the languages of Central and East Alor have more transparent numeral compounds, for instance Abui\il{Abui} \textit{yetingsua} `eight' {\textless} \textit{yeting} `five' + \textit{sua} `three'. Phonologically, however, in all the languages of the sample discussed here, we can still recognize compound forms because they consist of two stressed phonological words, the second of which has primary stress. (We return to this issue in {\S}~\ref{sec:8:Regulardistributive} below.) 

We have not attested an AP language with a number word for `null' or `zero'. The absence of entities is rather expressed predicatively, using a word meaning `(be) empty', such as Teiwa\il{Teiwa} \textit{hasak} in \REF{bkm:Ref355363568}.
\footnote{Compare proto-Alor-Pantar\ilt{proto-Alor-Pantar} *hasak \citep{HoltonRobinsonTVhistory}, reflected in Western Pantar\ilt{Western Pantar} \textit{hakkas}, Kaera\il{Kaera} \textit{isik}, Abui\ilt{Abui} \textit{taka}, Kamang\ilt{Kamang} \textit{saka}.} In the Teiwa\ilt{Teiwa} idioms in \REF{bkm:Ref355363568}, a subject precedes a nominal predicate that is headed by the place pronoun\is{pronoun} \textit{i} `it.(place)', so that absent entities are expressed as ``X is (an) empty place(s)'', compare (\REF{bkm:Ref355363568} (a-b).


\ea%bkm:Ref355363568
\label{bkm:Ref355363568}
{\upshape Teiwa}\\
\ea
\label{ex:8:1234}
\gll Guru       {\ob}i       hasak{\cb}\textsubscript{\upshape\textsc{pred}} \\  
     teacher\textsc{(ind)}   it.(place)    empty'  \\
\glt  `No / zero teachers' 
\ex
\gll Yaf {\ob}i     hasak{\cb}\textsubscript{\upshape\textsc{pred}} \\
    house      it.(place)   empty\\
\glt    `No / zero people' 
\z
\z

In sum, AP languages have up to six morphologically simple cardinals\is{cardinal numeral(s)}; in all AP languages, the non-borrowed\is{borrowing} cardinals\is{cardinal numeral(s)} `seven' and up are morphologically complex. Most cardinals\is{cardinal numeral(s)} are compounds, consisting of two or more morphemes in apposition, the second of which gets word stress. The definition of the morphological structure of these compounds varies along three dimensions: the choice of base, the arithmetic operations invoked for the interpretation of the cardinals\is{cardinal numeral(s)} that make up the numeral, and the ways in which these arithmetic operations are expressed. 

\subsection{Ordinal  numerals\ist{ordinal numeral(s)}}
\label{sec:8:Ordinal}
Ordinal\is{ordinal numeral(s)} numerals are words that identify the position that a given member of a set occupies relative to other members of the same set (e.g., `the third dog'). The main function of ordinal\is{ordinal numeral(s)} numerals is thus to indicate the position of an entity in an ordered sequence. 

All AP languages have distinct forms for cardinal\is{cardinal numeral(s)} and ordinal\is{ordinal numeral(s)} numerals, and all of them have ordinal\is{ordinal numeral(s)} numbers associated with any cardinal\is{cardinal numeral(s)} from `two' and above. Ordinals\is{ordinal numeral(s)} in AP languages are derived from cardinals\is{cardinal numeral(s)}, which is a cross-linguistically common pattern \citep{StolzEtAl2013}. Variation exists only in the expression of `first', which in some of the languages is unrelated to the numeral `one', as discussed below. 

The derivation of ordinals\is{ordinal numeral(s)} involves a third person possessive\is{possession} pronoun\is{pronoun} or prefix at the left periphery of the cardinal\is{cardinal numeral(s)} numeral. The ordered entity functions grammatically as the possessor\is{possession} of the cardinal\is{cardinal numeral(s)} number. For example, Kamang\il{Kamang} \textit{dum yeok} `child 3.\textsc{poss}{}-two', lit. `child its-two' {\textgreater} `second child'. 

Within the ordinal\is{ordinal numeral(s)} possessive\is{possession} constructions, three areal patterns are discernible. The first pattern is that of the languages of Pantar and the Straits, where the possessive\is{possession} ordinal\is{ordinal numeral(s)} construction includes an additional element specific for ordinals\is{ordinal numeral(s)}. The second pattern is found in Central-East Alor, where ordinals\is{ordinal numeral(s)} are also expressed like possessive\is{possession} constructions, but without including an additional ordinal\is{ordinal numeral(s)} element. The third pattern is found in Kula\il{Kula} and Sawila\il{Sawila} in East Alor, where the ordinals\is{ordinal numeral(s)} involve an applicative verb. We discuss the three patterns in turn.

In the languages of Pantar and the Straits, possessive\is{possession} constructions like those in \REF{bkm:Ref342649616} are the base for ordinal\is{ordinal numeral(s)} constructions such as those in \REF{bkm:Ref342649632}, where the elements \textit{maing}, \textit{ma} or \textit{mi} occur between the possessor\is{possession} prefix and the numeral. (Full paradigms of ordinal\is{ordinal numeral(s)} constructions are presented in the Appendix \ref{sec:8:app:1}.) 


\let\eachwordone=\itshape
\let\eachwordtwo=\itshape
\let\eachwordthree=\itshape
\let\eachwordfour=\itshape
\let\eachwordfive=\itshape 

\ea\label{bkm:Ref342649616}
\gllllll  {\upshape W Pantar\ilt{Western Pantar}}    aname    gai    bla\\
  {\upshape Teiwa\ilt{Teiwa}}      masar    ga-    yaf\\
  {\upshape Kaera\ilt{Kaera}}      masik    ge-    ma\\
  {\upshape Blagar\ilt{Blagar}}      mehal    {\textglotstop}e-    hava\\
  {\upshape Adang\ilt{Adang}}       nami    {\textglotstop}o-    bang\\
   { }        man   \textsc{3.poss}  house  \\
\glt `the man's house'  \\ 
\z

\ea
\label{bkm:Ref342649632}
\gllllll
      {\upshape W Pantar\ilt{Western Pantar}}     aname  gai  maing  atiga \\  
      {\upshape Teiwa\ilt{Teiwa}}          masar  ga-  ma-  yerig \\
      {\upshape Kaera\ilt{Kaera}}          masik  ge-  mi-  tug \\
      {\upshape Blagar\ilt{Blagar}}         mehal {\textglotstop}e-  mi-  tue  \\
      {\upshape Adang\ilt{Adang}}          nami  {\textglotstop}o-  mi-  towo  {}-mi \\      
      { }            man \textsc{3.poss}  \textsc{ord-} three  -\textsc{ord} \\
\glt  `the third man'
\z

\let\eachwordone=\itshape
\let\eachwordtwo=\upshape
\let\eachwordthree=\upshape
\let\eachwordfour=\upshape
\let\eachwordfive=\upshape 

In Western Pantar\il{Western Pantar}, the ordinal\is{ordinal numeral(s)} element is a free form \textit{maing}; in Teiwa\il{Teiwa}, Kaera\il{Kaera}, Blagar\il{Blagar}, and Adang\il{Adang} it is a bound morpheme (\textit{ma-} or \textit{mi-}). The ordinal\is{ordinal numeral(s)} elements are formally similar to existing words in the respective languages: Western Pantar\il{Western Pantar} \textit{mayang} `to place', Teiwa\il{Teiwa} \textit{ma} `come, \textsc{obl}', and Kaera\il{Kaera}/Blagar\il{Blagar}/Adang\il{Adang} \textit{mi} `\textsc{obl' (}{\textless}pAP\il{proto-Alor-Pantar} *mai `come' and *mi `be in/on' \citealt{HoltonRobinsonTVhistory}). Synchronically, the semantic and syntactic link between these free forms and the ordinal\is{ordinal numeral(s)} markers is obscure. It may be that the ordinal\is{ordinal numeral(s)} morphemes express notions that are (historically) related to notions of placement or location at a particular numeral rank. However, their position preceding the numeral does not parallel the position of verbs and oblique markers, which in AP languages always \textit{follow} their nominal complement. Note however, that the ordinals\is{ordinal numeral(s)} in Adang\il{Adang} involve two identical morphemes: one preceding and one following the numeral. This might reflect an earlier structure where the ordinal\is{ordinal numeral(s)} marker followed the numeral, paralleling the position of case markers and verbs.

The second areal pattern of ordinal\is{ordinal numeral(s)} constructions is found in Central-East Alor, where ordinal\is{ordinal numeral(s)} constructions are also possessive\is{possession} constructions but now without an ordinal\is{ordinal numeral(s)} element included. Compare the constructions in \REF{bkm:Ref342649962} and \REF{bkm:Ref342309284}. The basic possessive\is{possession} construction in \REF{bkm:Ref342649962} includes a possessor\is{possession}, an alienable\is{alienability} possessive\is{possession} prefix and a possessum. In the ordinal\is{ordinal numeral(s)} constructions in \REF{bkm:Ref342309284}, the ranked entity is the possessor\is{possession} of the numeral indicating the rank.


\let\eachwordone=\itshape
\let\eachwordtwo=\itshape 

\ea%bkm:Ref342649962
\label{bkm:Ref342649962}
{\upshape Nominal possessive\ist{possession} construction in Central-East Alor} \\
\glll {\upshape Kamang\ilt{Kamang}}  lami  ge-  kadii \\  
     {\upshape Abui\ilt{Abui}}     neng  he-  fala  \\
     {}      man  3\textsc{.poss-} house\\
 \glt `the man's house'
\z


\ea%bkm:Ref342309284
\label{bkm:Ref342309284}
{\upshape Ordinal\ist{ordinal numeral(s)} construction in Central-East Alor}\\ 
\glll {\upshape Kamang\ilt{Kamang}}   lami  ge- su \\  
    {\upshape Abui\ilt{Abui}}     neng  he-  sua   \\
    { }  man  3\textsc{.poss-}  three\\
\glt  `the third man'
\z 



\let\eachwordone=\itshape
\let\eachwordtwo=\upshape 

In East Alor, ordinal\is{ordinal numeral(s)} structures that diverge from both these areal patterns are found in Kula\il{Kula} and Sawila\il{Sawila}. Kula\il{Kula} (Nick Williams, p.c. 2013) and Sawila\il{Sawila} ordinals\is{ordinal numeral(s)} employ applicative verbs involving the cognate prefixes \textit{we-/wii-}, illustrated in \REF{bkm:Ref355275027} and \REF{bkm:Ref355277550}. In Kula\il{Kula} ordinals\is{ordinal numeral(s)} this verb combines with a possessive\is{possession} structure, \REF{bkm:Ref358037937}. In Sawila\il{Sawila}, possessive\is{possession} constructions  are not used in ordinals\is{ordinal numeral(s)}, \REF{bkm:Ref342737790}. 


\ea%bkm:Ref355275027
\label{bkm:Ref355275027}
 
{\upshape Kula}\\
\gll wanta   gi-we-araasiku\\  
  day   3.\textsc{poss}{}-\textsc{appl}{}-four  \\
\glt  `the fourth day'
\z
 

\ea%bkm:Ref358037937
\label{bkm:Ref358037937}
{\upshape Kula}\\
\gll Maria   gi-skola\\  
   Maria   3.\textsc{poss-}school \\
\glt`Maria's school'
\z
 
  

  


\ea%bkm:Ref355277550
\label{bkm:Ref355277550}
{\upshape Sawila}\\
\gll   imyalara   wii-tua\\  
   man    \textsc{appl-}three \\
\glt  `the third man'
\z

  

  




\ea%bkm:Ref342737790
\label{bkm:Ref342737790}
{\upshape Sawila}\\
\gll imyalara   gi-araasing \\  
    man    3.\textsc{poss}-house   \\
\glt`the man's house'
\z
 

In all AP languages, the ordinals\is{ordinal numeral(s)} for `second' and higher are regularly derived. There is no limit in the creation of ordinals\is{ordinal numeral(s)} on the basis of higher, morphologically more complex, cardinals\is{cardinal numeral(s)}.

Some variation exists, however, in the expression of `first'. Adang\il{Adang} and Kamang\il{Kamang} form `first' by the regular process used for `second' and above. Teiwa\il{Teiwa} and Abui\il{Abui} use forms for `first' that are unrelated to the numeral `one', compare \REF{bkm:Ref342651040} (a-b) and \REF{bkm:Ref342651068} (a-b). In Teiwa\il{Teiwa}, the regular derivation from \textit{nuk} does not exist, \REF{bkm:Ref342651040}(b); in Abui\il{Abui}, it does exist, but has a different meaning (`the only/single/particular'), \REF{bkm:Ref342651068}(b). Western Pantar\il{Western Pantar} has two options to express `first'. One is to use the regular construction derived from \textit{(a)nuku} `one', as in \REF{bkm:Ref342651067}(a) while the other option is to use a different root \textit{ye} \REF{bkm:Ref342651067}(b) with an unclear etymology. There is a functional difference between Western Pantar\il{Western Pantar} ordinal\is{ordinal numeral(s)} based on \textit{anuku} which is often used in predicative contexts (`you are the first'), and \textit{ye}, which is preferred in attributive contexts (`my first child'). 


\ea%bkm:Ref342651040
\label{bkm:Ref342651040}
{\upshape Teiwa}\\
\ea
\gll  uy    ga-xol \\  
     person    3\textsc{.poss}-first\\
\glt`first person'
\ex
\gll *uy    ga-ma    nuk\\
person    3\textsc{.poss-ord} one\\ 
\z\z
 


\ea%bkm:Ref342651068
\label{bkm:Ref342651068}
{\upshape Abui}\\
\ea
\gll ama     he-teitu\\  
    person    3\textsc{.poss}-first  \\
\glt`first person'
\ex
\gll ama     he-nuku\\  
  person    3\textsc{.poss}-one  \\
\glt `(the) only/single person,  particular person'
\z\z

 


\ea%bkm:Ref342651067
\label{bkm:Ref342651067}
{\upshape Western Pantar}\\
 
\ea
\gll aname    gai  maing   anuku\\  
     person    3\textsc{.poss}   \textsc{ord} one\\ 
\ex
\gll aname    gai  maing   ye\\
    person    3\textsc{.poss}   \textsc{ord} one\\
\glt `first person'
\z\z
 

      

In sum, the AP languages regularly derive ordinals\is{ordinal numeral(s)} from numerals with a possessor\is{possession} morpheme, so that syntactically the ordinal\is{ordinal numeral(s)} construction is a possessed\is{possession} nominal phrase. Apart from the third person possessor\is{possession} morpheme, which is used across the board, ordinals\is{ordinal numeral(s)} vary in structure when we go from west to east. In the western languages (Pantar-Straits-West Alor) special morphemes are employed which may be etymologically related to free forms encoding locations, though synchronically, this relation is not transparent. In the eastern languages, ordinals\is{ordinal numeral(s)} involve an applicative morpheme. At least three of the AP languages have an ordinal\is{ordinal numeral(s)} `first' involving a root that is different from the cardinal\is{cardinal numeral(s)} `one'. This is in line with the cross-linguistic tendency for languages with ordinals\is{ordinal numeral(s)} unrelated to cardinals\is{cardinal numeral(s)} to confine them to the lowest numerals \citep{StolzEtAl2013}.

\subsection{Distributive numerals\ist{distributive numerals}}
\label{sec:8:Distributive}
\subsubsection{Forms and distribution of distributives\ist{distributive numerals}}
Distributive numerals\is{distributive numerals} function to express notions such as `one by one' or `in groups of three'. AP languages create distributive\is{distributive numerals} numerals by reduplication\is{reduplication} of the cardinal\is{cardinal numeral(s)} numeral, or a part of it. Cross-linguistically, reduplication\is{reduplication} is the most common strategy to form distributives\is{distributive numerals}: in about 33\% of the 251 languages in Gil's (2013) \nocite{Gil2013} sample, distributives\is{distributive numerals} are created in this way. As Gil points out, the reduplicative\is{reduplication} strategy is iconically motivated: repeated copies of the cardinal\is{cardinal numeral(s)} correspond to multiple sets of objects.

Distributive numerals\is{distributive numerals} follow the noun or pronoun\is{pronoun} they modify, as illustrated in (\ref{bkm:Ref342655549}--\ref{ex:8:1235}). Distributives\is{distributive numerals} can modify different clausal arguments; for example, an actor subject in \REF{bkm:Ref342655549} and \REF{bkm:Ref342738939} or a patient object in \REF{bkm:Ref342655553}.


\ea%bkm:Ref342655549
\label{bkm:Ref342655549}
{\upshape Teiwa}\\
\gll Iman  nuk\textit{{\Tilde}}nuk  aria-n.\\  
    \textsc{3pl}   \textsc{rdp}\textit{{\Tilde}}one   arrive-\textsc{real}\\
\glt `They arrived one by one.' 
\z

    

   

\xbox{\textwidth}{
\ea%bkm:Ref342738939
\label{bkm:Ref342738939}
{\upshape Abui}\\
\gll  Ama    rifi{\Tilde}rifi    sei        hel  buku  nu  he-waalri. \\  
    person    \textsc{rdp{\Tilde}}thousand   come.down.\textsc{cont}    \textsc{top} land  \textsc{spec}   3.\textsc{loc-}gather.in.\textsc{compl}  \\
\glt `People came in thousands to that place.' 
\z
}
    
 

   

   
\newpage 
\ea%bkm:Ref342655553
\label{bkm:Ref342655553}
{\upshape Teiwa}\\
\gll Yi   ma   gula   yerig{\Tilde}yerig   mat.\\  
     2\textsc{pl } come  sweet  \textsc{rdp}\textit{{\Tilde}}three  take \\
\glt `You take three sweets each.'
\z

   

  

   

In some AP languages distributives\is{distributive numerals} may float outside the NP to a position adjacent to the verb; an example is Adang\il{Adang}, \REF{ex:8:1235}. The exact restrictions and possibilities of such constructions across the AP languages remain a topic for future research; here we focus on the morphological shape of the distributives\is{distributive numerals}.


\ea
\label{ex:8:1235}
{\upshape Adang}\\
 \gll  Sunui{\textltailn}   papan   du   teweng al{\Tilde}alu  {\ob}allo{\cb}.\\
   3\textsc{pl}    board  \textsc{def}   carry   \textsc{rdp}{\Tilde}two \\
 \glt `They carry the board two by two (i.e., two at a time).'
\z

    

 

 

The following sections describe how distributives\is{distributive numerals} are derived: the regular patterns are discussed in {\S}~\ref{sec:8:Regulardistributive}, and the irregularities in {\S}~\ref{sec:8:Irregulardistributive}.  Full paradigms of distributives\is{distributive numerals} in five languages of our sample are given in the Appendix.

\subsubsection{Regular distributive\is{distributive numerals} formation  }
\label{sec:8:Regulardistributive}






Regular distributive\is{distributive numerals} formation in Alor-Pantar involves reduplication\is{reduplication} of (a part of) the cardinal\is{cardinal numeral(s)} number. In complex numerals it is usually the right-most element, the prosodic head ({\S}~\ref{sec:8:Cardinal}), that is the base for the reduplication\is{reduplication}. The result is a distributive\is{distributive numerals} form that contains word-internal reduplication\is{reduplication}. 

Even in languages where the morphological make-up of compound cardinals\is{cardinal numeral(s)} is synchronically opaque, such as Teiwa\il{Teiwa}, distributive\is{distributive numerals} reduplication\is{reduplication} splits the cardinal\is{cardinal numeral(s)} in two parts, and only the rightmost element, the prosodic head, is reduplicated\is{reduplication}; see the numerals `seven' to `nine' in Table~\ref{tab:8:2}. Also in numerals that contain an operator expressing addition, it is the right-most morpheme that is reduplicated\is{reduplication}, see (\ref{bkm:Ref342661452}--\ref{bkm:Ref342744393}) below.
 
\begin{table}[p]

\caption{Teiwa\ilt{Teiwa} cardinals\is{cardinal numeral(s)} and distributives\ist{distributive numerals}}
\label{tab:8:2}
\begin{tabularx}{\textwidth}{XXX}
\lsptoprule
& Cardinal\is{cardinal numeral(s)} & Distributive\ist{distributive numerals}\\
\midrule
1 & \textit{nuk} & \textit{nuk{\Tilde}nuk}\\
2 & \textit{raq} & \textit{raq{\Tilde}raq}\\
3 & \textit{yerig} & \textit{yerig{\Tilde}yerig}\\
4 & \textit{{\textglotstop}}\textit{ut} & \textit{{\textglotstop}}\textit{ut{\Tilde}}\textit{{\textglotstop}}\textit{ut}\\
5 & \textit{yusan} & \textit{yusan{\Tilde}yusan}\\
6 & \textit{tiaam} & \textit{tiaam{\Tilde}tiaam}\\
7 & \textit{yes}\textbf{\textit{raq}} & \textit{yes}\textbf{\textit{raq}}\textit{{\Tilde}raq}\\
8 & \textit{yesn}\textbf{\textit{erig}} & \textit{yesne}\textbf{\textit{rig}}\textit{{\Tilde}rig}\\
9 & \textit{yesna}\textbf{\textit{{\textglotstop}}}\textbf{\textit{ut}} & \textit{yesna}\textbf{\textit{{\textglotstop}}}\textbf{\textit{ut}}\textit{{\Tilde}}\textit{{\textglotstop}}\textit{ut}\\
\lspbottomrule
\end{tabularx}
\end{table}

In Adang\il{Adang}, distributives\is{distributive numerals} are formed by partial reduplication\is{reduplication}, as shown in Table~\ref{tab:8:3}. In the mono-morphemic forms `one' through `six', reduplication\is{reduplication} copies the first two segments (CV or VC) of the cardinal\is{cardinal numeral(s)}. Note that this analysis assumes that distributive\is{distributive numerals} `two' \textit{allo} and `five' \textit{iwwihing} are (historical) contractions of \textit{al-alu} and \textit{iw-iwing}. Numerals `seven' to `nine' are subtractive compound forms, in which the right-most element is the base for the reduplication\is{reduplication} (cf. \textit{to} {\textless} \textit{towo} `three', \textit{lo {\textless} alu} `two', \textit{nu {\textless} nu} `one'). 
 


\begin{table}[p]
\caption{Adang\ilt{Adang} cardinals\ist{cardinal numeral(s)} and distributives\ist{distributive numerals}}
\label{tab:8:3}
\begin{tabularx}{\textwidth}{XXX} 
\lsptoprule
& Cardinal\ist{cardinal numeral(s)} & Distributive\ist{distributive numerals}\\
\midrule
1 & \textit{nu} & \textit{nu{\Tilde}nu}\\
2 & \textit{alu} [alu] & \textit{al{\Tilde}lo} [al:o]{\dag} \\
3 & \textit{towo} & \textit{to{\Tilde}towo}\\
4 & \textit{{\textglotstop}}\textit{ut} & \textit{{\textglotstop}}\textit{u{\Tilde}}\textit{{\textglotstop}}\textit{ut}\\
5 & \textit{iwihing} & \textit{iw{\Tilde}wihing} \\
6 & \textit{talang} & \textit{ta{\Tilde}talang}\\
7 & \textit{wit}\textbf{\textit{to}} & \textit{wit}\textbf{\textit{to}}\textit{{\Tilde}to} \\
8 & \textit{tur}\textbf{\textit{lo}} & \textit{tur}\textbf{\textit{lo}}\textit{{\Tilde}lo} \\
9 & \textit{ti}\textit{{\textglotstop}}\textit{i}\textbf{\textit{nu}} & \textit{ti}\textit{{\textglotstop}}\textit{i}\textbf{\textit{nu}}\textit{{\Tilde}nu} \\
\lspbottomrule
\end{tabularx}

\raggedright {\dag} {Synchronically, the vowel in the distributive\ist{distributive numerals} \textit{allo} has a distinct quality from the vowel in the cardinal\ist{cardinal numeral(s)}.}

\end{table}



Across the AP family, the formation of distributives\is{distributive numerals} by reduplicating\is{reduplication} (parts of) cardinals\is{cardinal numeral(s)} is a productive process. It applies not only to frequent or morphologically simple numerals such as `one' or `two', but also to less frequent and morphologically complex numerals like `27' in \REF{bkm:Ref342661452}, `201' in \REF{bkm:Ref342656299}, and `1054' in \REF{bkm:Ref342744393}. It must be noted that, while it is difficult to imagine a distributive\is{distributive numerals} context for numerals like these, speakers are able to mechanically derive their distributive\is{distributive numerals} form. 

\ea%bkm:Ref342661452
\label{bkm:Ref342661452}
{\upshape Abui}\\
\gll Kar   ayoku   wal   yeting   ayok{\Tilde}ayok-da\\  
   ten   two   \textsc{add}   five   \textsc{rdp}\textit{{\Tilde}}two-get.\textsc{cont} \\
\glt  `in groups of 27' 
\z
 



\let\eachwordone=\upshape
\let\eachwordtwo=\itshape 
\let\eachwordthree=\itshape 
\let\eachwordfour=\itshape 
\let\eachwordfive=\itshape 

\ea%bkm:Ref342656299
\label{bkm:Ref342656299} 
{\upshape Distributive\ist{distributive numerals} for `201'} \\
\glllll  {} 100    2    \textsc{add}    \textsc{rdp}{\Tilde}1\\  
 {\upshape W Pantar\ilt{Western Pantar}:} ratu    alaku    wali     ye{\Tilde}ye        \\
 {\upshape Teiwa\ilt{Teiwa}:}    ratu    raq    rug    nuk{\Tilde}nuk      \\
 {\upshape Abui\ilt{Abui}:}     aisaha     ayoku    wal    nuk{\Tilde}nukda      \\
 {\upshape Kamang\ilt{Kamang}:}   ataak     ok     waal    no{\Tilde}nok     \\  
\z

\xbox{\textwidth}{
\ea%bkm:Ref342744393
\label{bkm:Ref342744393}
{\upshape Distributive\ist{distributive numerals} for `1054'}\\
\glllll   {}   1000    1  10  5    \textsc{add}  \textsc{rdp}{\Tilde}4 \\      
   {\upshape W Pantar\ilt{Western Pantar}:}  (a)ribu   nuk   ke  yasing     wali  atu{\Tilde}atu    \\
   {\upshape Teiwa\ilt{Teiwa}:}    ribu     nuk  qaar  yusan    rug  ut{\Tilde}ut    \\
   {\upshape Abui\ilt{Abui}:}     rifi    nuku  kar  yeting    wal  buk{\Tilde}bukna      \\
   {\upshape Kamang\ilt{Kamang}:}   ribu    nok  ataak  wesing    waal  bye{\Tilde}biat     \\
\z
}

\let\eachwordone=\itshape
\let\eachwordtwo=\upshape 
\let\eachwordthree=\upshape 
\let\eachwordfour=\upshape 
\let\eachwordfive=\upshape 
    

In sum, distributives\is{distributive numerals} are productively derived from cardinals\is{cardinal numeral(s)} by reduplicating\is{reduplication} part of or the whole cardinal\is{cardinal numeral(s)} base. In morphologically complex forms, the right-most element is the prosodic head and the reduplicative\is{reduplication} base. 


\subsubsection{Irregularities in distributive\ist{distributive numerals} formation}
\label{sec:8:Irregulardistributive}
Exceptions to the regular derivations are mainly found in the formation of the morphologically complex low numerals `six' to `nine'. The irregularities include: (i) irregular segmental changes in reduplicated\is{reduplication} forms; (ii) irregular patterns of partial vs. full reduplication\is{reduplication}; and (iii) irregular choice of reduplicative\is{reduplication} base. 

Abui\il{Abui} shows the greatest amount of formal difference between its cardinal\is{cardinal numeral(s)} and distributive\is{distributive numerals} numerals, as shown in Table~\ref{tab:8:4}. The distributives\is{distributive numerals} are reduplicated\is{reduplication} verbal constructions: their verbal status is clear from the suffixes -\textit{da/-na/-ra} which encode light verbs and (continuative) aspect \citep{Kratochvil2007}. In Table~\ref{tab:8:4}, the parts printed in bold show the irregular relation between Abui\il{Abui} cardinals\is{cardinal numeral(s)} and the numeral morphemes used in distributives\is{distributive numerals}.
 

\begin{table}


\begin{tabularx}{\textwidth}{XXX}
\lsptoprule
  & Cardinal\ist{cardinal numeral(s)} & Distributive\ist{distributive numerals}\\
\midrule
1 & \textit{nuk}\textbf{\textit{u}} & \textit{nuk{\Tilde}nuk-da}\\
2 & \textit{ayok}\textbf{\textit{u}} & \textit{ayok{\Tilde}ayok-da} \\
3 & \textit{su}\textbf{\textit{a}} & \textit{su}\textbf{\textit{i}}\textit{{\Tilde}sui-da}\\
4 & \textit{bu}\textbf{\textit{ti}} & \textit{bu}\textbf{\textit{k}}\textit{{\Tilde}buk-na}\\
5 & \textit{ye}\textbf{\textit{ting}} & \textit{ye}\textbf{\textit{k}}\textit{{\Tilde}yek-na}\\
6 & \textit{tala}\textbf{\textit{ama}} & \textit{tala}\textbf{\textit{n}}\textit{{\Tilde}talan-ra}\\
\lspbottomrule
\end{tabularx}

\caption{Abui\ilt{Abui} cardinals\ist{cardinal numeral(s)} and distributives\ist{distributive numerals}}
\label{tab:8:4}
%\label{}
\end{table}

In Kamang\il{Kamang} distributives\is{distributive numerals}, the reduplicant\is{reduplication} varies in shape. In the numerals `one' to `four' and `six', a morpheme with the shape (C)VV is reduplicated\is{reduplication}, while in the numeral `five' and the complex numerals built on it---`seven' through `nine'---the reduplicant\is{reduplication} has the shape CVCV. This is shown in Table~\ref{bkm:Ref342656818}.

\begin{table}[hb]
\caption{Kamang\ilt{Kamang} distributive\ist{distributive numerals} numeral formation}
\label{bkm:Ref342656818}
\begin{tabularx}{\textwidth}{XXXX}            
\lsptoprule
            & Cardinal\ist{cardinal numeral(s)}  &  Reduplicant\is{reduplication} shape & Distributive\ist{distributive numerals}\\
\midrule
\textit{nok} & `1'  &  CV  &  \textit{no{\Tilde}nok}\\
\textit{ok}  & `2' &  V & \textit{o}\textit{{\textglotstop}}\textit{{\Tilde}ok}\upshapefootnotemark{}\\
\textit{su}  & `3' &  CV & \textit{su{\Tilde}su}\\
\textit{biat}  & `4'  &   CVV & \textit{bie{\Tilde}biat}\\
\textit{wesing} &  `5' &  CVCV & \textit{wesi{\Tilde}wesing}\\
\textit{taama} &  `6' &  CVV & \textit{taa{\Tilde}taama}    \\
\lspbottomrule
\end{tabularx}
\end{table}





\footnotetext{The glottal stop in this form is phonetic. It is required to break up the sequence of like vowels in separate syllables. Speakers insist on including it in writing in order to distinguish /o/ from /o:/, orthographically \{oo\}.} 

Kamang\il{Kamang} has an irregular choice of reduplicative\is{reduplication} base. Compare the reduplicants\is{reduplication} (in bold-face) in the numerals `seven' to `nine' in Table~\ref{tab:ex:8:1236}. We see that Abui\il{Abui} reduplicates\is{reduplication} only the right-most numeral, resulting in word-internal reduplication\is{reduplication}, which is consistent with the regular distributive\is{distributive numerals} pattern in AP languages ({\S}~\ref{sec:8:Regulardistributive}). By contrast, Kamang\il{Kamang} reduplicates\is{reduplication} the initial element \textit{wesing}. As main stress is on the final syllable of the numerals in Kamang\il{Kamang} just as it is in Abui\il{Abui}, we analyse this as a choice of reduplicative\is{reduplication} base in Kamang\il{Kamang} distributives\is{distributive numerals} which diverges from the overall pattern of AP languages. 

\begin{table} 
\caption{Reduplication\ist{reduplication} of base-5 numerals in Abui\ilt{Abui} and Kamang\ilt{Kamang}}
\label{tab:ex:8:1236}
\begin{tabularx}{\textwidth}{XXX}
\lsptoprule
   &  Abui\ilt{Abui}   &     Kamang\ilt{Kamang}\\
\midrule
  `7' &  \textit{yeting}\textbf{\textit{ayok}}\textit{{\Tilde}ayokda}  &   \textbf{\textit{wesi}}\textit{{\Tilde}wesingok}\\
  `8' &  \textit{yeting}\textbf{\textit{sui}}\textit{{\Tilde}suida}& \textbf{\textit{wesi}}\textit{{\Tilde}wesingsu}\\
  `9' &  \textit{yeting}\textbf{\textit{buk}}\textit{{\Tilde}bukna}& \textbf{\textit{wesi}}\textit{{\Tilde}wesingbiat}\\
\lspbottomrule
\end{tabularx}
\end{table}


In sum, AP languages derive distributive numerals\is{distributive numerals} by partial or full reduplication\is{reduplication} of the cardinal\is{cardinal numeral(s)}. In complex numerals, the right-most element is the prosodic head and as a rule this item is the base for the reduplication\is{reduplication}. Exceptions to the regular derivations of distributives\is{distributive numerals} are mainly found in the formation of the morphologically complex low numerals `six' through `nine' in Central-East Alor, and include segmental changes in reduplicated\is{reduplication} forms (Abui\il{Abui}); irregular patterns of reduplication\is{reduplication} (Abui\il{Abui}, Kamang\il{Kamang}), and an irregular choice of reduplicative\is{reduplication} base (Kamang\il{Kamang}). 

\section{Structures expressing arithmetic operations} 
\label{sec:8:Structures}
To complete the catalogue of numeral expressions in AP languages, this section presents the basic arithmetic operations in which numbers are combined. We describe addition (section \ref{sec:8:Addition}), subtraction (section \ref{sec:8:Subtraction}), multiplication (section \ref{sec:8:Multiplication}), division (section \ref{sec:8:Division}), and fractions (section \ref{sec:8:Fractions}). To elicit math constructions from speakers was generally easy and not forced at all. This is remarkable in light of the fact that for none of the languages is it the case that children acquire or use these arithmetic expressions in school: the language of education in Alor-Pantar is Indonesian\il{Indonesian}. 

\subsection{Addition}
\label{sec:8:Addition}
Across Alor-Pantar, addition takes the shape of imperative sentences involving more than one verb. In such constructions, the agent or actor is not expressed and the added numerals are the arguments of verbs in a serial construction\is{serial verb construction(s)}. The number that represents the sum amount is a predicate that follows a clause-coordinating element. Languages may abbreviate the expression by omitting a verb or the clause-coordinator. Examples \REF{bkm:Ref342663714} through \REF{bkm:Ref342663723} illustrate `three plus three is six':


\ea%bkm:Ref342663714
\label{bkm:Ref342663714}
{\upshape Western Pantar}\\
\gll Atiga  ma{\upshapefootnotemark}  atiga  tang  tiggung  (allang)  hisnakkung\\  
   three   come  three  on  add  (then)  six \\
\glt  `Bring three, add on three, (then) [get] six.'
\footnotetext{\textit{Ma} may be omitted; in that case there must be a pause between both occurrences of \textit{atiga}.}
\z


\ea
\label{ex:8:1237}
 
{\upshape Teiwa}\\
 \gll Yerig  ma  yerig  taxa'   si  a  tiaam\\
   three  come  three  add  \textsc{sim}  \textsc{3sg} six \\
 \glt`Add three with three so (it's) six.'
\z
 
\ea
\label{ex:8:1238}
{\upshape Adang}\\ 
 \gll Towo  med  towo  ta  talang.\\
    three   take   three   add   six\\
 \glt `Take three add three (it's) six.'
\z

\ea
\label{ex:8:1239}
{\upshape Abui}\\
 \gll Sua  mi  sua-ng  h-ai  maiye  talaama\\
  three  take  three-\textsc{see}  \textsc{3.pat}{}-add.to  if  six \\
 \glt  `If you add three to three, (it's) six.'  
\z




\ea%bkm:Ref342663723
\label{bkm:Ref342663723}
{\upshape Kamang}\\
\gll   Su  me  su  wo-tte  an-ing=bo  taama\\  
   three   take  three  3.\textsc{loc}{}-add  thus-\textsc{set}=\textsc{conj}  six  \\
\glt `Add three to three makes six.'  
\z



 

  

\subsection{Subtraction}
\label{sec:8:Subtraction}
Just like addition, subtraction is also expressed in imperative sentences. Syntactically, the subtrahend (i.e., the numeral subtracted) is expressed as the complement of transitive verbs such as `throw away X', `split off X', `move X', or `take X'. The grammatical role of the minuend (i.e., the numeral subtracted from) is less clear. As is the case with the sum of addition, the result of the subtraction typically occurs as the predicate of a separate clause, following a clause coordinating element. Examples \REF{bkm:Ref358042906} through \REF{bkm:Ref358115306} illustrate `five minus two is three':


\ea%bkm:Ref358042906
\label{bkm:Ref358042906}
{\upshape Western Pantar}\\
\gll  Yasing  alaku  sussung  allang  (gang)  atiga.  \\  
   five  two  throw.away  then  (\textsc{3sg) } three  \\
\glt `Discard two from five then there are three.'
\z


 

\ea
\label{ex:8:1240}
{\upshape Teiwa}\\
 \gll Muxui  kam  yusan,     haraq  ma  ga-fa'   mai   ha  si,{\hspace{2cm}}  kam  yerig   qai.      \\
    banana  \textsc{clf}  five   two  come  \textsc{3sg-}split.off  save  then  \textsc{sim}  \textsc{clf} three  only    \\
 \glt `Five bananas,    split off two [to] save then only three [are left].' 
\z
 
\ea
\label{ex:8:1241}
{\upshape Adang}\\ 
\gll Iwihing  a-no'   kurung  alu  towo\\
 five  \textsc{caus}{}-affect  less  two  three\\
\glt `Five minus two is three.' 
\z  


\ea
\label{ex:8:1242}
{\upshape Abui}\\
 \gll Yeting  nu  ayoku=ng  ha-bel  maiye  he-pot  sua\\
  five  \textsc{spec}   two=\textsc{loc} 3.\textsc{pat}{}-subtract  if  \textsc{3.alien\is{alienability}}{}-remainder  three \\
 \glt `If two is subtracted from five, the remainder is three.' 
\z 
  
\ea
\label{ex:8:1243}
{\upshape Abui}\\
 \gll Yeting  nu  ayoku  mi-a  maiye  sua\\
  five  \textsc{spec } two  take-\textsc{dur } if  three   \\
 \glt `If two is taken from five, it is three.' 
\z
 
  

\ea%bkm:Ref358115306
\label{bkm:Ref358115306}
  
{\upshape Kamang}\\
\gll  Wesing  ok  wo-met  an-ing=bo  su.   \\  
    five  two   3.\textsc{loc-}take  thus\textsc{{}-set}=\textsc{conj} three   \\
\glt `Take two from five thus there are three.' 
\z



 

 

\subsection{Multiplication}
\label{sec:8:Multiplication}
The strategy used in multiplication is variable. All languages start with the multiplicant, but its shape differs. In Western Pantar\il{Western Pantar} and Teiwa\il{Teiwa} it is an underived cardinal\is{cardinal numeral(s)} followed by a demonstrative, while in Abui\il{Abui} it is a morphologically derived distributive\is{distributive numerals} ({\S}~\ref{sec:8:Distributive}). Examples \REF{bkm:Ref358043144} through \REF{bkm:Ref358043179}  illustrate `five times four is twenty':


\ea%bkm:Ref358043144
\label{bkm:Ref358043144}
{\upshape Western Pantar}\\
\gll Attu  si  gaunung  me  yasing   allang  (gang)  ke alaku. \\  
   four  that  just  on  five  then  3\textsc{sg } ten two   \\
\glt `Five on just that four then (it's) twenty.'
\z


\ea
\label{ex:8:1244}
{\upshape Teiwa}\\ 
 \gll   Ut  ga'an  tag-an  ma-yusan  si,  a   qaar raq.  \\
    four  that   count-\textsc{real } come-five  \textsc{sim}  \textsc{3sg} ten two  \\
 \glt `Count these four five times and it's twenty.'
\z



\ea
\label{ex:8:1245}
{\upshape Abui}\\ 
 \gll Buk{\Tilde}bukna  ha-lakda  nu  ming  yekna  maiye  kar ayoku. \\
  \textsc{rdp}{\Tilde}group.of.four  \textsc{3.pat}-count.\textsc{cont}  \textsc{spec}  about  five.times  if  ten two  \\
 \glt  `If a group of four is counted five times, it is twenty.'
\z

 

Kamang\il{Kamang} expresses multiplication with an applicative verb derived from a cardinal\is{cardinal numeral(s)} base by prefixing \textit{mi-}. (Compare Teiwa\il{Teiwa}, where the applicative derivation is used for fractions, see {\S}~\ref{sec:8:Fractions}). 




\ea%bkm:Ref358043179
\label{bkm:Ref358043179}
{\upshape Kamang}\\
\gll   Biat=a  mi-wesing  an-ing=bo  ataak   ok.  \\  
    four=\textsc{spec}  \textsc{appl-}five  thus\textsc{-set}=\textsc{conj} ten  two   \\
\glt  `Five times these four makes twenty.'
\z


\subsection{Division}
\label{sec:8:Division}
Expressions for division involve the transitive verbs `split' and `divide' in Western Pantar\il{Western Pantar}, Teiwa\il{Teiwa}, and Adang\il{Adang}. The following examples illustrate `ten divided by two is five': 


\ea%bkm:Ref342664493
\label{bkm:Ref342664493}
{\upshape Western  Pantar}\\
 
\gll   Ke anuku  daai  alaku  allang  yasing\\  
    ten one  split  two  then  five \\
\glt `Ten split (by) two then (it's) five.'
\z



 

 


\ea
\label{ex:8:1246}
{\upshape Teiwa}\\  
 \gll Qaar  nuk  paxai  g-et  haraq  si  yusan.   \\
  ten  one  divide  3sg-eye  two  \textsc{sim}  five   \\
 \glt `Ten divided in two parts (lit. eyes), then (it's) five.'
\z
 

\ea%bkm:Ref342664505
\label{bkm:Ref342664505}
{\upshape Adang}\\
\gll  'Air nu  'aba'ang  {\textglotstop}o-alu  iwihing\\  
    ten one  divide  \textsc{poss-}two  five \\
\glt `Ten divided (by) two (is) five.'  
\z

 

 

  

Note that the order of the verb relative to its complement `two' in \REF{bkm:Ref342664493}-\REF{bkm:Ref342746525} is unexpected, as it goes against the canonical AP object-verb order, found in subtraction ({\S}~\ref{sec:8:Subtraction}). Note that the equivalent expression in Indonesian\il{Indonesian}/Malay\il{Malay} is \textit{sepuluh bagi dua (adalah)} \textit{lima} lit. `ten divide two (is) five', with verb complement order. It may be the case that the constructions in \REF{bkm:Ref342664493}-\REF{bkm:Ref342746525} are calques from Indonesian\il{Indonesian}/Malay\il{Malay}. 

Abui\il{Abui} divisions are expressed as imperative sentences with regular serial verb constructions, where the result follows a coordinating element, see \REF{bkm:Ref358116296}. Kamang\il{Kamang} expresses a fraction by marking the dividing numeral with \textit{wo-}, the same prefix that is used to express, for instance, fractions resulting from an action, e.g., \textit{bo'ne wo-ok} `hit into two pieces', \REF{ex:8:1247}. 


\ea%bkm:Ref358116296
\label{bkm:Ref358116296}
{\upshape Abui}\\
\gll  Kar  nuku  nu  mi  ayoku  he-yeng  maiye  yek{\Tilde}yekna \\  
    ten  one  \textsc{spec}  take  two  3.\textsc{loc-}divide  if  \textsc{rdp}{\Tilde}group.of.five \\
\glt  `If a ten is divided into two (you get) a group of five.'
\z



 




\ea
\label{ex:8:1247}
{\upshape Kamang}\\
 \gll Ataak  nok=a  wo-ok  an-ing=bo  wesing \\
  ten  one=\textsc{spec}  \textsc{3.loc}-two  thus-set=\textsc{conj}  five \\
 \glt`Ten divided into two makes five.'
\z


\subsection{Fractions}
\label{sec:8:Fractions}
Expressions for fractions show much variety across the AP languages. Western Pantar\il{Western Pantar}, Teiwa\il{Teiwa} and Adang\il{Adang} express fractions using a verb, while Kamang\il{Kamang} uses fraction adverbs, and no fractions appear to exist in Abui\il{Abui}. 

Western Pantar\il{Western Pantar} derives fractions productively with the verb `divide', \REF{bkm:Ref342746525}. In Teiwa\il{Teiwa}, expressions for fractions contain an applicative verb derived from a cardinal\is{cardinal numeral(s)} base by prefixing \textit{g-un-,} a fossilized combination of a 3\textsc{sg} object prefix and an applicative prefix \textit{un-}. The fraction verb occurs as second verb in a serial verb construction, \REF{bkm:Ref342746583}: 

\newpage 
\ea%bkm:Ref342746525
\label{bkm:Ref342746525}
{\upshape Western Pantar}\\
\gll   Ye  daai  atiga,  ye   daai  attu\\  
   one   divide  three   one  divide  four \\
\glt  `one third, one fourth'   
\z

 

\ea%bkm:Ref342746583
\label{bkm:Ref342746583}
{\upshape Teiwa}\\
\gll  Taxaran  g-un-yerig,  g-un-ut,  g-un-qaar nuk \\  
   divide  \textsc{3sg}-\textsc{appl}-three  \textsc{3sg}-\textsc{appl}-four  \textsc{3sg}-\textsc{appl}-ten one   \\
\glt  `a third, a fourth, a tenth' (lit. `Divide into three, four, ten')
\z


 

\ea
\label{ex:8:1248}
{\upshape Adang}\\
\gll Nu  'aba'ang  {\textglotstop}o-ut\\
  one  divide  \textsc{poss}{}-four\\
\glt `one fourth'
\z 

  

In Kamang\il{Kamang}, fractions are verbs derived by prefixing \textit{wo-} `3.\textsc{loc}' to the numeral base, as in \REF{bkm:Ref342746224}. In \REF{bkm:Ref342746258}, the derived verb is part of a resultative serial verb construction.


\ea%bkm:Ref342746224
\label{bkm:Ref342746224}
{\upshape Kamang}\\
\gll wo-ok,    wo-su,  wo-biat,  wo-ataak     \\  
   3.\textsc{loc}-two    3.\textsc{loc}-three  3.\textsc{loc}-four  3.\textsc{loc}-ten   \\
\glt `half, a third, a fourth/quarter, a tenth' 
\z

   

 

 


\ea%bkm:Ref342746258
\label{bkm:Ref342746258}
{\upshape Kamang}\\
\gll  Nala  le  nok  katee  wo-biat.  \\  
   \textsc{1sg} mango  one  eat  3.\textsc{loc}-four    \\
\glt `I eat a fourth of the mango.', `I eat the mango in fourths.'
\z

     

 

Abui\il{Abui} does not seem to have a construction dedicated to derive fractions. It does have a word for `half' that is unrelated to `two':  


\ea
\label{ex:8:1249}
{\upshape Abui}\\
 \gll   Nalama  pingai  nuku  ahama   \\
    cooked.rice  plate  one  half   \\
 \glt  `One and a half plates of rice'  
\z

 
Words for `half' that are unrelated to `two' are also found in Western Pantar\il{Western Pantar},  Teiwa\il{Teiwa}, and Adang\il{Adang}, as shown below.  In Western Pantar\il{Western Pantar}, `half' can be a nominal \textit{gamme} `half, portion', but also a fraction involving the verb `divide', compare (\ref{bkm:Ref342746707}--\ref{bkm:Ref342746708}). In Teiwa\il{Teiwa}, `half' may be a nominal (\textit{qaas} `side, half', \textit{abaq} `half' in \ref{bkm:Ref342746924}--\ref{bkm:Ref342746926}), but may also be expressed by an applicative verb derived from `two', as in \REF{bkm:Ref342746963}.


\ea%bkm:Ref342746707
\label{bkm:Ref342746707}
{\upshape Western Pantar}\\
\gll   Gang  maggi  gamme  na  \\  
      3\textsc{sg}  banana  half  eat  \\
\glt  `He ate half a banana.'
\z








\ea%bkm:Ref342746708
\label{bkm:Ref342746708}
{\upshape Western Pantar}\\
\gll  ye  daai  alaku   \\  
    one  divide  two    \\
\glt  `half' 
\z








\ea%bkm:Ref342746924
\label{bkm:Ref342746924}
{\upshape Teiwa}\\
\gll  Ha  wou  ga'an  tu'un  qaas  na-mian \\  
     \textsc{2sg} mango  that  peel  side  \textsc{1sg}{}-give \\
\glt  `Peel that mango (and) give me half.'
\z








\ea%bkm:Ref342746926
\label{bkm:Ref342746926}
{\upshape Teiwa}\\
\gll   Yir  sluan  abaq   \\  
    water  glass  half      \\
\glt  `half a glass of water'
\z

               

              




\ea%bkm:Ref342746963
\label{bkm:Ref342746963}
{\upshape Teiwa}\\
\gll   Taxaran  g-un-raq   \\  
   divide  \textsc{3sg}-\textsc{appl}-two      \\
\glt `half' 
\z

               

               
\ea
\label{ex:8:1250}
{\upshape Adang}\\
\gll na  be  bo'oden  solo  `adi  no'o  me-nani? \\
  1\textsc{sg } mango  half  only  eat  can  or-not\\
\glt `Can I only eat half a mango?'
\z
 
 

 

\section{Summary and conclusions}
\label{sec:8:Summary}
The majority of cardinal\is{cardinal numeral(s)} numerals in AP languages are morphologically complex expressions---most are compounds. These forms have quinary or decimal bases, though mathematical operations always employ a decimal base. No AP language has a numeral `null' or a word for `zero'---the absence of entities is expressed predicatively instead.

\largerpage
Ordinals\is{ordinal numeral(s)} are derived from cardinals\is{cardinal numeral(s)} by means of a third person possessor\is{possession} morpheme. Syntactically, ordinals\is{ordinal numeral(s)} are possessive\is{possession} phrases where the ranked numeral is possessed\is{possession} by the ranked item. In the languages of Pantar, the Straits and West Alor, ordinal\is{ordinal numeral(s)} constructions also contain a dedicated ordinal\is{ordinal numeral(s)} morpheme; an applicative morpheme is used in the ordinals\is{ordinal numeral(s)} of languages of Central and East Alor. 

Most languages derive distributives\is{distributive numerals} from cardinals\is{cardinal numeral(s)} by reduplicating\is{reduplication} part or whole of the cardinal\is{cardinal numeral(s)}. In complex forms, the right-most lexeme, which is the prosodic head of the compound, is taken as the base for the reduplication\is{reduplication}. This applies even to those forms that are synchronically morphologically opaque. Kamang\il{Kamang} is exceptional in that it reduplicates\is{reduplication} the left-most element of the compound rather than the prosodic head, and in Abui\il{Abui}, distributives\is{distributive numerals} and cardinals\is{cardinal numeral(s)} are only indirectly related. 

Across the languages, there is more homogeneity in the expressions of addition and subtraction than there is in the expression of multiplication and division. Addition and subtraction typically take the shape of imperative sentences. In additive expressions, the added numerals each have their own predicate. The second numeral is often the grammatical object of a transitive verb (`add X') that has an implied subject, the imperative addressee. In subtraction, the subtrahend is also the object of a transitive verb (`throw away X') but the grammatical role of the `minuend' is less clear. In both addition and subtraction, the result follows a clause coordinating element.

The strategies used in multiplication, division, and fractions vary significantly across the languages. While all the languages express multiplication by a multiplicant followed by a verb, the morpho-syntactic shape of the multiplicant and the choice of verb differ. In expressions for division, the number of verbs involved range from zero to two, and word orders in the western languages go against the head-final order that is typical for AP and follow the order of Indonesian\il{Indonesian}/Malay\il{Malay}, suggesting they may be calques. Across the AP languages, the expression of fractions shows the largest variety. The lack of homogeneity in the expressions for multiplication, division and fractions suggests that these expressions are more labile than those for addition and subtraction, which is probably due to their lower frequency in everyday language.

The indigenous numeral forms of the AP languages, as well as the indigenous structures for arithmetic operations are currently under pressure from Indonesian\il{Indonesian} as the language of interethnic trade and national education. This will inevitably lead to their replacement with Indonesian\il{Indonesian} forms and constructions. This chapter keeps a snapshot of them for future generations. 

\subsection{Sources}
\label{sec:8:Sources}
The data sets on which this paper is based were collected from 2010-2012 by the authors. We used a questionnaire on numerals designed in 2010 by Marian Klamer and Antoinette Schapper for the purpose of documenting the numerals and numeral systems in AP languages (see Appendix B). The core dataset discussed in this chapter thus comes from questionnaires filled in for Teiwa\il{Teiwa} (by Klamer and Robinson), Western Pantar\il{Western Pantar} (by Holton), Adang\il{Adang} (by Robinson), Abui\il{Abui} (by Kratochv\'il and Schapper), and Kamang\il{Kamang} (by Schapper). Comparative information on additional languages was provided through personal communication with Hein Steinhauer (Blagar\il{Blagar}), Nick Williams (Kula\il{Kula}), Franti{\v{s}}ek Kratochv\'il (Sawila\il{Sawila}) and Marian Klamer (Kaera\il{Kaera}).

\clearpage
\startappendix\label{sec:8:app}
\subsection{Ordinal\ist{ordinal numeral(s)} and Distributive Numerals\ist{distributive numerals}} \label{sec:8:app:1}
\begin{table}[h!]

 \caption{Western Pantar\ilt{Western Pantar} ordinals\ist{ordinal numeral(s)} in a construction with \textit{bla} `house' and \textit{aname} `person'}
\label{tab:8:5}
\begin{tabularx}{\textwidth}{rXXXl}
\lsptoprule
\multirow{2}{*}{1\textsuperscript{st}} & \textit{bla/aname} & \textit{gai} & \textit{maing} & \textit{ye}\\
 & \textit{bla/aname} & \textit{gai} & \textit{maing} & \textit{anuku}\\
2\textsuperscript{nd} & \textit{bla/aname} & \textit{gai} & \textit{maing} & \textit{alaku}\\
3\textsuperscript{rd} & \textit{bla/aname} & \textit{gai} & \textit{maing} & \textit{atiga}\\
4\textsuperscript{th} & \textit{bla/aname} & \textit{gai} & \textit{maing} & \textit{at\'u}\\
5\textsuperscript{th} & \textit{bla/aname} & \textit{gai} & \textit{maing} & \textit{yasing}\\
6\textsuperscript{th} & \textit{bla/aname} & \textit{gai} & \textit{maing} & \textit{hisnakkung}\\
7\textsuperscript{th} & \textit{bla/aname} & \textit{gai} & \textit{maing} & \textit{betalaku}\\
8\textsuperscript{th} & \textit{bla/aname} & \textit{gai} & \textit{maing} & \textit{betiga}\\
9\textsuperscript{th} & \textit{bla/aname} & \textit{gai} & \textit{maing} & \textit{anuku tannang}\\
10\textsuperscript{th} & \textit{bla/aname} & \textit{gai} & \textit{maing} & \textit{ke anuku}\\
100\textsuperscript{th} & \textit{bla/aname} & \textit{gai} & \textit{maing} & \textit{ratu}\\
\lspbottomrule
\end{tabularx}
\end{table}

\begin{table}[h!]

\caption{Teiwa\ilt{Teiwa} ordinals\ist{ordinal numeral(s)} with \textit{yaf} `house' and \textit{uy} `person' }
\label{tab:8:6}
\begin{tabularx}{\textwidth}{rXXXl}
\lsptoprule
1\textsuperscript{st} & \textit{yaf/uy} & \textit{ga-} &  & \textit{xol}\dag{}\\
\multirow{2}{*}{2\textsuperscript{nd}} & \textit{yaf/uy} & \textit{ga-} & \textit{ma-} & \textit{ga-mar} [gama'gamar] `3s-\textsc{ord}-3s-take' \\
 & \textit{yaf/uy} & \textit{ga-} & \textit{ma-} & \textit{raq} \\
3\textsuperscript{rd} & \textit{yaf/uy} & \textit{ga-} & \textit{ma-} & \textit{yerig}\\
4\textsuperscript{th} & \textit{yaf/uy} & \textit{ga-} & \textit{ma-} & \textit{ut}\\
5\textsuperscript{th} & \textit{yaf/uy} & \textit{ga-} & \textit{ma-} & \textit{yusan} \\
6\textsuperscript{th} & \textit{yaf/uy} & \textit{ga-} & \textit{ma-} & \textit{tiaam} \\
7\textsuperscript{th} & \textit{yaf/uy} & \textit{ga-} & \textit{ma-} & \textit{yes raq} \\
8\textsuperscript{th} & \textit{yaf/uy} & \textit{ga-} & \textit{ma-} & \textit{yes nerig} \\
9\textsuperscript{th} & \textit{yaf/uy} & \textit{ga-} & \textit{ma-} & \textit{yes na}\textit{{\textglotstop}}\textit{ut} \\
10\textsuperscript{th} & \textit{yaf/uy} & \textit{ga-} & \textit{ma-} & \textit{qaar nuk} \\
100th & \textit{yaf/uy} & \textit{ga-} & \textit{ma-} & \textit{ratu nuk} \\
\lspbottomrule
\end{tabularx}


\dag{Teiwa\ilt{Teiwa} \textit{ga-nuk} means `one from a group', \textit{ga-ma-nuk} is not a Teiwa\ilt{Teiwa} word.}

\end{table}

\begin{table}
\caption{Kaera\ilt{Kaera} ordinals\ist{ordinal numeral(s)} with \textit{ma} `house' and \textit{ui} `person'}
\label{tab:8:7}
\begin{tabularx}{\textwidth}{rllll}
\lsptoprule
1\textsuperscript{st} & \textit{ma/ui} & \textit{(ge-)} &  & \textit{tuning } (\textit{tuni} `gate', \textit{tuning} `placenta')\\
2\textsuperscript{nd} & \textit{ma/ui} & \textit{ge-} & \textit{mi} & \textit{(a)raxo}\\
3\textsuperscript{rd} & \textit{ma/ui} & \textit{ge-} & \textit{mi} & \textit{(u)tug}\\
4\textsuperscript{th} & \textit{ma/ui} & \textit{ge-} & \textit{mi} & \textit{ut}\\
5\textsuperscript{th} & \textit{ma/ui} & \textit{ge-} & \textit{mi} & \textit{isim}\\
6\textsuperscript{th} & \textit{ma/ui} & \textit{ge-} & \textit{mi} & \textit{tiam}\\
7\textsuperscript{th} & \textit{ma/ui} & \textit{ge-} & \textit{mi} & \textit{yesraxo} \\
8\textsuperscript{th} & \textit{ma/ui} & \textit{ge- } & \textit{mi} & \textit{yentug}\\
9\textsuperscript{th} & \textit{ma/ui} & \textit{ge-} & \textit{mi} & \textit{yeniut}\\
10\textsuperscript{th} & \textit{ma/ui} & \textit{ge-} & \textit{mi} & \textit{xar nuko}\\
100\textsuperscript{th} & \textit{ma/ui} & \textit{ge-} & \textit{mi} & \textit{ratu nuko}\\
\lspbottomrule
\end{tabularx}

\end{table}

\begin{table}
\caption{Adang\ilt{Adang} ordinals\ist{ordinal numeral(s)} with \textit{bang} `house' and \textit{nami} `person'}
\label{tab:8:8}
\begin{tabularx}{\textwidth}{rXXXXX}
\lsptoprule
1\textsuperscript{st} & \textit{bang/nami} & \textit{{\textglotstop}o-} & \textit{mi-} & \textit{nu} & \textit{mi}\\
2\textsuperscript{nd} & \textit{bang/nami} & \textit{{\textglotstop}o-} & \textit{mi-} & \textit{alu} & \textit{mi}\\
3\textsuperscript{rd} & \textit{bang/nami} & \textit{{\textglotstop}o-} & \textit{mi-} & \textit{towo} & \textit{mi}\\
4\textsuperscript{th} & \textit{bang/nami} & \textit{{\textglotstop}o-} & \textit{mi-} & \textit{ut} & \textit{mi}\\
5\textsuperscript{th} & \textit{bang/nami} & \textit{{\textglotstop}o-} & \textit{mi-} & \textit{(i)wihing} & \textit{mi}\\
6\textsuperscript{th} & \textit{bang/nami} & \textit{{\textglotstop}o-} & \textit{mi-} & \textit{talang} & \textit{mi}\\
7\textsuperscript{th} & \textit{bang/nami} & \textit{{\textglotstop}o-} & \textit{mi-} & \textit{witto} & \textit{mi}\\
8\textsuperscript{th} & \textit{bang/nami} & \textit{{\textglotstop}o-} & \textit{mi-} & \textit{turlo} & \textit{mi}\\
9\textsuperscript{th} & \textit{bang/nami} & \textit{{\textglotstop}o-} & \textit{mi-} & \textit{ti}\textit{{\textglotstop}}\textit{inu} & \textit{mi}\\
10\textsuperscript{th} & \textit{bang/nami} & \textit{{\textglotstop}o-} & \textit{mi-} & \textit{{\textglotstop}}\textit{{\quotesinglbase}}\textit{air nu} & \textit{mi}\\
100\textsuperscript{th} & \textit{bang/nami} & \textit{{\textglotstop}o-} & \textit{mi-} & \textit{rat nu} & \textit{mi}\\
\lspbottomrule
\end{tabularx}

\end{table}

\begin{table}
\caption{Abui\ilt{Abui} ordinals\ist{ordinal numeral(s)} with \textit{fala} `house' and \textit{ama} `person'}
\label{tab:8:9}
\begin{tabularx}{\textwidth}{rXXX}
\lsptoprule
1\textsuperscript{st} & \textit{fala/ama} & \textit{he-} & \textit{teitu}\\
 & \textit{fala/ama} & \textit{he-} & \textit{nuku} \\
2\textsuperscript{nd} & \textit{fala/ama} & \textit{he-} & \textit{ayoku}\\
3\textsuperscript{rd} & \textit{fala/ama} & \textit{he-} & \textit{sua}\\
4\textsuperscript{th} & \textit{fala/ama} & \textit{he-} & \textit{buti}\\
5\textsuperscript{th} & \textit{fala/ama} & \textit{he-} & \textit{yeting}\\
6\textsuperscript{th} & \textit{fala/ama} & \textit{he-} & \textit{talaama}\\
7\textsuperscript{th} & \textit{fala/ama} & \textit{he-} & \textit{yeting ayoku}\\
8\textsuperscript{th} & \textit{fala/ama} & \textit{he-} & \textit{yeting sua}\\
9\textsuperscript{th} & \textit{fala/ama} & \textit{he-} & \textit{yeting buti}\\
10\textsuperscript{th} & \textit{fala/ama} & \textit{he-} & \textit{kar nuku}\\
100\textsuperscript{th} & \textit{fala/ama} & \textit{he-} & \textit{aisaha nuku}\\
\lspbottomrule
\end{tabularx}
\end{table}

\begin{table}
\caption{Kamang\ilt{Kamang} ordinals\ist{ordinal numeral(s)} for \textit{kadii} `house' and \textit{alma} `person'}
\label{tab:8:10}
\begin{tabularx}{\textwidth}{rXXX}
\lsptoprule
1\textsuperscript{st} & \textit{kadii / alma} & \textit{ye-} & \textit{nok}\\
2\textsuperscript{nd} & \textit{kadii / alma} & \textit{ye-} & \textit{ok}\\
3\textsuperscript{rd} & \textit{kadii / alma} & \textit{ye-} & \textit{su}\\
4\textsuperscript{th} & \textit{kadii / alma} & \textit{ye-} & \textit{biat}\\
5\textsuperscript{th} & \textit{kadii / alma} & \textit{ye-} & \textit{wesing}\\
6\textsuperscript{th} & \textit{kadii / alma} & \textit{ye-} & \textit{taama}\\
7\textsuperscript{th} & \textit{kadii / alma} & \textit{ye-} & \textit{wesing ok}\\
8\textsuperscript{th} & \textit{kadii / alma} & \textit{ye-} & \textit{wesing su}\\
9\textsuperscript{th} & \textit{kadii / alma} & \textit{ye-} & \textit{wesing biat}\\
10\textsuperscript{th} & \textit{kadii / alma} & \textit{ye-} & \textit{ataak nok}\\
100\textsuperscript{th} & \textit{kadii / alma} & \textit{ye-} & \textit{asaka nok}\\
\lspbottomrule
\end{tabularx}
\end{table}

\begin{table}
\caption{Sawila\ilt{Sawila} ordinals\ist{ordinal numeral(s)} with \textit{araasing} `house' and \textit{imyalara} `man' }
\label{tab:8:11}
\begin{tabularx}{\textwidth}{rXXX}
\lsptoprule
1\textsuperscript{st} & \textit{araasing/imyalara} & \textit{wii-} & \textit{suna}\\
2\textsuperscript{nd} & \textit{araasing/imyalara} & \textit{wii-} & \textit{yaku}\\
3\textsuperscript{rd} & \textit{araasing/imyalara} & \textit{wii-} & \textit{tuo}\\
4\textsuperscript{th} & \textit{araasing/imyalara} & \textit{wii-} & \textit{araasiiku}\\
5\textsuperscript{th} & \textit{araasing/imyalara} & \textit{wii-} & \textit{yooting}\\
6\textsuperscript{th} & \textit{araasing/imyalara} & \textit{wii-} & \textit{yootsuna}\\
7\textsuperscript{th} & \textit{araasing/imyalara} & \textit{wii-} & \textit{yootingyaku}\\
8\textsuperscript{th} & \textit{araasing/imyalara} & \textit{wii-} & \textit{yootingtuo}\\
9\textsuperscript{th} & \textit{araasing/imyalara} & \textit{wii-} & \textit{yootingaraasiiku}\\
10\textsuperscript{th} & \textit{araasing/imyalara} & \textit{wii-} & \textit{adaaku}\\
100\textsuperscript{th} & \textit{araasing/imyalara} & \textit{wii-} & \textit{asaka}\\
\lspbottomrule
\end{tabularx}
\end{table}

\begin{table}
\caption{Distributive numerals\ist{distributive numerals} in Pantar-West Alor languages}
\label{tab:8:12}
\begin{tabularx}{\textwidth}{rlll} 
\lsptoprule
& Western Pantar\ilt{Western Pantar} & Teiwa\ilt{Teiwa} & Adang-Lawahing\ilt{Adang}\\
\midrule
1 & \textit{ye{\Tilde}ye} & \textit{nuk{\Tilde}nuk} & \textit{nu-nu}\\
2 & \textit{alaku{\Tilde}alaku} & \textit{raq{\Tilde}raq} & \textit{al-lo} \\
3 & \textit{atiga{\Tilde}atiga} & \textit{yerig{\Tilde}yerig} & \textit{to-towo}\\
4 & \textit{atu{\Tilde}atu} & \textit{{\textglotstop}}\textit{ut{\Tilde}}\textit{{\textglotstop}}\textit{ut} & \textit{u-ut}\\
5 & \textit{yasing{\Tilde}yasing} & \textit{yusan{\Tilde}yusan} & \textit{iw-wihing}\\
6 & \textit{hisnakkung{\Tilde}nakkung} & \textit{tiaam{\Tilde}tiaam} & \textit{ta-talang}\\
7 & \textit{betalaku}\textit{{\Tilde}}\textit{talaku} & \textit{yesraq{\Tilde}raq} & \textit{witto-to} \\
8 & \textit{betiga}\textit{{\Tilde}}\textit{tiga} & \textit{yesnerig{\Tilde}rig} & \textit{turlo-lo} \\
9 & \textit{anuktannang{\Tilde}tannang} & \textit{yesna}\textit{{\textglotstop}}\textit{ut{\Tilde}}\textit{{\textglotstop}}\textit{ut} & \textit{ti'inu-nu} \\
10 & \textit{ke anuku{\Tilde}nuku} & \textit{qaar nuk{\Tilde}nuk} & \textit{{\textglotstop}}\textit{air nu-nu}\\
11 & \textit{ke anuku wali ye{\Tilde}ye} & \textit{qaar nuk rug nuk{\Tilde}nuk} & \textit{{\textglotstop}}\textit{air nu waling nu-nu}\\
100 & \textit{ratu}\textit{{\Tilde}}\textit{ratu} & \textit{ratu nuk{\Tilde}nuk} & \textit{rat nu-nu}\\
1000 & \textit{aribu}\textit{{\Tilde}}\textit{aribu} & \textit{ribu nuk{\Tilde}nuk} & \textit{rib nu-nu}\\
\lspbottomrule
\end{tabularx}
\end{table}



\begin{table}[t]

\caption{Distributive numerals\ist{distributive numerals} in Central-East Alor languages}
\label{tab:8:13}
\begin{tabularx}{\textwidth}{rXX}

\lsptoprule
& Abui\ilt{Abui} & Kamang\ilt{Kamang}\\
\midrule
1 & \textit{nuk{\Tilde}nukda} & \textit{no{\Tilde}nok, nokda{\Tilde}nokda}\\
2 & \textit{ayok{\Tilde}ayokda} & \textit{o{\Tilde}ok } \\
3 & \textit{sui{\Tilde}suida} & \textit{su{\Tilde}su}\\
4 & \textit{buk{\Tilde}bukna} & \textit{bye{\Tilde}biat}\\
5 & \textit{yek{\Tilde}yekna} & \textit{wesi{\Tilde}wesing}\\
6 & \textit{talan{\Tilde}talanra} & \textit{taa{\Tilde}taama}\\
7 & \textit{yeting ayok{\Tilde}ayokda} & \textit{wesi{\Tilde}wesingok}\\
8 & \textit{yeting sui{\Tilde}suida} & \textit{wesi{\Tilde}wesingsu}\\
9 & \textit{yeting buk{\Tilde}bukna} & \textit{wesi{\Tilde}wesingbiat}\\
10 & \textit{kar nuk{\Tilde}nukda} & \textit{ataak no{\Tilde}nok}\\
11 & \textit{kar nuku wal nuk{\Tilde}nukda} & \textit{ataak nok waal no{\Tilde}nok}\\
100 & \textit{aisaha nuk}\textit{{\Tilde}}\textit{nukda} & \textit{asaka no}\textit{{\Tilde}}\textit{nok}\\
1000 & \textit{rifi nuk}\textit{{\Tilde}}\textit{nukda} & \textit{ribu no}\textit{{\Tilde}}\textit{nok}\\
\lspbottomrule
\end{tabularx} 
\end{table}


\clearpage
\subsection{Numeral Questionnaire used in the field}
\subsubsection{Numerals} 
It is preferred to elicit the data for this questionnaire using words and constructions in the language of investigation as much as possible. The Malay\il{Malay} examples below are not given as prompts to be translated, but rather as additional background for you to help you steer a discussion in Malay\il{Malay}. Expressions containing numerals and ordinals\is{ordinal numeral(s)}, and morphological derivations relating to numerals and ordinals\is{ordinal numeral(s)} in the AP languages are expected to be quite different from what they are in Malay\il{Malay}. 

\subsubsection{Tasks}
\begin{enumerate}
\item Ask a person to count in sequence from 1-20 and record this.
\item Elicit 1-100 on paper. Appendix 1: answer sheet.
\item Elicit higher cardinals\ist{cardinal numeral(s)} 2000, 3000,..., 10.000. Appendix 2: answer sheet. 
\item Elicit 100-1000 on paper. Suggestion: You could give (a) speaker(s) an empty  notebook to work on this at their leisure at home. After they have written up all the numbers, please go over it, to check   
  \begin{itemize}
  \item if the writing is legible 
  \item if you know which letter is used for which sound
  \item if this letter-sound correspondence in their orthography is consistent (or consistent enough to be used by us) 
  \item if there are any (possible) morphemes or morpheme boundaries that need additional elicitation or discussion --these notes can go with the manuscript.
  \end{itemize}
\item Elicit expressions for basic calculations if any exist: 
  \begin{itemize}
  \item 3 + 3 = 6: \textit{3 tambah 3 sama dengan enam}
  \item 5 -- 2 = 3: \textit{lima kurang dua sama dengan tiga}
  \item 4 x 5 = 20: \textit{empat kali lima sama dengan dua puluh}
  \item 10 : 2 = 5: \textit{sepuluh bagi 2 sama dengan lima} 
  \end{itemize}
\item If expressions for basic calculations don't exist, or if they are borrowed\is{borrowing} or calqued from Malay\ilt{Malay}, can consultants think of any other strategies how such basic calculations can be done? Situations to suggest could include: 
  \begin{itemize}
  \item talking about the number of children alive in a family (e.g. 8 children born, 3 died as babies, 5 are still alive), 
  \item counting / adding / subtracting pupils in a class setting
  \item cigarettes in a packet
  \item members of the church who have newly arrived / have left / died
  \item multiplying/dividing rupiahs earned by a group of people
  \item measuring land to buy or sell e.g. to build a house on
  \item etc. 
  \end{itemize}
\item Elicit the years 1978, 1999, 2010. If there is no consensus or consistency across speakers, please note down any differences you notice. 
\item Elicit fractions, if they exist 
\begin{itemize}
  \item half 
  \item one third 
  \item quarter 
  \item try smaller fractions? 
  \item a tenth 
\end{itemize}
Please ask for examples in context, e.g. \textit{Saya bisa makan setengah buah manggo saja} `I can only eat half a mango',\textit{Tolong berikan sepertiga/seperempat (bagian) saja} `Please give me a third/quarter only'.
\item If expressions for fractions don't exist, can consultants think of other ways to talk about parts of fruits, subgroups of people, parts of piece of land?
\item Ordinals\ist{ordinal numeral(s)}: Elicit 1\textsuperscript{st}{}-10\textsuperscript{th.} e.g., \textit{Saya lihat barisan anak di muka rumah. Yang pertama bernama... yang kedua... yang ketiga...} etc.

Please try also for higher ones: contrast \textit{Anggota gereja yang ketiga} `the third member of the church'  with \textit{anggota yang kesepuluh, yang keratus, yang keseribu} ... It is best to use a local language prompt here, as the higher ones are ungrammatical in Malay\il{Malay}!

\end{enumerate}

\newpage
\subsubsection{Points for further elicitation}

\begin{enumerate}
\item Is there a word for zero?
\item Is there an indigenous word for million/\textit{jutah}? 
\item Are there indigenous numbers higher than million?
\item Distinguish non-referential counting (1, 2, 3, {\dots} 10) and enumeration (\textit{satu ekor ayam, dua orang, tiga buku, sepuluh rumah}): are different numerals used?
\item Check if there is a contrast in counting small animates versus large animates and animate vs. inanimate entities: 
  \begin{itemize}
  \item Small animates 
    \begin{itemize}
    \item ant/\textit{semut}  
    \item fly/\textit{lalat} 
    \item bee/\textit{lebah} 
    \item house lizzard/\textit{cecak}   
    \end{itemize}
  \item Large animates  
    \begin{itemize}
    \item child/\textit{anak}  
    \item dog/\textit{anjing} 
    \item pig/\textit{babi}  
    \end{itemize}
  \item Inanimates 
    \begin{itemize}
    \item house/\textit{rumah}  
    \item rock/\textit{batu karang}  
    \item star/\textit{bintang}  
    \item coconut tree/\textit{pohon kelapa} 
    \end{itemize} 
  \end{itemize}
\item Note down the distribution of cardinals\ist{cardinal numeral(s)} as part of NP (in `attributive' function), for example in a context like:

\textit{Ada tiga orang di rumah.} \textit{Dua orang pergi ke kota, satu orang tinggal di rumah.} 
`There are three people at home. Two went to town, one stayed at home.' 


\begin{itemize}
\item Is the position of numeral w.r.t. noun fixed or is there variability? E.g. \textit{Orang tiga vs tiga orang} in the above example.
\item If there is variability, check if it is related to higher vs. lower cardinals\ist{cardinal numeral(s)}. E.g. Malay\il{Malay} 
\begin{itemize}
\item \textit{Ada dua orang di rumah} \textbf{\textit{vs}}\textit{. ada} \textit{orang dua di rumah}
\item \textit{Ada} \textit{sebelas orang di rumah} \textbf{\textit{vs}} \textit{ada orang sebelas di rumah}
\item \textit{Ada lima puluh orang di rumah} \textbf{\textit{vs.}} \textit{ada orang lima puluh di rumah}
\end{itemize}
\item What is the position of the numeral in the NP if it contains a demonstrative? E.g. \textit{Those five girls... } 
\begin{itemize}
\item \textit{Dua orang itu ada di rumah, Orang dua itu ada di rumah,} \textit{Sebelas orang itu ada di rumah, Orang sebelas itu ada di rumah,} etc.
\end{itemize}
\end{itemize}
\item Is there any agreement morphology between numeral and noun?
\item Note down the distribution of cardinals\ist{cardinal numeral(s)} as predicate (in `predicative' function), if they are used as such, e.g.:  
\begin{itemize}
\item \textit{Waktu itu kami} \textit{masih bertiga} `At that time we were still three'; 
\item \textit{Mereka datang berlima}, \textit{berdua mereka pergi} `They came with five and left with two'
\end{itemize}
\item If cardinals\ist{cardinal numeral(s)} may be used in predicative function, can a higher numeral also be used as such? Note that this not generally possible in Malay\il{Malay}, where the predicative \textit{ber-} construction is not productively used with higher numerals: \textit{*Waktu itu kami berdua puluh}. Instead one would say \textit{Waktu itu kami duapuluh orang} `We were twenty at the time'. 

\begin{itemize}
\item Check e.g. 12, 15, 20, 35, 50, 76, 95. 
\end{itemize}
\item In Malay\ilt{Malay}, certain particular high cardinals\ist{cardinal numeral(s)} do appear in the \textit{ber-} construction: \textit{Kami akan datang berseribu} `we will come (as) a (group of) thousand'. So perhaps a language does not treat all higher cardinals\ist{cardinal numeral(s)} in the same way. 
\begin{itemize}
\item Check e.g. 1000, 2000, 100, 500, 1 000 000, 2 000 000
\end{itemize}
\item Can cardinals\ist{cardinal numeral(s)} be used as elliptical for a fuller NP (subject or object): \textit{Mau berapa buah pisang? Saya mau} \textit{dua (dua buah/dua pisang)}
\item Can cardinals\ist{cardinal numeral(s)} be used as abstract entities, e.g. in contexts like: 
\begin{itemize}
\item \textit{Nomor HP saya mulai dengan angka/nomor tiga} `My mobile phone number starts with digit/number three' 
\item \textit{Waktu mengajar anak menulis guru bilang:} ``\textit{Coba menulis angka/nomor dua dan angka dua belas sekarang''}  `When the teacher taught the children to write, he said: `` Please write digit/number two and number/digit 12 now'.
\end{itemize}
Try the same with some higher numerals: 
\begin{itemize}
\item Guru bilang kepada anak: \textit{Angka dua puluh itu masih terlalu kecil } `that number 20 is still too small' 
\end{itemize}
Try the same for \textit{angka lima belas, tiga puluh, seratus, seribu, dua ribu, (se)jutah}.
\item  \textbf{Reduplication\ist{reduplication} of cardinals\ist{cardinal numeral(s)}:} Can numerals be reduplicated\ist{reduplication}? If so, give some examples in sentential context. 
\begin{itemize}
\item Try 1, 2, 3, 4, 5, 10, 12, 17, 15, 20, 50, 100, 500, 1000.
\end{itemize}
\item What does the reduplication\ist{reduplication} mean? E.g. Malay\il{Malay} \textit{beribu-ribu orang datang ke kota itu} `People came in thousands to that town' (vs. \textit{ribuan orang} `thousands of people') 
\item Does reduplicated\ist{reduplication} `one' have any special meaning? E.g. Bunaq \textit{uen{\Tilde}uen} means `same, equal'; Kamang\ilt{Kamang} \textit{no-nok} `one by one'.
\item Where do numeral reduplications\ist{reduplication} occur: before or after the noun? Before or after the verb? Please provide some example sentences.
\item Do reduplicated\ist{reduplication} numerals occur as part of NPs in `attributive' function (as in Malay\il{Malay} \textit{beribu-ribu orang})? Or do they occur in `predicative' function? 
\item  Check reduplication\ist{reduplication} of NPs encoding subject/actor vs NPs encoding object/undergoer: 
\begin{itemize}
\item \textit{Dua orang laki-laki membawa papan.} \textit{Satu demi satu mereka membawa papan} = one carrier at the time vs. 
\item \textit{Dua orang laki-laki membawa papan.} \textit{Mereka membawa papan} \textit{satu demi satu} = one plank at the time
\end{itemize}
\item  Note down the \textbf{distribution of Ordinals\ist{ordinal numeral(s)}}: 
\begin{itemize}
\item as part of NP: 
\begin{itemize}
\item \textit{Orang pertama yang membeli tv adalah Markus} `The first person to buy a radio was Markus'  
\end{itemize}
\item as sth. similar to a non-verbal predicate: 
\begin{itemize}
\item \textit{Lidia} \textit{adalah} \textit{orang pertama yang pergi ke Kupang} `Lidia was the first person who went to Kupang' 
\end{itemize}
\item with an inanimate noun: 
\begin{itemize}
\item \textit{Mereka masuk} \textit{jalan kedua}
\item \textit{Kepala desa membangunkan} \textit{rumahnya kedua}\textit{(}or \textit{rumah keduanya}\textit{) pada tahun yang lalu } 
\end{itemize}
\item modifying the predicate, in adverbial-like function: 
\begin{itemize}
\item \textit{Mereka pergi ke Kupang} \textit{pada kali yang kedua}.
\end{itemize}
\end{itemize}
\item  Are the ordinals\ist{ordinal numeral(s)} etymologically clearly related to cardinals\ist{cardinal numeral(s)}? e.g. Indonesian\ilt{Indonesian} ordinal\ist{ordinal numeral(s)} \textit{pertama} is not derived from cardinal\ist{cardinal numeral(s)} \textit{satu.}
\item  Are there any words that are used like ordinals\ist{ordinal numeral(s)} but have no numeral or ordinal\ist{ordinal numeral(s)} root?
\item  \textbf{Plural marking}: Is plural marked with an affix? 
\item  Does the language have a plural word? E.g. \textit{non} `PLURAL' in Teiwa.

A plural word is a morpheme whose meaning and function is similar to that of plural affixes in other languages, but which is a separate word that functions as a modifier of the noun.  Plural words are overrepresented in isolating or analytic languages, in languages with classifiers\is{numeral classifier}, and in head-marking languages (cf. M. Dryer, Plural words, \textit{Linguistics} 27 (1989), 865-895.) \nocite{Dryer1989}

\textbf{\textit{Questions 24-29 only apply when the language has a plural word:}}

\item  If the language has a plural word, do you observe any animacy or size effects in the use of the plural word? Check:

\begin{itemize}
\item \textit{orang perempuan} 
\item \textit{kakak perempuan} 
\item \textit{anak laki-laki} 
\item \textit{babi, anjing} 
\item \textit{tikus, burung} 
\item \textit{nyamuk, semut, lebah, lalat} 
\item \textit{batu kecil, jarum, kancing}
\item \textit{kendi, panci, mok}
\item \textit{batu karang, pohon kelapa}
\item \textit{bintang}
\item \textit{rumah}
\end{itemize}
\item  Plural words as `numerals': Can plural word and numeral co-occur? (If so, this could be evidence that they belong to different categories.)

\item  Can plural word and non-numeral quantifiers (\textit{beberapa, semua, sedikit, banyak}) co-occur? 
\item Can plural word and possessor\ist{possession} noun co-occur? 
\item  Can plural word and possessive\ist{possession} prefix co-occur? 
\item  Plural words are reported to derive from e.g. third person plural\ist{plurality} pronoun\ist{pronoun}, plural article\is{plurality}, words meaning \textit{all} or \textit{many}, nouns meaning \textit{group} or \textit{set}, classifier\is{numeral classifier},{\dots} etc. Do you have ideas about the possible diachronic origin of the plural\ist{plurality} word in the language of study?
\item  \textbf{Quantifiers (non-numeral)} \textit{semua, banyak, sedikit, beberapa}
What does the quantifier inventory look like for
\begin{itemize}
\item Countable objects 
\begin{itemize}
\item \textit{orang, babi, anjing, rumah, kursi, gelas}
\end{itemize}
\item Uncountable objects or masses 
\begin{itemize}
\item \textit{garam, gula, air, nasi, jagung (?), semut, lebah, lalat,} 
\item \textit{gunung-gunung (?), awan-awan (?)}
\end{itemize}
\item Liquids 
\begin{itemize}
\item \textit{air, air susu, anggur, arak, teh}
\end{itemize}
\item Edibles 
\begin{itemize}
\item \textit{buah pinang, daun papaya, daging babi, ikan}  
\end{itemize}
\end{itemize}
\item  Do particular semantics play a role in the interpretation of the value of the quantifiers? (e.g. (un)expected/(un)wanted value, e.g. many people come to church, more than expected, or when only a little bit of gas is sold less than expected (\textit{misalnya kalau banyak orang datang ke g<ereja, lebih dari harapan} (\textit{atau} \textit{hanya} \textit{sedikit minyak dijual}, \textit{kurang dari harapan).}
\item  \textbf{Classifiers\ist{numeral classifier}}: We will make a separate questionnaire \& stimuli for this at a later stage. If you have made some observations about the classifiers\ist{numeral classifier}, please include them here. 
\end{enumerate}

\subsubsection{Numerals 1--100}

\begin{tabularx}{\textwidth}{rll}
\lsptoprule

No & Language: .... & Notes\\
\midrule
1 &  & \\
2 &  & \\
{\dots} & & \\
99 &  & \\
100 &  & \\
\lspbottomrule
\end{tabularx}
 

\subsubsection{Higher cardinals\ist{cardinal numeral(s)}}

\begin{tabularx}{\textwidth}{rll}
\lsptoprule
No & Language: ...... & Notes\\
\midrule
1000 &  & \\
2000 &  & \\
{\dots} & & \\
9000 &  & \\
10 000 &  & \\
\lspbottomrule
\end{tabularx}

\largerpage[3]
\section*{Abbreviations}
\begin{tabularx}{\textwidth}{>{\sc}lX>{\sc}lX>{\sc}lX}
= & clitic boundary &             conj & conjunction &        pl & plural\\
{\Tilde} & reduplication &        cont & continuous &         poss & possessive\\
1 & 1st person &                  def & definite &            pred & predicate\\
2 & 2nd person &                  dur & durative &            rdp & reduplication\\
3 & 3rd person &                  ind & Indonesian &     sg & singular\\
alien & alienable &               loc & locative &            sim & simultaneous\\
AP & Alor-Pantar &                N & noun &                  spec & specific\\
appl & applicative  &             NP & noun phrase &          top & topic\\
caus & causative &                num & numeral &             V & verb\\
clf & classifier &                ord & ordinal &  \\        
compl & completive &              pat & patient&     \\
\end{tabularx}                    
                                  
                                  
                                  


