%10 
%examples done
%bib done
%tables done
%crossrefs 

  
\section{Introduction}
\label{sec:10:1}
\hypertarget{Toc385255248}{}The Alor-Pantar languages sometimes index event participants using pronominal indexing\is{pronominal indexing} on the verb. The factors that determine when this happens vary significantly across the languages. For some, referential properties\is{referential properties} dependent on the semantics of the event (e.g., volitionality\is{volitionality} and affectedness\is{affectedness}) play a major role, while for others it appears that the indexing is lexically determined by the verb.

In order to illustrate the range of possibilities, we concentrate on three languages: Teiwa\il{Teiwa}, Kamang\il{Kamang} and Abui\il{Abui}. For a map, see the introduction to this volume. These three languages have been chosen as they constitute three representative types in the micro-typology of the Alor-Pantar languages. 

Our main source of data are the respective corpora of Teiwa\il{Teiwa} (Klamer, Teiwa corpus), Kamang\il{Kamang} (Schapper, Kamang corpus) and Abui\il{Abui} (Kratochv\'il, Abui corpus). In additon we use the following sources: Teiwa (Pantar; \citet{Klamer2010grammar}; Klamer, fieldnotes; Robinson, fieldnotes), Kamang (Eastern Alor, \citet{SchapperEtAl2011kamus}; Schapper, fieldnotes) and Abui (Central-Western Alor; \citet{Kratochvil2007,Kratochvil2011transitivity}; Kratochv\'il, fieldnotes; Schapper, fieldnotes).

An important source of data are 42 specially designed videos created to represent events varying with respect to specific semantic variables (participant number, animacy\is{animacy}, volitionality\is{volitionality}) \citep[see][]{FeddenEtAl2013}. For Abui and Teiwa we present the data for the associated video elicitation experiments. For Kamang\il{Kamang} our video data show that prefixation occurs very often, although less so than for Abui. In order to understand Kamang's place in the typology, with its set of obligatorily prefixed verbs that must be conventionally associated with a particular prefix type, rather than this being determined by semantic values, we present data from the sources listed above.

We focus especially on the difference between properties expressing a relationship between participants and events, namely affectedness\is{affectedness} in Abui\il{Abui} and Kamang\il{Kamang} and volitionality\is{volitionality} in Abui, on the one hand, and the lexical properties of words (animacy\is{animacy}, verb classes\is{verb classes}) in Teiwa, on the other hand. We find that Abui, Kamang and Teiwa are located at different points on a continuum of lexical stipulation:\is{lexical stipulation} Abui is at one end, where event semantics play the greatest role, and Teiwa is at the other end, where lexical properties play the greatest role, with Kamang located somewhere between these two extremes.

The languages under investigation can be contrasted along further dimensions: \textit{alignment type} and \textit{number of prefix series}. Abui\il{Abui} and Kamang\il{Kamang} have semantic alignment\is{semantic alignment}, Abui being more fluid in its alignment than Kamang, as we will see in the course of this chapter. Teiwa, on the other hand, has accusative syntactic alignment\is{syntactic alignment}. Finally, the Alor languages Abui and Kamang have multiple prefix series, five and six, respectively, while the Pantar language Teiwa has only one.

It is important to bear in mind that we are not dealing with morphological case in the Alor-Pantar languages but with indexing on the verb. In this chapter we use the term `pronominal indexing'\is{pronominal indexing} to describe a structure where there is a pronominal affix on the verb\footnote{In the Alor-Pantar languages pronominal indices are exclusively prefixal.} and a co-referent noun phrase or free pronoun\is{free pronoun} optionally (indicated by brackets) in the same clause, as in \REF{bkm:Ref306281876}. Co-reference is indicated by the index \textit{k}. There is no pronominal indexing in \REF{bkm:Ref306281880}. As the Alor-Pantar languages have APV and SV word order any overt A or S precedes the verb.\footnote{We use the following primitives for core participants: S for the single argument of an intransitive verb, A for the more agent-like argument of a transitive verb, and P for the more patient-like argument of a transitive verb.} 

\ea%bkm:Ref306281876
\label{bkm:Ref306281876} 
\upshape 
    (noun phrase\textit{\textsubscript{k}}/free pronoun\textit{\textsubscript{k}})    prefix\textit{\textsubscript{k}}{}-verb\footnote{The co-occurrence of a pronominal prefix and a co-referent free pronoun is generally restricted in the Alor-Pantar languages but the constraints for this differ between the languages. In some languages the co-occurrence of the free pronoun and pronominal prefix is possible under certain circumstances, but we do not address the issue here.}
\z


\ea%bkm:Ref306281880
\label{bkm:Ref306281880}
\upshape 
(noun phrase/free pronoun)    verb 
\z

    

We concentrate in this chapter on animacy\is{animacy} and volitionality\is{volitionality} since these were varied systematically in the video stimuli. We also discuss affectedness\is{affectedness} and its impact on indexation in Kamang\il{Kamang} and Abui\il{Abui}. As affectedness is a complex issue \citep{Tsunoda1985,Beavers2011} we decided to exclude it as a factor from the video elicitation task when we were designing the video stimuli. 

Similar factors to those found in constructions involving pronominal prefixes\is{pronominal indexing} in the Alor-Pantar languages have been reported for differential object marking\is{differential object marking}, including animacy\is{animacy} \citep{Bossong1991,Croft1988,Aissen2003}, specificity\is{specificity} \citep{VonHeusingerEtAl2005}, and affectedness\is{affectedness} \citep{HopperEtAl1980,Tsunoda1981,Tsunoda1985,VonHeusingerEtAl2011}. Volitionality\is{volitionality} is, among other things, argued to play a role in differential subject marking\is{differential subject marking} in Hindi\il{Hindi} \citep{Mohanan1990}.

In \sectref{sec:10:2} and \sectref{sec:10:3} we briefly sketch the systems of syntactic and semantic alignment in Abui\il{Abui}, Kamang\il{Kamang} and Teiwa\il{Teiwa}, and discuss the number of prefix series that one finds in these languages, respectively. In \sectref{sec:10:4} we discuss our video elicitation method. We discuss the effects of animacy and volitionality in \sectref{sec:10:5}. Teiwa does not use indexing to directly represent information about events and participants but relies strongly on verb classes, conventionally associated with animacy, but also with a high degree of arbitrary stipulation. Although verb classes also play a role in Abui and Kamang, indexing in these languages is used to directly encode information about events and participants, such as volitionality and affectedness in Abui, and affectedness in Kamang. Finally in \sectref{sec:10:6}, we summarize and give a conclusion of our findings.

\section{Alignment}\is{alignment}
\label{sec:10:2}
The person prefixes found on the verbs in the Alor-Pantar languages are all very similar in form, pointing to a common historical origin.\footnote{Similar prefixes occur on nouns to mark possession. There are parallels, particularly because inalienable possession usually involves possessors linearly preceding the possessed in the same way that arguments linearly precede the verb.} However, pronominal indexing\is{pronominal indexing} is conditioned by a variety of constraints which differ between the languages. Teiwa\il{Teiwa}, a language of Pantar, has syntactic alignment\is{syntactic alignment} (accusative), whereas both Kamang\il{Kamang}, from eastern Alor, and Abui\il{Abui}, from central western Alor, have semantic alignment\is{semantic alignment}. Although there is no case marking on noun phrases, alignment can be defined relative to pronominal indexing. 

For almost all Teiwa verbs the following holds: only P's are indexed whereas S's and A's are never indexed. There is a small subset of three reflexive-like verbs which index the S (see below). Generally, therefore, Teiwa treats S like A and unlike P and can be said to have syntactic alignment of the accusative type. In Abui\il{Abui} and Kamang\il{Kamang} P's are also indexed, as are more patient-like S's (S\textsubscript{P}), while more agent-like S's (S\textsubscript{A}) are not indexed. As in Teiwa, A's are not indexed. Such systems in which S's behave differently depending on semantic factors are generally called semantic alignment systems \citep{DonohueEtAl2008} or active/agentive systems \citep{Mithun1991}.

The Alor-Pantar languages are of interest at the macro-typological level for a number of reasons. First, the nominative-accusative alignment system in Teiwa's prefixal marking is typologically the most common \citep[53]{Siewierska2004}, yet in Teiwa it is associated with the rare property of marking only the person\is{person} of the P on the verb \citep{Siewierska2013}. Second, the Alor-Pantar languages which have semantic alignment are subject to differing semantic factors in determining their pronominal indexing, including animacy\is{animacy}, volitionality\is{volitionality} and affectedness\is{affectedness}. These are, of course, implicated in many phenomena of a wider macrotypological interest.

For almost all Teiwa verbs, S's are encoded with a free pronoun\is{free pronoun}, as illustrated in \REF{ex:10:1234} and \REF{bkm:Ref383697262}.


\ea 
\label{ex:10:1234}
\langinfo{Teiwa}{}{\citealt[169]{Klamer2010grammar}} \\
 \gll    A   her. \\
   3\textsc{sg}  climb \\
  \glt `He climbs up.'
\z



\ea%bkm:Ref383697262
\label{bkm:Ref383697262}
\langinfo{Teiwa}{}{Response to video clip P04\_wake.up\_07, SP3} \\
\gll       A  uri.\\  
     3\textsc{sg} wake.up \\
\glt  `He wakes up.'
\z







An example of an indexed S is provided in \REF{bkm:Ref357868399}. Teiwa has only three verbs which follow this pattern. These are \textit{{}-o'on} `hide', \textit{{}-ewar} `return' and \textit{{}-ufan} `forget'. 


\ea%bkm:Ref357868399
\label{bkm:Ref357868399}
\langinfo{Teiwa}{}{\citealt[98]{Klamer2010grammar}} \\
\gll      Ha   h-o'on. \\  
   2\textsc{sg} 2\textsc{sg}{}-hide   \\
\glt  `You hide.'
\z







Indexation of P on the Teiwa\il{Teiwa} verb is associated with animacy\is{animacy} of P. In the Teiwa corpus (Klamer, Teiwa corpus) indexing is restricted to 49 out of 224 transitive verbs (types), i.e., {\Tilde}22\%, comprising 44 verbs which always index P and five verbs in which the presence of the index depends on the animacy value of P. The rest of the transitive verbs never index their object\is{object}. This is illustrated in \REF{bkm:Ref306280773} below for the prefixing transitive verb \textit{{}-}\textit{unba'} `meet', where the object is animate and in the third person singular, while the subject is in the second person singular. In \REF{bkm:Ref306280777}, we see the non-prefixing transitive verb \textit{ari'} `break', which typically takes an inanimate object.


\ea%bkm:Ref306280773
\label{bkm:Ref306280773}
\langinfo{Teiwa}{}{\citealt[159]{Klamer2010grammar}} \\
\gll      Name,  ha'an  n-oqai  g-unba'?\\  
  sir  2\textsc{sg} 1\textsc{sg}.\textsc{poss}{}-child  3-meet   \\
\glt `Sir, did you see (lit. meet) my child?'
\z

 

 

 


\ea%bkm:Ref306280777
\label{bkm:Ref306280777}
\langinfo{Teiwa}{}{\citealt[101]{Klamer2010grammar}} \\
\gll    Ha'an  meja  ga-fat  ari'.\\  
    2\textsc{sg}  table  3.\textsc{poss}{}-leg  break  \\
\glt  `You broke that table leg!' 
\z







Kamang\il{Kamang}, on the other hand, has semantic alignment\is{semantic alignment}, where the S is coded like the A,  in \REF{ex:10:1235}, or like the P, if the S is affected in \REF{bkm:Ref353455458}. 


\ea 
\label{ex:10:1235}
\langinfo{Kamang}{}{Response to video clip C03\_dance\_05, SP13} \\
 \gll      Almakang=a   pilan.    \\
    people=\textsc{spec} dance.lego-lego    \\
 \glt  `The people are dancing a \textit{lego-lego} (traditional dance).'
\z

 

 




\ea%bkm:Ref353455458
\label{bkm:Ref353455458}
\langinfo{Kamang}{}{\citealt[324]{Schapperta}} \\
\gll     Na-maitan-si. \\  
    \textsc{1sg.pat-}hunger-\textsc{ipfv}  \\
\glt  `I'm hungry.'
\z

 





Some verbs allow alternation between having a prefix and an affected S and having no prefix and a non-affected S. This is illustrated in \REF{bkm:Ref353455505}, where the dog runs off because it was chased away, whereas \REF{bkm:Ref324857914} does not have this affected\is{affectedness} meaning.


\ea%bkm:Ref353455505
\label{bkm:Ref353455505}
\langinfo{Kamang}{}{\citealt[236]{Schapperta}} \\
\gll   Kui  ge-tak.     \\  
     dog  3.\textsc{gen}{}-run     \\
\glt  `The dog ran off (was forced to run).'
\z




 



\ea%bkm:Ref324857914
\label{bkm:Ref324857914}
\langinfo{Kamang}{}{\citealt[326]{Schapperta}} \\
\gll    Kui  tak.  \\  
     dog  run   \\
\glt  `The dog runs.'
\z







Kamang\il{Kamang} indexes P's, for instance on the verbs \textit{{}-tan} `wake someone up' and \textit{{}-tak} `see' in examples \REF{ex:10:1236} and \REF{bkm:Ref383697282}, respectively.


\ea 
\label{ex:10:1236}
\langinfo{Kamang}{}{Response to video clip P07\_wake.up.person\_19, SP15} \\
 \gll    {\ob}\dots{\cb}  ge-pa-l  sue  ga-tan.  \\
    [{\dots}]  3.\textsc{gen}{}-father-\textsc{contr\_foc} arrive  3.\textsc{pat}{}-wake\_up   \\
 \glt  `[{\dots}] his father comes and wakes him.'
\z








\ea%bkm:Ref383697282
\label{bkm:Ref383697282}
\langinfo{Kamang}{}{Response to video clip C19\_be.afraid.of.axe\_40, SP15} \\
\gll    Ga  sue  paling  ga-tak. \\  
      3\textsc{sg} arrive  axe  3.\textsc{pat}{}-see  \\
\glt  `He comes and sees the axe.'
\z

 





Abui\il{Abui} also has semantic alignment\is{semantic alignment}. An important semantic factor in the indexing of S's is volitionality\is{volitionality}. Volitional S's are expressed with a free pronoun\is{free pronoun} and no prefix, as in \REF{bkm:Ref306280914}. Non-volitional S's are indexed with a prefix, as in \REF{bkm:Ref306280918}, where a free pronoun can optionally be used. The free translations try to capture the difference in volitionality involved here.


\ea%bkm:Ref306280914
\label{bkm:Ref306280914}
\langinfo{Abui}{}{\citealt[15]{Kratochvil2007}} \\
\gll     Na  laak.\\  
     1\textsc{sg} leave \\
\glt  `I go away.'
\z








\ea%bkm:Ref306280918
\label{bkm:Ref306280918}
\langinfo{Abui}{}{\citealt[15]{Kratochvil2007}} \\ 
\gll     (Na)  no-laak.\\  
     (1\textsc{sg})  1\textsc{sg.rec}{}-leave \\
\glt  `I (am forced to) retreat.'
\z







These Abui\il{Abui} examples do not involve transitive verbs, but there is a natural connection with the situation in Teiwa\il{Teiwa}. Prefixation in Teiwa is typical of animate objects\is{animacy}\is{object}, and objects are, among other things, expected to be non-volitional\is{volitionality} (\citealt[90]{Givon1985}, \citealt[79]{Malchukov2005}, \citealt[4]{VonHeusingerEtAl2011}). It is semantic factors, such as volitionality, which leads \citet[177-178, 257]{Kratochvil2007} to treat the Abui system as based on actor\is{actor} and undergoer\is{undergoer} roles \citep{FoleyEtAl1984}, rather than notions of subject\is{subject} and object\is{object}, which can more easily be applied to Teiwa.

Abui\il{Abui} indexes P's. There are no verbs in the corpus which are never prefixed. An example of a prefixed transitive verb indexing a P is \REF{bkm:Ref306280933}. Animacy\is{animacy} is much less important in Abui; both \textit{fik} `pull' and \textit{{}-bel} `pull' in \REF{bkm:Ref306280933} would be prefixed, even if their P's were inanimate. Another example, with the verb -\textit{kol} `tie', is given in \REF{bkm:Ref383697305}.


\ea%bkm:Ref306280933
\label{bkm:Ref306280933}
\langinfo{Abui}{}{Response to video clip C01\_pull\_person\_25, SP8} \\ 
\gll    Wiil  neng  nuku  di  de-feela  ha-fik    ha-bel-e.      \\  
     child  male  one  3\textsc{act} 3.\textsc{al.poss}{}-friend  3.\textsc{pat}{}-pull  3\textsc{.pat}{}-pull-\textsc{ipfv}  \\
\glt  `A boy is pulling his friend.'  
\z









   


\ea%bkm:Ref383697305
\label{bkm:Ref383697305}
\langinfo{Abui}{}{\citealt[91]{Kratochvil2007}} \\ 
\gll   Maama  di  bataa  ha-kol. \\  
     father  3\textsc{act} wood  3.\textsc{pat}{}-tie \\
\glt  `Father ties up the wood.'
\z







For some Abui\il{Abui} verbs a difference of affectedness\is{affectedness} in the P can be encoded by the choice of prefix, namely a prefix from the \textsc{loc} series for a lower degree of affectedness and a prefix from the \textsc{pat} series for a higher degree of affectedness. We take this up in \sectref{sec:10:5.1.2} below.

To sum up the role of conditions, Abui and Kamang index P's and some S's. This is in part determined by affectedness (in Abui and Kamang) and volitionality (in Abui). In both languages lexical verb classes also play a role to some degree, in Kamang more than in Abui. Teiwa\il{Teiwa} indexes P's in part determined by animacy. The role of animacy in Teiwa in the formation of verb classes will be taken up in \sectref{sec:10:5.3}.

\section{Number of person prefix series}
\label{sec:10:3}
All Alor-Pantar languages have at least one series of person prefixes. In the languages which have only one series, like Teiwa and Western Pantar\il{Western Pantar} \citep{Holton2010person}, this is always the series which is identified by \textit{a}{}-vowels in the singular and \textit{i}{}-vowels in the plural.\footnote{Teiwa
 actually has four verbs for which a difference between an animate and an inanimate object can be encoded by the choice of two different prefixes in the third person \ist{person} only. We discuss this alternation in detail in \sectref{sec:10:5.3}.}
The Teiwa prefixes are given in Table \ref{tab:10:11}.

\begin{table}
\caption{Teiwa person prefixes \citep[77, 78]{Klamer2010grammar}}
\label{tab:10:11}
\begin{tabularx}{.5\textwidth}{>{\sc}XX}
\lsptoprule
& Prefix\\
\midrule
1sg & {\itshape n(a)-}\\
2sg & {\itshape h(a)-}\\
3sg & {\itshape g(a)-}\\
1pl.excl & {\itshape n(i)-}\\
1pl.incl & {\itshape p(i)-}\\
2pl & {\itshape y(i)-}\\
3pl & {\itshape g(i)-, ga-}\\
\lspbottomrule
\end{tabularx}
\end{table}

Abui\il{Abui} and Kamang\il{Kamang} are innovative in that they developed multiple prefix series. The Abui prefixes are given in Table \ref{tab:10:12}.


\begin{table}[h]
\caption{Abui person prefixes}
\label{tab:10:12}
\begin{tabularx}{\textwidth}{>{\sc}XXXXXX}
\lsptoprule
 & \multicolumn{4}{c}{Prefixes}\\
 & {\scshape pat} & {\scshape rec} & {\scshape loc} & {\scshape goal} & {\scshape ben}\\
\midrule
sg & {\itshape n(a)-} & {\itshape no-} & {\itshape ne-} & {\itshape noo-}& {\itshape nee-}\\
2sg & \textit{a-}\footnotemark{} & {\itshape o-} & {\itshape e-} & {\itshape oo-}& {\itshape ee-}\\
3 (\textalpha-type) & {\itshape d(a)-} & {\itshape do-} & {\itshape de-} & {\itshape doo-}& {\itshape dee-}\\
3 (\textbeta-type) & {\itshape h(a)-} & {\itshape ho-} & {\itshape he-} & {\itshape hoo-}& {\itshape hee-}\\
1pl.excl & {\itshape ni-} & {\itshape nu-} & {\itshape ni-} & {\itshape nuu-}& {\itshape nii-}\\
1pl.incl & {\itshape pi-} & {\itshape po-/pu-} & {\itshape pi-} & {\itshape puu-/poo-} & {\itshape pii-}\\
2pl & {\itshape ri-} & {\itshape ro-/ru-} & {\itshape ri-} & {\itshape ruu-/roo-}& {\itshape rii-}\\
\lspbottomrule
\end{tabularx}
\end{table}

\footnotetext{\textit{{\O}-} before vowel.}

For each series of prefixes Abui\il{Abui} has two contrasting types for the third person\is{person}. The α-type starts with /d/ and indexes an actor\is{actor}.\footnote{\citet[78-79]{Kratochvil2007} calls these ``3\textsc{i''} (our α-type) and ``3\textsc{ii''} (our β-type).} The β-type starts with /h/ and indexes an undergoer\is{undergoer}. The difference between the α-type and the β-type is illustrated by the following two examples \REF{bkm:Ref384648605} and \REF{bkm:Ref384648613}, respectively. 

\xbox{\textwidth}{
\ea%bkm:Ref384648605
\label{bkm:Ref384648605}
\langinfo{Abui}{}{\citealt[185]{Kratochvil2007}} \\ 
\gll   Fani  el  da-wel-i. \\  
    \textsc{pn} before  3.\textsc{pat}{}-pour-\textsc{pfv}   \\
\glt  `Fani washed himself.' [α-type prefix: \textit{da-}]
\z
}


\ea%bkm:Ref384648613
\label{bkm:Ref384648613}
\langinfo{Abui}{}{\citealt[185]{Kratochvil2007}} \\ 
\gll    Fani  el  ha-wel-i.\\  
    \textsc{pn} before  3.\textsc{pat}{}-pour-\textsc{pfv}   \\
\glt `Fani washed him.' [β-type prefix: \textit{ha-}]
\z





 

Kamang\il{Kamang} has six prefix series. The Kamang prefixes are given in Table \ref{tab:10:13}.

\begin{table}
\caption[Kamang person prefixes]{Kamang person prefixes.
{In the third person prefixes, /g/ can be realized as [j] before front vowels, i.e., in the \textsc{gen} and \textsc{dat} series.}
{\dag}{The assistive (\textsc{ast}) indexes the participant who assists in the action.} 
} 
\label{tab:10:13}
\begin{tabularx}{\textwidth}{>{\sc}XXXXXXX}
\lsptoprule
 & \multicolumn{5}{c}{Prefixes}\\
 & {\scshape pat} & {\scshape loc} & {\scshape gen} & \textsc{ast}{\dag} & {\scshape dat} & {\scshape dir}\\
\midrule
1sg & {\itshape na-} & {\itshape no-} & {\itshape ne-} & {\itshape noo-} & {\itshape nee-}&	{\itshape nao-}	\\
2sg & {\itshape a-} & {\itshape o-} & {\itshape e-} & {\itshape oo-} & {\itshape ee-}&		{\itshape ao- }	\\
3 & {\itshape ga-} & {\itshape wo-} & {\itshape ge-} & {\itshape woo-} & {\itshape gee-}&		{\itshape gao- }	\\
1pl.excl & {\itshape ni-} & {\itshape nio-} & {\itshape ni-} & {\itshape nioo-} & {\itshape nii-}&		{\itshape nio- }	\\
1pl.incl & {\itshape si-} & {\itshape sio-} & {\itshape si-} & {\itshape sioo-} & {\itshape sii-}&		{\itshape sio-}	\\
2pl & {\itshape i-} & {\itshape io-} & {\itshape i-} & {\itshape ioo-} & {\itshape ii-}&		{\itshape io-}	\\
\lspbottomrule
\end{tabularx}
\end{table}
Having multiple person prefix series is not restricted to Alor-Pantar languages with semantic alignment. For example, Adang \citep{Haan2001,RobinsonEtAltaadang} has syntactic alignment like Teiwa\il{Teiwa} (i.e., only P's are indexed with a prefix) but, having three series, more readily fits with the semantically aligned languages Abui\il{Abui} and Kamang\il{Kamang} along this dimension of the micro-typology.

\section{Video elicitation}
\label{sec:10:4}
As our goal is to compare across related languages we are faced with the problem of how to obtain comparable data. Translation-based elicitation brings with it the danger that the responses are heavily biased towards the constructions of the meta-language, and prompted elicitation using the target language brings with it, among other things, well known difficulties of determining exactly what the consultant is making a judgment about and the extent to which they are trying to accommodate the researcher. We therefore decided to choose video elicitation, as this obviates many of the problems associated with other techniques. While this method entails substantial preparatory work, we can have more confidence in the results.

\subsection{Video stimuli}\is{video stimuli}
\label{sec:10:4.1}
We used 42 short video elicitation stimuli specifically designed to investigate the impact that various semantic factors have on the patterns of pronominal marking in the Alor-Pantar languages. The full set of video clips can be downloaded from \url{www.smg.surrey.ac.uk/projects/alor-pantar/pronominal-marking-video-stimuli}. A list of the clips and instructions on how to use them are provided in the appendix.

  Given that we are dealing with some systems where there is semantic alignment and others where there is a syntactic alignment system conditioned partly by semantic factors, it made sense to test the role of conditions which have been identified either for semantic alignment or for their salience in marking grammatical relations such as objects. Animacy\is{animacy} is important in Teiwa (\citealt[171]{Klamer2010grammar}; \citealt{KlamerEtAl2006}) and volitionality\is{volitionality}, telicity\is{telicity}, and the stative/active distinction were identified as major factors in the typological work on semantic alignment\is{semantic alignment} systems (\citealt{Arkadiev2008}; \citealt{Klamer2008} on semantic alignment in eastern Indonesia). We therefore chose the following five factors, each with two possible values:
  
\begin{enumerate}
\item Number of participants: 1 vs. 2
\item Animacy: Animate vs. Inanimate
\item Volitionality: Volitional vs. Non-volitional
\item Telicity: Telic vs. Atelic\footnote{We define \textit{telic} loosely as ``denoting a change of state'' and \textit{atelic} as an ``unbounded process or activity''.}
\item Dynamicity: Stative vs. Dynamic\footnote{We use the definition given by \citet[49]{Comrie1976}: ``With a state, unless something happens to change that state, then the state will continue [...]. With a dynamic situation, on the other hand, the situation will only continue if it is continually subject to a new input of energy [...]''.}
\end{enumerate}

From this, we constructed a possibility space in which we systematically varied the values. The value for \textit{animacy} only varies for S and P. The factor \textit{volitionality} varies only with respect to S and A. 

There are therefore 32 (2\textsuperscript{5}) possibilities or cells in the possibility space. Two of these value combinations are logically incompatible, namely the combination of [--Animate] and [+Volitional] and the combination of [+Telic] and [--Dynamic]. As there generally are no volitional inanimates or telic states, we eliminated these value combinations. This eliminated 7 cases from the one-participant predicates. (There are 4 telic states and 3 additional volitional inanimates. The fourth case with the combination ``volitional inanimate'' is also a telic state.) For two-participant verbs, only 4 cases had to be eliminated, namely the four telic states. As volitionality and animacy are coded for different participants, a combination of these does not cause a problem. 

Telicity and dynamicity have not been identified for the Alor-Pantar languages but we designed the experiment to include these factors because they have been repeatedly recognized as factors which impact on the realization of participant-related information in semantically aligned languages (see \citet{Arkadiev2008} and references therein). On the potential effect of telicity, see \citet{FeddenEtAl2013}. Dynamicity did not have an effect on the indexation patterns in our video elicitation task. We will say no more about these two factors here.

  The factors definiteness\is{definiteness} and specificity\is{specificity} which are also well-known to have an effect on participant marking \citep{Aissen2003} were not tested because video elicitation is not the right technique to investigate those. The values of discourse-related factors like definiteness and specificity cannot be systematically varied in any straightforward way in video elicitation.

We tested 21 factor combinations (32-7-4=21). For practical fieldwork purposes, we created a core set of video stimuli for each of the combinations and a peripheral set. Fieldworkers would use the core set as the first task and then the peripheral set where possible. For the languages discussed here both sets were completed. Because there are two sets for each of the 21 combinations, there are 42 clips. For each set the order of the clips was randomized. The order in which the clips were to be shown was fixed after randomization. 

\subsection{Speakers and procedure}
\label{sec:10:4.2}
The video stimuli were administered to a total of 11 male native speakers (four for Abui, four for Kamang\il{Kamang} and three for Teiwa).\footnote{For the purpose of citing examples anonymously we assigned each speaker a code (SP1 to SP15). This range also includes one speaker of Western Pantar and three speakers of Adang. We say nothing about these two languages here.} The video clips were shown to individual participants or groups of participants, one of whom was the primary speaker whose responses were recorded. Elicitation was conducted in Indonesian. Descriptions of the scenes in the clips were elicited using neutral cues, such as \textit{Apa yang lihat?} `What did you see?' or \textit{Apa yang terjadi?} `What happened?'. If the initial description did not include a verb which roughly corresponded to the English verb in the clip label, the field experimenters probed for the intended verb in a minimal way. All sessions were audio-recorded and the responses transcribed.

  Responses that we counted as valid had to conform to the specific factor combination for which they were given as a description. For example, the description of the clip ``hear person'' had to involve an animate entity as the object, e.g., ``hear the man''. So responses involving a non-person referent, such as ``he hears the man's voice'' were not counted for the relevant feature combination. Tables giving the proportion of prefixed verbs measured against the total of valid responses for a certain factor or combination of factors are used to show the effect of animacy or volitionality on prefixation. Figures are given for individual speakers as well as aggregated data for all speakers of each language. All percentages are conventionally rounded to yield whole numbers.

\section{Participant properties}
\label{sec:10:5}\largerpage
In this section we focus on the difference between properties expressing a relationship between participants and events (affectedness\is{affectedness}, volitionality\is{volitionality}) on the one hand and lexical properties (animacy\is{animacy}, verb classes\is{verb classes}) on the other.

While volitionality as a term suggests that it is exclusively a property of a human (or at least an animate) participant it is typically not a property of the lexical semantics of nouns that they are volitional or non-volitional agents. Nouns such as \textit{person, child,} or \textit{man} can be used in contexts in which they may be subject to non-volitional acts (e.g., fall) or volitional ones (e.g., walk), while they remain constant in their values for animacy. This means that a distinction on the basis of volitionality would not yield an exhaustive partition of the lexicon in the way that the animate-inanimate opposition would. Typically, volitionality is a property of a participant which is observed in the context of an event. In this sense we can attribute it to the event as a whole. Volitionality as we use it here (or the absence thereof) is more likely a part of the lexical semantics of verbs, as can be seen in examples like `stumble', `trip', `fall', and `vomit'. But, as with the noun examples mentioned, it is possible to find verbs where there is no requirement that their lexical semantics are committed to a value for volitionality. This entails that, while volitionality may be relevant for some verbs such as the ones we mention, it does not partition the verb lexicon in the way that animacy partitions the noun lexicon.   Animacy, on the other hand, is a lexical property. As \citet[43]{Hurford2007} notes in his discussion of the pre-linguistic basis for semantics, animacy is a more permanent property and is `less perception dependent'.

\subsection{Abui}\il{Abui}
\label{sec:10:5.1}
Of the three languages in our sample Abui\il{Abui} shows the greatest flexibility of combining verbs with prefixes from different series. However, the \textsc{pat} prefix series is much more lexically limited than the other inflections  in Abui. Verbs that take this prefix are the only verbs showing lexical classing, i.e., the absence of alternation. 

\subsubsection{   Inflection classes in Abui}\is{inflection classes}\is{lexical stipulation}
The discussion here is based on a detailed examination of the prefixal behaviour of 210 verbs. The numbers reflect the state of the documentation and analysis of the language at present (see Kratochv\'il, Abui corpus).

For 33 Abui\il{Abui} verbs inflection with a \textsc{pat} prefix is either obligatory or optional. Table \ref{tab:10:14}  presents the distribution of the \textsc{pat} prefix across the whole sample (all percentages rounded to whole numbers). Obligatory inflection with a \textsc{pat} prefix means that a verb has to have a prefix and that the prefix has to be from the \textsc{pat} prefix series. In other words, these verbs exclusively appear with the \textsc{pat} prefix. This is the case for 14\% of the verbs in the sample (29 out of 210 verbs). Within optional \textsc{pat} verbs we distinguish two cases. First, there are the verbs that always require a prefix, and this may be from any series, including the \textsc{pat} series. These are a minority (4 out of 210 verbs, or 2\% in Table \ref{tab:10:14}). Second, there are the verbs that may occur without a prefix or with a prefix, from any series, including the \textsc{pat} series. These form a substantial subset (68 out of 210 verb or 32\% in Table \ref{tab:10:14}).  

\begin{table}
\caption{Distribution of the Abui \textsc{pat} prefixes \ist{prefix alternation}}
\label{tab:10:14}

\begin{tabularx}{\textwidth}{lQQQ}
\lsptoprule
& \textsc{pat} obligatory & \multicolumn{2}{c}{\textsc{pat} optional}\\\cmidrule(lr){3-4}
 & Prefix required & A prefix is required & A prefix is not required\\
\midrule
 & 29 verbs & 4 verbs & 68 verbs\\
Total (of 210 verbs) & 14\% (29/210) & 2\% (4/210) & 32\% (68/210)\\
\lspbottomrule
\end{tabularx}
\end{table}

There are 29 verbs in our sample which obligatorily occur with the \textsc{pat} prefix. Examples are -\textit{ieng}~`see s.o./sth.', -\textit{kai} `drop s.o./sth.', \textit{{}-lal} `laugh', -\textit{rik} `hurt s.o.' and -\textit{tamadia} `repair sth.'.\footnote{The addition of \textit{someone} (s.o.) and/or \textit{something} (sth.) in the glosses indicates whether a verb can appear with an animate or an inanimate P in the corpus. If there is no such addition, e.g., \textit{maha} `want' the prefix indexes the S.} An example with \textit{-ful} `swallow s.o./sth.' is provided in \REF{bkm:Ref323745334} As the verbs which obligatorily take a \textsc{pat} prefix do not form a semantic class we treat them as an inflection class, defined by the fact that these verbs can only occur with a \textsc{pat} prefix.

\ea%bkm:Ref323745334
\label{bkm:Ref323745334}
\langinfo{Abui}{}{\citealt[463]{Kratochvil2007}} \\ 
\gll     Kaai   afu  ha-ful.\\  
    dog  fish  3.\textsc{pat}{}-swallow  \\
\glt  `The dog swallowed the fish.'
\z







For 72 verbs the \textsc{pat} prefix is optional. Four of these always require a prefix, i.e., the verb cannot occur without a prefix. These are \textit{{}-dak} `grab firmly s.o./sth.', \textit{\discretionary{-}{-}{-}luol} `follow, collect s.o./sth.', \textit{{}-k} `throw at s.o./sth.; feed s.o./sth.' and \textit{{}-maha} `want'. An example is given in \REF{bkm:Ref383697342}.


\ea%bkm:Ref383697342
\label{bkm:Ref383697342}
\langinfo{Abui}{}{\citealt[364]{Kratochvil2007}} \\ 
\gll    Kokda   di  ha-luol  we  hu  ama  fen-i. \\  
   younger  3\textsc{act} 3.\textsc{pat}{}-follow  leave  \textsc{spec} person  injure.\textsc{compl}{}-\textsc{pfv}  \\
\glt  `When the younger one followed him, people killed (him).'
\z



 



68 verbs optionally take \textsc{pat} but occurrence without a prefix is also possible. Examples of these are \textit{(-)}\textit{aahi} `take away sth.', \textit{(-)}\textit{dik} `stab s.o./sth.', \textit{(-)}\textit{wik} `carry s.o./sth.' and \textit{(-)}\textit{yok} `cover s.o./sth.'. An example is given in \REF{bkm:Ref383697351}, where the verb \textit{(-)wik} `carry s.o./sth.' occurs without a prefix.


\ea%bkm:Ref383697351
\label{bkm:Ref383697351}
\langinfo{Abui}{}{\citealt[502]{Kratochvil2007}} \\ 
\gll     Na   ne-sura  wik-e.\\  
     1\textsc{sg} \textsc{1sg.al.poss}{}-book  carry-\textsc{ipfv} \\
\glt  `I carry my book.'
\z







The verbs which optionally occur with a \textsc{pat} prefix can take a prefix from at least one other series instead of \textsc{pat}. The majority of these verbs can alternate\is{prefix alternation} between the \textsc{rec}, \textsc{loc}, \textsc{goal}, and \textsc{ben} prefixes, whereby semantic differences in the indexed participant are observable when alternating one prefix with another.

To sum up, apart from lexical classing found in verbs which only occur with the \textsc{pat} prefix, the Abui\il{Abui} prefix system is highly fluid and verbs can occur with most, perhaps all of the prefixes, or be unprefixed. In the following sections we deal in turn with affectedness and volitionality as factors which impact on the prefixation patterns.

\subsubsection{Affectedness in Abui}\label{sec:10:5.1.2}\is{affectedness}
Affectedness is one of the factors that has an impact on pronominal indexing\is{pronominal indexing} in Abui\il{Abui}. Affected participants undergo a persistent change. On affectedness as a criterion for high transitivity, see \citet{HopperEtAl1980} and \citet{Tsunoda1981,Tsunoda1985}. On affectedness as a parameter of semantic distinctness between the two participants of a transitive clause, see \citet{Naess2004,Naess2006,Naess2007}.

Affectedness is clearly a relation between a participant and an event because, while the participant is the affected entity, the predicate contains the information whether the change of state is entailed \citep[337]{Beavers2011}. 

Abui\il{Abui} allows the expression of different degrees of affectedness by choosing between the \textsc{pat} and the \textsc{loc} prefix series for P, as illustrated in Table \ref{bkm:Ref383856262}.


\begin{table}
\caption{Degrees of affectedness in Abui (Abui \citealt[596; p.c.]{Kratochvil2011transitivity})} \ist{affectedness}
\label{bkm:Ref383856262}
\begin{tabularx}{\textwidth}{XX} 
\lsptoprule
 Lower degree of affectedness: \textsc{loc} prefix & Higher degree of affectedness: \textsc{pat} prefix\\
 \midrule 
 \trs{he-dik}{stab s.o./sth.} & \trs{ha-dik}{pierce s.o./sth. through} \\
 \trs{he-akung}{cover sth.} & \trs{h-akung}{extinguish sth.} \\
 \trs{he-pung}{hold sth.} & \trs{ha-pung}{catch sth.} \\
 \trs{he-komangdi}{make sth. less sharp} & \trs{ha-komangdi}{make sth. completely blunt} \\
 \trs{he-lilri}{warm sth. up (water)} & \trs{ha-lilri}{boil sth. (water)} \\
 \trs{he-lak}{take sth. apart} & \trs{ha-lak}{demolish sth.} \\
\lspbottomrule
\end{tabularx}
\end{table}




The \textsc{loc} series is chosen if the change of state in P is either not entailed, e.g., \textit{he-pung} `hold sth.' vs. \textit{ha-pung} `catch sth.', or if it is that P is less strongly affected \textit{he-dik} `stab s.o./sth.' vs. \textit{ha-dik} `pierce s.o./sth. through' or \textit{he-lak} `take sth. apart' vs. \textit{ha-lak} `demolish sth.'. The \textsc{pat} series on the other hand is chosen if P is highly affected and a change of state in P is entailed. Full examples for \textit{{}-dik} are provided in \REF{bkm:Ref383697370} and \REF{bkm:Ref383697375}.


\ea%bkm:Ref383697370
\label{bkm:Ref383697370}
\langinfo{Abui}{}{\citealt[194]{Kratochvil2007}} \\ 
\gll  Rui  ba  tukola  mi-a  ma-i  yo  wan  e  he-dik-i?      \\  
   rat  \textsc{lnk} hole  be.in-\textsc{dur} be.\textsc{prox-pfv} \textsc{dem}  already    before  3.\textsc{loc}{}-prick-\textsc{pfv} \\
\glt `Has that rat that was in a hole already been stabbed?'
\z
  


\ea%bkm:Ref383697375
\label{bkm:Ref383697375}
\langinfo{Abui}{}{\citealt[194]{Kratochvil2007}} \\ 
\gll   Rui  tukola  mi-a  hare  bataa    mi  ha-dik-e! \\  
    rat  hole  be.in-\textsc{dur} so  wood  take  3.\textsc{pat}{}-stab-\textsc{ipfv} \\
\glt  `There are rats in the hole, so take a stick and run them through!'
\z



 



These Abui\il{Abui} examples show the impact of different degrees of affectedness depending on which prefix series is chosen for the indexing of P.

\subsubsection{Volitionality in Abui}\is{volitionality}
Next we deal with the factor of volitionality. Volitionality in a linguistic context has been defined in various ways in the literature which make sense intuitively, but to our knowledge there has been no serious attempt to formalize volitionality in a way that \citet{Beavers2011} did for affectedness. \citet[286]{HopperEtAl1980} define volitionality as the ``degree of planned involvement of an A[gent] in the activity of the verb''. \citet[52]{DeLancey1985} equates volitionality with conscious control over the activity of the verb. Furthermore, it has been observed in the literature that control and volition often coincide (\citealt[392]{Tsunoda1985}; \citealt[56]{DeLancey1985}) and that instigation is sometimes used interchangeably with control \citep[45]{Naess2007}. On volition as an entailment which identifies (Proto-)Agents, see \citet{Dowty1991}.

As we noted, volitionality may be a property associated with nouns denoting human (or some animate) participants, but many nouns of this type are non-committal as to the volition of what they denote. This contrasts with animacy\is{animacy}, where a noun either denotes an animate or an inanimate entity. For verbs there is also no requirement that their lexical semantics be committed to a value for volitionality, but this information can be encoded by the choice of indexing they take. Animacy, on the other hand, is a lexical property. Hence, where volitionality is involved, this is a semantic factor associated with the event as a whole.

In Abui\il{Abui}, a language with semantic alignment\is{semantic alignment}, volitionality is an important factor. It determines whether an S is indexed. The absence of a prefix signals volitional S's, whereas free pronouns\is{free pronoun} are outside the system of volitionality and non-volitionality. This is illustrated with the following pair: \textit{na laak} [1\textsc{sg} leave] `I go away' vs. \textit{(na)} \textit{no-laak} [(1\textsc{sg}) 1\textsc{sg.rec}{}-leave] `I (am forced to) retreat'. These examples illustrate this with the first person, which has the potential to differ in terms of volitionality. We can therefore identify a relative scale with respect to the factors, where affectedness\is{affectedness} is about the event and volitionality can be about the event, but where the lexical semantics of certain items restricts the possibilities for its application.

In the video elicitation task Abui\il{Abui} had the most instances of prefixation of the S in one-place predicates (Table \ref{tab:10:15}).

\begin{table}[htb]
\caption{Indexation of S's in one-place predicates in Abui \ilt{Abui} (responses to the video stimuli)}
\label{tab:10:15}
\begin{tabularx}{\textwidth}{lSSSSS} 
\lsptoprule
&  SP8 &  SP9 &  SP10 &  SP11 &  All\\
\midrule 
One-place predicates &  17 &  12 &  10 &  12 &  51\\
Prefixed &  8 &  6 &  4 &  5 &  23\\
Proportion &  47\% &  50\% &  40\% &  42\% &  45\%\\
\lspbottomrule
\end{tabularx}
\end{table}


A proportion of 45\% is very high in comparison to Teiwa\il{Teiwa}, where S's were not indexed at all in the video elicitation tasks, and to Kamang\il{Kamang}, where an average of 19\% of S's were indexed.

As we shall see, non-volitionality, when combined with animacy, appears to play a bigger role in prefixation in Abui\il{Abui} intransitives than in any of the other languages. This is consistent with Kratochv\'il's analysis of Abui\il{Abui} as a semantically aligned\is{semantic alignment} language. Free pronouns\is{free pronoun} on their own, that is without a co-referent prefix, are reserved for typical agents, i.e., participants who have volition with respect to the event and are not affected by it. The set of free pronouns\is{free pronoun} includes the third person pronoun \textit{di},\footnote{The free pronoun \textit{di} is probably of verbal origin and has grammaticalized from the auxiliary \textit{d} `hold' \citep{Kratochvil2011transitivity}. Participants marked with \textit{di} are mainly humans, but nonhuman participants of considerable agentive force, e.g., a storm, are also possible.} which can appear on its own or be adnominal following a noun phrase. In our experiment, there were no instances where an S was encoded with \textit{di} in any of the responses. In all cases noun phrases without \textit{di} were used, for example in \REF{bkm:Ref283206980}.


\ea%bkm:Ref283206980
\label{bkm:Ref283206980}
\langinfo{Abui}{}{Response to video clip P20\_run\_06, SP8} \\ 
\gll    Ama  nuku  furai  ba  weei.\\  
      man  one  run  and  go\\
\glt  `A man is running along.' 
\z







Other examples from the experiment are: \textit{mit} `sit', \textit{natet} `stand' and \textit{it} `lie'. Further examples from the Abui\il{Abui} corpus are: \textit{ayong} `swim', \textit{kalol} `foretell (fortune or the future)', \textit{kawai} `argue', \textit{luuk} `dance', \textit{miei} `come', \textit{taa} `lie', \textit{yaa(r)} `go'. Semantically, these are mainly motion verbs, posture verbs, and social activities. Typically these express their S with a free pronoun and not a prefix because they typically denote events with volitional participants\is{volitionality}. 

Free pronouns\is{free pronoun} can be combined with a co-referent prefix (in the third person this needs to be an α-type prefix) to express reflexive situations, in which the agent is volitional but also affected by his (own) action. As there are no examples of this construction in the responses to the video elicitation task, a textual example is given in \REF{bkm:Ref283207000}.


\ea%bkm:Ref283207000
\label{bkm:Ref283207000}
\langinfo{Abui}{}{\citealt[203]{Kratochvil2007}} \\ 
\gll    Ata  di  do-kafi-a.\\  
      \textsc{pn} 3\textsc{act}  3.\textsc{rec}{}-scrape-\textsc{dur}\\
\glt  `A. scratches himself (intentionally).' [α-type prefix: \textit{do-}]
\z







In the video elicitation tasks, non-volitional S's are expressed only with a prefix. An example of this is given in \REF{bkm:Ref383854029}.


\ea%bkm:Ref383854029
\label{bkm:Ref383854029}
\langinfo{Abui}{}{Response to video clip P09\_person.fall\_14, SP 9} \\ 
\gll    Neng  nuku  laak-laak-i  ba  me  la  da-kaai  yo  eya!\\  
    man  one  walk-walk-\textsc{pfv} and  come  just  3.\textsc{pat}{}-stumble  \textsc{dem} \textsc{exclam}\\
\glt  `A man walks along and stumbles there, whoops!' [α-type prefix: \textit{do-}]
\z



 



In the responses to the video elicitation task, α-type prefixes were exclusively used in the descriptions of one-participant events. In each case the prefix cross-references the sole participant in the event denoted by the verb. Prefixes of the α-type are used with non-volitional\is{volitionality} S's, namely the S of \textit{minang} `wake up', \textit{liel} `tall', \textit{lal} `laugh', \textit{kaai} `stumble', and \textit{yongf} `forget' (which was employed in descriptions of the sleep event [i.e., video clip C05\_sleep\_11]). Speakers also very consistently used α-type prefixes with volitional S's with the two positional verbs \textit{ruid} `rise, stand up' and \textit{reek} `lie'.


\ea 
\langinfo{Abui}{}{Response to video clip P21\_stand.up\_02, SP11} \\ 
 \gll    Wil  neng  da-ruid-i  ba  laak-i.\\
    child  male  3.\textsc{pat}{}-stand.up-\textsc{pfv} and  leave-\textsc{pfv} \\
 \glt `The guy stands up and leaves.' [α-type prefix: \textit{da-}]
\z





 

However, just looking at the effect of volitionality alone on the coding in the experiment does not give us a clear picture. The proportions for non-volitional and volitional S's are about equal (see Table \ref{tab:10:16}).

\begin{table}[htb]

\caption{Indexation of non-volitional and volitional S's in Abui \ilt{Abui} (responses to the video stimuli)}
\label{tab:10:16}
\begin{tabularx}{\textwidth}{lSSSSS} 
\lsptoprule
&  SP8&  SP9&  SP10&  SP11&  All\\
\midrule 
Non-volitional S &  11&  6&  4&  6&  27\\
Prefixed &  5&  3&  2&  2&  12\\
Proportion &  45\%&  50\%&  50\%&  33\%&  44\%\\
 &  &  &  &  & \\
Volitional S &  6&  6&  6&  6&  24\\
Prefixed &  3&  3&  2&  3&  11\\
Proportion &  50\%&  50\%&  33\%&  50\%&  46\%\\
\lspbottomrule
\end{tabularx}
\end{table}

The impact of non-volitionality becomes more obvious when one looks at non-volitional animate\is{animacy} S's. Of all S's in one-place predicates, non-volitional animate S's are most likely to be indexed (Table \ref{tab:10:17}).

\begin{table}[htb]

\caption{Indexation of non-volitional animate S's in Abui \ilt{Abui} (responses to the video stimuli)}
\label{tab:10:17} 
\begin{tabularx}{\textwidth}{lSSSSS} 
\lsptoprule
&  SP8&  SP9&  SP10&  SP11&  All\\
\midrule 
Non-volitional 

AND animate S &  6&  4&  3&  3&  16\\
Prefixed &  4&  3&  2&  2&  11\\
Proportion &  66\%&  75\%&  66\%&  66\%&  69\%\\
\lspbottomrule
\end{tabularx}
\end{table}

In Abui\il{Abui} animate S's that are non-volitional are indexed with a prefix for an average of 69\% of the cases, whereas animate S's (55\%), volitional animate S's (46\%), and inanimate (and thus by definition non-volitional) S's (9\%) show much lower proportions. This pattern may have a functional explanation, in that use of prefixation encodes information that the default expectation is not met that an animate participant is volitional.

In sum, Abui\il{Abui} has a high degree of semantic fluidity, and prefixation patterns depend on the factors affectedness and volitionality. We now turn to the neighbouring language Kamang, in which arbitrary inflection classes (at least synchronically) play a larger role than in Abui.
 

\subsection{Kamang}\il{Kamang}

\label{sec:10:5.2}
Kamang\il{Kamang}, like Abui\il{Abui}, has semantic alignment\is{semantic alignment} and several prefix series. However, in Kamang\il{Kamang} the actual use of prefixes differs radically from Abui. Kamang\il{Kamang} is more restricted in terms of the possible combinations of verbs with prefixes than Abui. More than in Abui, lexical classes\is{verb classes} in Kamang\il{Kamang} play an important role in determining prefixation patterns of the S in intransitive clauses and the P in transitive clauses. We have based our analysis of Kamang\il{Kamang} on a corpus of 510 verbs (Schapper, Kamang\il{Kamang} corpus; \citet{SchapperEtAl2011kamus}). In Kamang\il{Kamang} the primary verb class divide is between:

(i)  \textit{Obligatorily prefixed verbs}: These require a prefix on the verb in order to be well-formed. The prefix comes from one of the six series, is lexically fixed for each verb and does not alternate. For verbs in this group the different prefixal inflections have no obvious semantic functions, but rather define arbitrary inflection classes. Of the 510 verbs in the corpus, 166 are obligatorily prefixed (approx.~33\%).

(ii) \textit{Non-obligatorily prefixed verbs}: These do not require a prefix. Where prefixes are added to these verbs they have semantically transparent functions. Prefixation can either be argument-preserving, whereby prefixation of the verb does not add another argument or alter the valency of the verb, or argument-adding, whereby the prefix indexes an additional argument. 344 verbs belong into this class (approx. 67\%).

We see in Table \ref{tab:10:18} that there is a substantial difference in the prefixal requirements of transitive and intransitive verbs (all percentages rounded to whole numbers). In the classification of verbs as either intransitive or transitive we follow \citet{SchapperEtAl2011kamus}.

\begin{table}[hbt]

 \caption{Kamang verbs (obligatorily prefixed and non-obligatorily prefixed)}
\label{tab:10:18} 
\begin{tabularx}{\textwidth}{lSS}
\lsptoprule
& Obligatorily prefixed & Non-obligatorily prefixed\\
\midrule 
Transitive & 45\% (113/250 verbs) & 55\% (137/250 verbs)\\
Intransitive & 20\% (53/260 verbs) & 80\% (207/260 verbs)\\
Total (of 510 verbs) & 33\% (166/510 verbs) & 67\% (344/510 verbs)\\
\lspbottomrule
\end{tabularx}
\end{table}



Almost half of the transitive verbs that we sampled from the corpus are obligatorily prefixed, whereas substantially fewer of the intransitive verbs (only 20\%) are. 

\subsubsection{Inflection classes in Kamang}\is{inflection classes}\il{Kamang}
As noted, one third of the verbs in Kamang\il{Kamang} are obligatorily prefixed and fall into arbitrary inflection classes. All of these verbs require a prefix and the prefix series is lexically fixed and independent of verb semantics. 

Table \ref{tab:10:19} presents the percentages of obligatorily prefixed intransitive verbs across inflection classes. The prefix indexes S. Well over half occur in the \textsc{pat} inflection, whereas less than one fifth goes in each of the \textsc{loc} and \textsc{gen} inflection classes. The remainder is made up of the \textsc{ast} class. There are no instances of obligatorily prefixed intransitive verbs outside these four inflection classes.

\begin{table}[htb]
 
\caption{Proportion of obligatorily prefixed intransitive verbs by prefix class}
\label{tab:10:19}
\ist{prefix alternation}
\begin{tabularx}{\textwidth}{XXXX}
\lsptoprule
{\scshape pat} & \textsc{loc} & \textsc{gen} & {\scshape ast}\\
\midrule
65\% (33 verbs) & 15\% (8 verbs) & 18\% (11 verbs) & {\textless}2\% (1 verb)\\
\lspbottomrule
\end{tabularx}
\end{table}

Table \ref{tab:10:110} presents the percentages of obligatorily prefixed transitive verbs across inflection classes (rounded to whole numbers). The prefix indexes P. Over half of these verbs belong to the \textsc{loc} inflection, while roughly 35\% are in the \textsc{pat} inflection. The remainder is made up by a handful of transitive verbs from the other four inflections.

\begin{table}[htb]

\caption{Proportion of obligatorily prefixed transitive verbs by prefix class}
\label{tab:10:110}
\begin{tabularx}{\textwidth}{XXX}
\lsptoprule

\textsc{pat} & \textsc{loc} & Other\\
\midrule
35\% (46 verbs) & 60\% (82 verbs) & {\textless}5\% (9 verbs)\\
\lspbottomrule
\end{tabularx}
\end{table}


The distribution of verbs over these classes is independent of verb semantics. Within the obligatorily prefixed intransitive verbs {}-\textit{waawang} `remember', -\textit{mitan} `understand' and \textit{{}-pan} `forget' have similar semantics, yet they belong to the inflection classes \textsc{pat}, \textsc{gen}, and \textsc{ast}, respectively. Similarly, {}-\textit{iwei} `vomit', \textit{{}-tasusin} `be sweaty' and -\textit{wilii} `defecate' belong to the classes \textsc{pat,} \textsc{loc}, and \textsc{gen}. Within the obligatorily prefixed transitive verbs \textit{-set} `shake up and down' belongs to \textsc{pat}, while \textit{-gaook} `shake back and forth' belongs to \textsc{loc}. Similarly, -\textit{kut} `stab s.o./sth.' belongs to \textsc{pat} and \textit{{}-fanee} `strike, shoot s.o./sth.' to \textsc{gen}. The inflection classes \textsc{dat} and \textsc{dir} contain one verb each and are therefore too small for any common semantics to be discernible.

In the following examples we illustrate the inflection classes in Kamang\il{Kamang}. For each class we give an intransitive and a transitive example and provide a list of verbs so the reader can further appreciate that classing is independent of verb semantics.

Examples \REF{bkm:Ref324337569} and \REF{bkm:Ref324338165} show an intransitive verb encoding S with a \textsc{pat} prefix and a transitive verb encoding P with a \textsc{pat} prefix, respectively.


\ea%bkm:Ref324337569
\label{bkm:Ref324337569}
\langinfo{Kamang}{}{Response to video clip P21\_stand.up\_02, SP12} \\ 
\gll    Lami  saak  nok  ga-serang  maa  we.\\  
    husband  old  one  3.\textsc{pat}{}-get.up  walk  go  \\
\glt  `A guy gets up and goes.'
\z


\ea%bkm:Ref324338165
\label{bkm:Ref324338165}
\langinfo{Kamang}{}{\citealt[73]{SchapperEtAl2011kamus}} \\ 
\gll   Gal  na-kut.     \\  
   3  1\textsc{sg.pat}-stab     \\
\glt `He stabbed me.'
\z

Examples of intransitive verbs in the \textsc{pat} inflection class are: \textit{-iloi} `feel nauseous', \textit{-ook} `shiver, tremble', and \textit{-tan} `collapse, fall over'. Examples of transitive verbs in the \textsc{pat} inflection class are: \textit{-asui} `disturb s.o./sth.', \textit{-beh} `order s.o./sth.', and \textit{-kut} `stab s.o./sth.'.

Examples \REF{bkm:Ref324338359} and \REF{ex:10:1238} below show an intransitive verb encoding S with a \textsc{loc} prefix and a transitive verb encoding P with a \textsc{loc} prefix, respectively.


\ea%bkm:Ref324338359
\label{bkm:Ref324338359}
\langinfo{Kamang}{}{\citealt[286]{SchapperEtAl2011kamus}} \\ 
\gll    No-tasusing.   \\  
    \textsc{1sg.loc}{}-sweat   \\
\glt  `I'm sweaty.'
\z


\ea 
\label{ex:10:1238}
\langinfo{Kamang}{}{\citealt[50]{SchapperEtAl2011kamus}} \\ 
 \gll    Ga  bong=a  wo-gaook.  \\
 \textsc{3agt} tree=\textsc{spec} 3.\textsc{loc}{}-shake.back.and.forth     \\
 \glt  `He shook the tree.'
\z



Examples of intransitive verbs in the \textsc{loc} inflection class are: \textit{-biee} `angry' and \textit{-tasusin} `sweaty'. Examples of transitive verbs in the \textsc{loc} inflection class are: \textit{-aakai} `trap, trick s.o./sth.', \textit{-eh} `measure sth.' and \textit{-ra} `carry (s.o./sth.)'.

Examples \REF{bkm:Ref353451839} and \REF{bkm:Ref324339697} below show an intransitive verb encoding S with a \textsc{gen} prefix and a transitive verb encoding P with a \textsc{gen} prefix, respectively.


\ea%bkm:Ref353451839
\label{bkm:Ref353451839}
\langinfo{Kamang}{}{Schapper, fieldnotes} \\ 
\gll     Ne-soona-ma.  \\  
    1\textsc{sg.gen}{}-slip-\textsc{pfv}   \\
\glt  `I slipped over.'
\z








\ea%bkm:Ref324339697
\label{bkm:Ref324339697}
\langinfo{Kamang}{}{Schapper, fieldnotes} \\ 
\gll     Leon  ne-fanee-si.\\  
  Leon  1\textsc{sg}.\textsc{gen}{}-shoot-\textsc{ipfv}   \\
\glt  `Leon shoots at me.'
\z



 



Examples of intransitive verbs in the \textsc{gen} inflection class are: \textit{-foi} `dream', \textit{-iyaa} `go home', \textit{-laita} `shy', \textit{-taiyai} `cooperate, work together', and \textit{-wilii} `defecate'.

There are only two transitive verbs in the \textsc{gen} inflection class, namely \textit{-fanee} `strike, shoot s.o./sth.' and \textit{-towan} `carry sth. on a pole between two people'.

Examples \REF{bkm:Ref324340307} and \REF{bkm:Ref324340314} below show an intransitive verb encoding S with an \textsc{ast} prefix and a transitive verb encoding P with an \textsc{ast} prefix, respectively.


%\label{bkm:Ref372879178}
\ea%bkm:Ref324340307
\label{bkm:Ref324340307}
\langinfo{Kamang}{}{\citealt[103]{SchapperEtAl2011kamus}} \\ 
\gll     Oo-pan-si  naa.\\  
    2\textsc{sg.ast}{}-forget-\textsc{ipfv} \textsc{neg}  \\
\glt  `Don't you forget.'
\z








\ea%bkm:Ref324340314
\label{bkm:Ref324340314}
\langinfo{Kamang}{}{\citealt[131]{SchapperEtAl2011kamus}} \\ 
\gll   Dum  kiding=a  ga-filing  woo-tee. \\  
  child  small=\textsc{spec} \textsc{3.poss}{}-head  3.\textsc{ast}{}-protect   \\
\glt  `The child protected his head.'
\z



 



The number of verbs in the \textsc{ast} inflection class is very small. Transitive verbs are \textit{-sui} `dry sth. off', \textit{-tee} `protect sth.', and \textit{-waai} `be facing s.o./sth.'. There is only one intransitive verb \textit{-pan} `forget'. Like other cognition verbs and sensory perception verbs in Kamang\il{Kamang} (e.g., \textit{-mitan} `understand', \textit{-mai} `hear') this verb is intransitive. This is seen by the fact that it is unable to occur with an NP encoding the stimulus in its basic form, such as that in example \REF{bkm:Ref324340307} above, as shown in \REF{bkm:Ref372879210}. The stimulus -- or better said that which is to be remembered -- must be retrieved simply from the discourse context. To explicitly include an extra participant with such a verb is possible in two ways: (i) by using an applicative morpheme, such as \textit{wo-} in \REF{bkm:Ref372879221}, or (ii) by having a complement clause following the clause with the cognition verb, as in \REF{bkm:Ref372879227}. 


\ea%bkm:Ref372879210
\label{bkm:Ref372879210}
\langinfo{Kamang}{}{\citealt[324]{Schapperta}} \\ 
\gll *Mooi  oo-pan-si  naa.\\  
      banana  \textsc{2sg.ast}{}-forget-\textsc{ipfv} \textsc{neg}  \\
\glt  `Don't you forget the bananas.'
\z



 



\ea%bkm:Ref372879221
\label{bkm:Ref372879221}
\langinfo{Kamang}{}{Schapper, fieldnotes} \\ 
\gll    Mooi  wo-oo-pan-si  naa.\\  
  banana  \textsc{appl-2sg.ast}{}-forget-\textsc{ipfv} \textsc{neg}  \\
\glt `Don't you forget the bananas.'
\z



  

  

\ea%bkm:Ref372879227
\label{bkm:Ref372879227}
\langinfo{Kamang}{}{Schapper, fieldnotes} \\ 
\gll  Oo-pan-si  naa  mooi  met. \\  
    \textsc{2sg.ast}{}-forget-\textsc{ipfv} \textsc{neg}  banana  take \\
\glt  `Don't you forget to bring the bananas.'
\z

 

 



Class size decreases even further in the inflection classes\is{inflection classes} \textsc{dat} with \textit{-sah} `block s.o./sth.' and \textsc{dir} with \textit{-surut} `chase s.o.'. They each include a single transitive verb only. There are no intransitive verbs in either \textsc{dat} or \textsc{dir}.

  To sum up, obligatorily prefixed verbs in Kamang\il{Kamang} fall into inflection classes. Synchronically, there is no semantically transparent reason why one prefixal inflection is used with one verb and another inflection with another one. The relation between prefix and verb is simply lexically fixed. None of these verbs can ever occur without a prefix. We now turn to prefixation in non-obligatorily prefixed verbs and the semantic factor of affectedness which influences the prefixation patterns. 

\subsubsection{Affectedness in Kamang}\is{affectedness}\il{Kamang}
Affectedness can be identified as a semantic factor which plays a role in indexing in non-obligatorily prefixed verbs in Kamang\il{Kamang}. It is a property expressing a relationship between participants and events. Stative verbs like \textit{saara} `burn' or \textit{suusa} `be in difficulty' take a \textsc{loc} prefix to express that the S is affected. In \REF{ex:10:1239}, the S is affected in its entirety. Kamang\il{Kamang} expresses this by indexing the S with a \textsc{loc} prefix on the verb. On the other hand, in \REF{ex:10:1240}, where the S is less affected, the prefix is absent.


\ea 
\label{ex:10:1239}
\langinfo{Kamang}{}{\citealt[325]{Schapperta}} \\ 
 \gll   Kik  nok  wo-saara.   \\
    palm.rib  one  3\textsc{.loc-}burn   \\
 \glt  `A palm rib burns down/on (i.e., is consumed over time).'
\z



\ea 
\label{ex:10:1240}
\langinfo{Kamang}{}{\citealt[325]{Schapperta}} \\ 
 \gll   Kik  nok  saara.   \\
  palm.rib  one  burn     \\
 \glt  `A palm rib burns.'
\z



  The possibility of indexing affected participants with a prefix is not restricted to inanimates\is{animacy}. Compare \REF{bkm:Ref384657017}, with an inanimate, and \REF{bkm:Ref384657025}, with an animate participant.


\ea%bkm:Ref384657017
\label{bkm:Ref384657017}
\langinfo{Kamang}{}{Schapper, fieldnotes} \\ 
\gll   Buk  taa  kamal.   \\  
    mountain  top  cold    \\
\glt `The mountains are cold.' (i.e., `In the mountains, it is cold.')
\z



\ea%bkm:Ref384657025
\label{bkm:Ref384657025}
\langinfo{Kamang}{}{Schapper, fieldnotes} \\ 
\gll    No-kamal-da-ma.     \\  
    1\textsc{sg}.\textsc{loc}{}-cold-\textsc{aux-pfv}   \\
\glt  `I have cooled.' (i.e., `My fever has come down.')
\z



In \REF{bkm:Ref384657017} \textit{kamal} `cold' describes a constant property, whereas in \REF{bkm:Ref384657025} it denotes a change of state in an (animate) participant affected by the process of the dropping of their body temperature.

In sum, affectedness plays an important role in the indexing patterns in Kamang\il{Kamang}. In contrast to Abui\il{Abui} the degree of lexical stipulation\is{lexical stipulation} is much higher. While Abui coerces only one sixth of its verbs into one fixed inflection, namely the \textsc{pat} inflection, Kamang\il{Kamang} (unevenly) assigns one third of its verbal vocabulary to six inflection classes. Because of practical constraints we have sampled a larger number of Kamang\il{Kamang} verbs than is the case for Abui or Teiwa. It is a reasonable expectation that a larger sample size would give us the opportunity to see the verbs more evenly distributed across the classes, and yet Kamang\il{Kamang} does not show this. This suggests that this contrast between Abui and Kamang\il{Kamang} is a real and important factor.

In the remainder of this chapter we look at the importance of animacy as a factor in Teiwa.

\subsection{Animacy and verb classes in Teiwa}\is{animacy}\is{verb classes}\il{Teiwa}
\label{sec:10:5.3}
Teiwa has syntactic alignment\is{syntactic alignment}, whereby only P's are indexed on the verb. This is a rare type cross-linguistically, occurring in only 7\% of the languages from \citeauthor{Siewierska2013}'s (2013)  WALS sample. Animacy is the core semantic factor which plays a role in whether an object is indexed on the verb. It has often been observed in the literature that objects\is{object} are typically not animate, definite, or specific and that it is marked, if they are animate, definite, or specific in a given context (see for example \citet{Givon1976,Aissen2003}; also see \citet[205-205]{Bickel2008}. There is a cross-linguistically robust association between marked objects and topicality. This association may have been obscured by grammaticalization, but what we still find in some languages is that marked objects are associated with semantic features typical of topics, such as animacy \citep[2]{DalrympleEtAl2011}. 

In the video elicitation task, all three Teiwa participants used prefixes exclusively with animate objects of transitive verbs. The number of prefixes used is too small to say anything reliable about the possible impact of (non-)volitionality on prefixation. Participants consistently used prefixes for the same three verbs, all of which are transitive and have animate objects. These are \textit{{}-tan (tup)} [lit. call get.up] `wake someone up', \textit{{}-u'an} `hold someone in one's arms', and \textit{{}-arar} `be afraid of someone'. An example is given in \REF{bkm:Ref283206596}.


\ea%bkm:Ref283206596
\label{bkm:Ref283206596}
\langinfo{Teiwa}{}{Response to video clip P07\_wake.up.person\_19, SP4} \\ 
\gll   Kri  nuk  ma    {bif goqai}  ga-tan-an  tup. \\  
   old.man  one  come  child  3-call-\textsc{real} get.up  \\
\glt `An old man comes and wakes up a small child.' 
\z




Having an animate object is not a sufficient condition for the object to be indexed by a prefix. In our experiment, many animate objects were not indexed with a prefix. In fact, indexation of an animate object in Teiwa\il{Teiwa} accounts for 50\% of the instances, as in Table \ref{tab:10:111}. 

\begin{table}
\caption{Prefixation with animate P's in Teiwa}
\label{tab:10:111}
\begin{tabularx}{\textwidth}{XXXXX} 
\lsptoprule
&  SP2&  SP3&  SP4&  All\\
\midrule 
Animate P's &  5&  6&  7&  18  \\
Prefix &  3&  3&  3&  9\\
Proportion &  60\%&  50\%&  43\%&  50\%\\
\lspbottomrule
\end{tabularx}
\end{table}


The results suggest that the animacy\is{animacy} of the object\is{object} cannot be the whole story. It is therefore worth considering whether (a) the rule of object indexation is at all productive in Teiwa and if so, whether (b) the effects we have observed in relation to a property of the object might more readily be associated with the verb itself. 

  To address the first question we did a corpus search for Teiwa\il{Teiwa} inspired by the quantitative method in \citet{Baayen1992} and subsequent work based on that. The Teiwa\il{Teiwa} corpus we used for this consists of about 16,900 words of which roughly one third is elicited material. The assumption is that, if a morphological process is productive\is{productivity} in a language, hapax legomena in the corpus will exhibit it. The basic intuition behind this is that lower frequency items will need to rely on the creativity associated with rules, whereas memory will have a greater role in relation to high frequency items. Therefore, if in Teiwa\il{Teiwa} most instances of transitive verbs with animate objects which occur only once have a prefix, then the rule can be considered productive. If, on the other hand, there is no difference in the behaviour of the hapax legomena, i.e., if there is a more or less even split, then it is impossible to conclude anything.

The results for transitive verb hapaxes are summarized in Table \ref{tab:10:112}. The number before the slash includes hapaxes in elicited material, the number after the slash excluded elicited items.

\begin{table}[h]
\caption{Hapax legomena of transitive verbs in Teiwa\il{Teiwa}}
\label{tab:10:112} 
\begin{tabularx}{\textwidth}{lrrr}
\lsptoprule
& Total \mbox{number} of \mbox{hapaxes} & With prefix & Proportion\\
\midrule 
With \mbox{animate} object & 9 / 7 & 8 / 6 & 88.8\%/85.7\%\\
With \mbox{inanimate} object & 13 / 12 & 1 / 1 & 7.7\%/8.3\%\\
\lspbottomrule
\end{tabularx}
\end{table}
  

Bear in mind that we did not search for all verb hapaxes, only transitive ones. The number of intransitive verb hapaxes is not relevant to the question whether morphological rules in transitive verbs are productive, as intransitive verbs are not prefixed in Teiwa\il{Teiwa} at all.

These results strongly indicate that prefixation of animate objects is indeed productive in Teiwa\il{Teiwa} and not an artefact associated with high frequency. 88.8\% of transitive verb hapaxes with an animate object actually also have a prefix. If the elicited hapaxes (2 in total) are eliminated, the proportion is still 85.7\%. Conversely, if we look at transitive verbs with an inanimate object, only about 8\% of the hapaxes have prefixes. Of course, the Teiwa\il{Teiwa} corpus is nowhere near as large in its coverage as the ones Baayen used, but they give us the best evidence we can obtain at the moment.

Having established that object indexation seems to be a productive rule in Teiwa\il{Teiwa} we turn to the second question, namely whether the observed animacy effects might be associated with the verb itself. 

If prefixation in Teiwa\il{Teiwa} were purely a matter of sensitivity to the animacy\is{animacy} property of the object\is{object}, rather than a manifestation of the class\is{verb classes} to which a verb belongs, we would expect one and the same verb to alternate\is{prefix alternation} between prefixation and non-prefixation, depending on the animacy of the object it happened to be taking. This, however, is typically not the case. There are instances where the very same verb has a prefix regardless of the animacy value of the object. This is illustrated for the verb \textit{-uyan}, which is prefixing in \REF{ex:10:1241}, where it appears with an animate P, and also prefixing in \REF{ex:10:1242}, where it appears with an inanimate P:


\ea 
\label{ex:10:1241}
\langinfo{Teiwa}{}{\citealt[88]{Klamer2010grammar}} \\ 
 \gll    A  qavif    ga-uyan  gi  si  \dots\\
    3\textsc{sg} goat  3-search  go  \textsc{sim} \\
 \glt `He went searching for a goat, [{\dots}]'

\z



\ea 
\label{ex:10:1242}
\langinfo{Teiwa}{}{\citealt[340]{Klamer2010grammar}} \\ 
 \gll   {\dots}  ha        gi    ya'          siis nuk  ga-uyan   pin   aria'.\\
        {}  2\textsc{sg} go    bamboo\_sp.  dry  one  3-search  hold  arrive \\
 \glt `[{\dots}] You go look for dry bamboo to bring here.'   
\z


The converse case is more frequent. There are many transitive verbs that never index their P, regardless of its animacy value. This is illustrated in \REF{bkm:Ref306281423} and \REF{ex:10:1243} where the verb \textit{tumah} occurs with an animate and an inanimate P, respectively. The examples \REF{bkm:Ref383697393} and \REF{bkm:Ref383697402} illustrate this for the serial verb construction \textit{ta tas} [on stand] `stand on'. The verb does not have a pronominal index regardless of the animacy value of the object.


\ea%bkm:Ref306281423
\label{bkm:Ref306281423}
\langinfo{Teiwa}{}{Response to video clip C13\_bump\_into\_person\_38, SP4} \\ 
\gll  Uy    masar  nuk  wa    kri  tumah. \\  
   person  male  one  go  old\_man  bump  \\
\glt  `A man is going and bumps into an old man.'
\z
 

\ea 
\label{ex:10:1243}
\langinfo{Teiwa}{}{Response to video clip C16\_bump\_into\_tree\_42, SP4} \\ 
 \gll   Kri  nuk  tewar  wa  tei  tumah.\\
old\_man  one  walk  go  tree  bump   \\
 \glt `An old man walks and bumps into a tree.'
\z

 
 
\ea%bkm:Ref383697393
\label{bkm:Ref383697393}
\langinfo{Teiwa}{}{Response to video clip C04\_step.on.person\_32, SP4} \\ 
\gll   {Bif goqai}  ma  oma'  ta  tas\\  
    child  come  father  on  stand \\
\glt `A child comes and steps on his father.'
\z

 
 

\ea%bkm:Ref383697402
\label{bkm:Ref383697402}
\langinfo{Teiwa}{}{Response to video clip C20\_step.on.banana\_33, SP4} \\ 
\gll   kri   nuk  ma  moxoi  muban  ta  tas  {\ob}...{\cb}\\  
   old.man  one  come  banana  ripe  on  stand  [{\dots}]  \\
\glt `An old man comes and steps on a ripe banana {\dots}'
\z

 
 

      

In Teiwa\il{Teiwa} we find the formation of a class of prefixed vs. a class of not prefixed verbs\is{verb classes} based on the animacy\is{animacy} value of the objects\is{object} a verb typically occurs with. There are four classes of verbs.

The first class of transitive verbs consists of prefixed verbs. These always index their P with a prefix and they typically occur with animate objects. A separate noun phrase constituent may optionally be present. In addition to the transitive verbs used in the video elicitation task \textit{{}-arar} `be afraid of s.o.', \textit{{}-tan (tup)} [lit. call get.up] `wake s.o. up', and \textit{{}-u'an} `carry s.o.', further examples from the corpus are: -\textit{ayas} `throw at s.o.', \textit{{}-bun} `answer s.o.', \textit{{}-fin} `catch s.o.', -\textit{lal} `show to s.o.', \textit{{}-liin} `invite s.o.', \textit{{}-pak} `call s.o.', \textit{{}-panaat} `send to s.o.', \textit{{}-regan} `ask s.o.', \textit{{}-rian} `look after s.o.', \textit{{}-sas} `feed s.o.', \textit{{}-soi} `order s.o.', \textit{{}-tiar} `chase s.o.', \textit{{}-ua'} `hit s.o.', \textit{{}-'uam} `teach s.o.', and \textit{{}-wei} `bathe s.o.'. 

The second class of transitive verbs consists of unprefixed verbs. These never index their P and typically occur with inanimate objects. A separate noun phrase constituent may optionally be present. Examples from the video elicitation task are: \textit{si'} `wash sth.', \textit{miman} `smell sth.', and \textit{wuraq} `hear sth.'. Further examples from the corpus are: \textit{bali} `see s.o./sth.', \textit{bangan} `ask for sth.', \textit{boqai} `cut sth. up', \textit{dumar} `push sth. away', \textit{hela} `pull sth.', \textit{mat} `take sth.', \textit{me'} `be in sth.', \textit{moxod} `drop s.o./sth.', \textit{ol} `buy sth.', \textit{pin} `hold s.o./sth.', \textit{qas} `split sth.', \textit{taxar} `cut sth. in two', \textit{tian} `carry sth. on head or shoulder'.

An explanation of the behaviour of the verb (i.e., whether it has a prefix) based on verb semantics is likely to fail. Verbs with similar semantics can vary, such as the verb `to cradle' in \REF{ex:10:1244}, in contrast to the verb `to hold' in \REF{ex:10:1245}:


\ea 
\label{ex:10:1244}
\langinfo{Teiwa}{}{Response to video clip P15\_hold.person\_24, SP3} \\ 
 \gll    Kri  nuk  g-oqai  g-u'an-an  tas-an.\\
  old\_man  one  3-child  3-cradle-\textsc{real} stand-\textsc{real}  \\
 \glt `An old man is standing cradling his child.'
\z
 
\ea 
\label{ex:10:1245}
\langinfo{Teiwa}{}{\citealt[436]{Klamer2010grammar}} \\ 
 \gll   Qau  ba  iman  ta    mauqubar  g-oqai  pin  bir-an  gi  {\ob}{\dots}{\cb} \\
    good  \textsc{seq}  \textsc{3pl}  \textsc{top}  frog  3-child  hold  run-\textsc{real}  go \\
 \glt  `So they hold the baby frog and go, [{\dots}].' 
\z

 



       

Some verbs which typically occur with inanimates, e.g., \textit{pin} `hold', could well occur with animates, as in \REF{ex:10:1245}. It is very difficult to identify certain verb semantics which would be associated with the verb taking a prefix. Generally, when looking at verbs of similar semantics, some verbs will have a prefix while others do not.

As mentioned above, it is not the case that prefixation in Teiwa\il{Teiwa} is purely a matter of sensitivity to the animacy property of the object, but rather a manifestation of the class to which a verb belongs. We do, however, find a few cases where one and the same verb alternates\is{prefix alternation} between prefixation and non-prefixation or between two different sets of prefixes, depending on the animacy of the object the verb happened to be taking. Such verbs make up the classes 3 and 4, respectively.

Transitive verbs of class 3 either have a prefix and an animate\is{animacy} object\is{object} or no prefix and an inanimate object. This class is small and consists of five verbs, given in Table \ref{tab:10:Ref306281469}.

\begin{table}[h]
\caption{Transitive verbs with or without prefix (class 3)} 
\label{tab:10:Ref306281469}
\begin{tabularx}{\textwidth}{>{\it}XX>{\it}XX}  
\lsptoprule
{-dee}  & `burn s.o.'     & \textit{dee}&  `burn sth.'\\
{-mai}&  `keep for s.o.'  & \textit{mai}&  `save sth.' \\
{-mar}&  `follow s.o.'    & \textit{mar}&  `take/get sth.'\\
{-mian}&  `give to s.o.'  & \textit{mian}&  `put at sth.'\\
{-sii}&  `bite s.o.'     & \textit{sii}&  `bite (into) sth.'\\
\lspbottomrule
\end{tabularx}
\end{table}

For these verbs the animate-inanimate distinction constitutes an agreement feature realized by the presence or the absence of the prefix.

Transitive verbs of class 4 select one prefix set with animate objects and another prefix set with inanimate objects. This class comprises only four items, listed in Table \ref{tab:10:Ref306281514}.




\begin{table} 
\caption{Transitive verbs taking different prefixes (class 4)}
\label{tab:10:Ref306281514}
\begin{tabularx}{\textwidth}{>{\it}XXX} 
\lsptoprule
 -{kiid} & `cry for s.o.'& `cry about sth.'\\
 -{tad} & `strike s.o.'& `strike at sth.'\\
 -{wultag}&  `talk to s.o.'& `talk about sth.'\\
 -{wulul} & `tell s.o.'& `tell sth.'\\
\lspbottomrule
\end{tabularx}
\end{table}



Class 4 shows alternation\is{prefix alternation} between two different prefixes in the 3\textsuperscript{rd} person\is{person}. Inanimate objects are indexed with the normal \textit{ga-} prefix whereas animate objects take an augmented form (with a glottal stop). Compare examples \REF{ex:10:1246} and \REF{ex:10:1247}.


\ea 
\label{ex:10:1246}
\langinfo{Teiwa}{}{\citealt[92]{Klamer2010grammar}} \\ 
 \gll    Ha  gi  ga'-wulul.\\
   2\textsc{sg} go  3.\textsc{an}-talk  \\
 \glt  `You go tell him.'
\z
 
\ea 
\label{ex:10:1247}
\langinfo{Teiwa}{}{\citealt[92]{Klamer2010grammar}} \\ 
 \gll    Ha  gi  ga-wulul.    \\
   2\textsc{sg} go  3-talk      \\
 \glt  `You go tell it (i.e., some proposition)!'
\z







This contrast exists in the third person only. Although the first and second persons are always animate they nonetheless take the unaugmented prefix forms with the class 4 verbs, e.g., \textit{ha gi na-wulul/*na'-wulul} `You go tell me'.

There is a potential issue in these examples because the semantic roles of the non-subject arguments in \REF{ex:10:1246}, a human recipient, and \REF{ex:10:1247}, a proposition or message expressed as the object, are different but this need not concern us because Teiwa\il{Teiwa} (as indeed all Alor-Pantar languages) has secundative alignment \citep[449, 454]{Klamer2010ditransitive}.\footnote{On the notion of secundative alignment, \ist{secundative alignment} see \citet{Dryer1986}.} This means that the language generally treats recipients (and goals, including those of ballistic motion and comitatives) like patients, both of which are indexed with a prefix, e.g., \textit{-an} `give to s.o.', \textit{-honan} `come to s.o.', \textit{-ayas} `throw at s.o.', and \textit{-yix} `descend with s.o.'. Therefore it is fully expected that the non-subject arguments in \REF{ex:10:1246} and \REF{ex:10:1247} -- despite their difference in semantic role -- are both indexed with a prefix. For the verbs in class 4, we can see the development of a small inflectional paradigm in which the animate-inanimate distinction constitutes an agreement\is{agreement} feature realized by different prefix types. Importantly, it also contrasts with class 3, which in essence realizes the same animate-inanimate\is{animacy} distinction, but uses prefixation vs. lack of prefixation to do it rather than different prefix forms. These are therefore examples of arbitrary inflection classes, as the same animate-inanimate distinction (in classes 3 and 4) has different reflexes depending on the verb. So there is strong evidence for Teiwa\il{Teiwa} contrasting with Abui\il{Abui} and Kamang\il{Kamang}, and this appears to be associated with a move from semantic related factors to a greater role for animacy and verb classes. 

\section{Discussion and conclusion}
\label{sec:10:6}
The Alor-Pantar languages are of significant macrotypological interest for pronominal indexing, because they show contrasting behaviours in terms of the degree to which purely lexical information is involved. For Abui\il{Abui}, prefixation is determined to a greater extent by the semantics of the event, rather than the semantics associated directly with the lexical item. Volitionality\is{volitionality} and affectedness\is{affectedness} are interpreted at the level of the event itself, rather than a constant and indefeasible part of a verb's semantics. For Kamang\il{Kamang}, which has what would still be broadly defined as a semantic alignment system, affectedness also plays a role, but there appears to be greater scope for arbitrary association between a prefix-class and a particular verb, so that verbs are more restricted in terms of the choice of prefix with which they may occur. It is reasonable to infer that the restriction of a given verb to one prefix series, as happens in Kamang\il{Kamang}, results from the strengthening of associations between particular verbs and the prefix series on the basis of those verbs' frequent occurrences in constructions related to the original event-related semantics. These prefixes then become conventionally\is{conventionalization} associated with subsets of verbs, as is the case in Kamang\il{Kamang}, and are restricted to those verbs. In contrast with Kamang\il{Kamang}, for Teiwa\il{Teiwa} animacy\is{animacy} plays an important role in effecting this conventional association. While we cannot be entirely sure about the diachronic scenario, the most entrenched conventionalization is associated with the prefix series which is the oldest, namely the \textsc{pat} series.

  The video elicitation task confirms the importance of animacy in Teiwa\il{Teiwa}. There is interesting interaction of animacy and volitionality in Abui\il{Abui}, where volitionality and animacy work together to increase the likelihood of the intransitive subject\is{subject} (S) being indexed on the verb. Our experimental method confirmed the fascination of the Alor-Pantar languages for understanding the role of the usual suspects in realizing grammatical relations. While it is possible to identify roles for the different factors, their influence is manifested in different ways and to different degrees. This is further evidence that it is impossible to assume a direct relationship between the semantics and the formal realization of indexation. The experiment shows that none of these systems of indexation is semantically fully transparent. Being an animate P is not a sufficient condition to be indexed in Teiwa\il{Teiwa}. Many animate P's are, in fact, not indexed and the number of verbs which alternate between having an animate object, which is indexed with a prefix, or having an inanimate object, which is not indexed or indexed with a different prefix, is quite small.

The three Alor-Pantar languages considered in this chapter provide important typological insights into the relationship between referential properties\is{referential properties} and lexical stipulation\is{lexical stipulation} as evinced in a language's patterns of pronominal indexing\is{pronominal indexing}. In all of the languages we have discussed here, properties of the verb play some role. In the semantically aligned\is{semantic alignment} languages, this emerges from the lexical semantics of verbs with regard to affectedness\is{affectedness} or volitionality\is{volitionality}. But we can observe a change in orientation from properties expressing a relationship between participants and events, as in Abui\il{Abui} and Kamang\il{Kamang}, to properties involving lexical features of the verb itself. Semantic factors in events are reinterpreted as constraints on individual verbs. The role of animacy\is{animacy} is increasingly important in Teiwa\il{Teiwa}. The language has a very small set of verbs (classes 3 and 4) in which animacy figures as an agreement feature. Thus, in Teiwa\il{Teiwa} a conventionalization\is{conventionalization} has taken place where verb classes become associated with the animacy value of the objects with which the verbs in a given class typically occur.

Across the three languages, the nature of the semantic restrictions on pronominal indexing differs, and animacy\is{animacy} is a property which actually allows for arbitrary classes\is{verb classes} to emerge, much more so than affectedness and volitionality. This is because it classifies the argument of the verb according to animacy but also involves an expectation based on the verb's own semantics (about the properties of the objects it selects for), while at the same time not directly classifying the relationship between the participant and the event. Given this dual nature of animacy, there is therefore a strong potential for properties based on what is expected to clash with what actually occurs, and there is greater potential for arbitrary classes to emerge. A reasonable hypothesis is that the Teiwa\il{Teiwa} system represents one possible trajectory within Alor-Pantar from a system which is highly dependent on the event semantics to one where the restrictions on prefixes lead to a much smaller number of verbs being prefixed. 

\startappendix
\subsection{The Video Elicitation Task} 
\subsubsection{Background}

These short video elicitation stimuli are a means to systematically study the variation in the patterns of pronominal marking in the Papuan languages of Alor and Pantar. The design of an elicitation task consisting of video clips, which systematically vary the parameters under investigation, is inspired and influenced by the video elicitation tools developed by the Max Planck Institute for Psycholinguistics in Nijmegen. See \citet{BohnemeyerEtAl2001,BowermanEtAl2004} and \citet{EvansEtAl2004}.

All Alor-Pantar languages share the typologically rare trait that they mark objects or undergoers on the verbs, rather than subjects or actors \citep{Siewierska2013}. However, there is considerable within-group variation as to how this is done and also what the relevant semantic parameters are which govern the indexation patterns. For instance, Teiwa \citep{Klamer2010grammar} aligns its arguments on a nominative-accusative basis indexing the object of some (but not all) transitive verbs. The prime factor which determines whether a verb indexes its object is animacy \citep{KlamerEtAl2006,Klamer2010grammar}. Abui \citep{Kratochvil2007,Kratochvil2011transitivity}, on the other hand, has a semantic alignment system, in which the undergoer is marked on the verb. In intransitive clauses, more undergoer-like arguments are indexed, e.g., `He is ill', whereas more actor-like ones are not, e.g., `He runs'. 

  Although the video clips were designed with the argument-indexing typology of the Alor-Pantar languages in mind they can readily be used to elicit patterns of participant marking in languages which employ case and/or adpositions or a combination of argument indexing and case/adpositional marking. 

\subsubsection{Task}

\paragraph{Materials}

The task consists of 42 video clips to be described by the consultants. The clips have been divided into two sets, a core set and a peripheral set, each consisting of 21 clips. From the pair of clips for each combination of factors, one clip is in the core set, one is in the peripheral set. The clips have been randomly ordered within their sets and afterwards been numbered from C01 to C21 (core set) and P01 to P21 (peripheral set). 

The clips are named in the following way, e.g., C14\_sit.down\_01.mp4.

The initial letter identifies a clip as belonging either to the core (C) or the peripheral (P) set. The letter is followed by a number, which indicates the order in which the clips are to be tested. Then comes a short characterization of the event shown in the clip. The final number before the file extension refers to the number of the clip before randomization.

For example: C14\_sit.down\_01.mp4 -- This clip belongs to the core set, it is number 14 in the randomized clip order, it depicts a man sitting down, before randomization it was clip number 01, and it is a MP4-file.

{\bfseries Do test the clips on your laptop before you go to the field!}

\paragraph{Requirements}

Laptop with Windows Media Player (or indeed any play\-er which handles MPEG-4 video files) or Quicktime (for Mac). The videos have a sound track which is not essential for understanding what is going on but which provides ambient sounds, so make sure you turn up the volume on your laptop. Without sound the clips will probably feel less natural. Record responses on audio- and/or video-tape with an external microphone.

\paragraph{Number of speakers}

Run the stimuli with four different couples of speakers. If feasible, it might be a good idea to have one speaker describe the clips to the other, who is sitting behind the computer screen and is not able to see the clips. That way the speaker doing the experiment has someone to address when describing the clips. If this is not feasible or undesirable for any reason, having both speakers looking at the clips will also be fine. For each speaker, you should record full meta-data, such as age, sex, education, language used in the task, other languages known by the speaker, etc. Of course, it is fine to run the experiment with individual speakers rather than pairs of speakers.

\paragraph{Procedure}

\begin{enumerate}
 \item Make sure you audio- and/or video-tape each elicitation session. 
 \item  You and your speaker(s) sit in front of the laptop. Explain to each speaker that they will see scenes in which someone does something or something happens, and that they should afterwards describe what happened. You then prompt them after each clip, saying ``Can you describe the scene?''. You can stop prompting speakers in this way once it's no longer necessary.
 \item  You can repeat a clip as often as you need to, if the speaker wants to see it again. You can also go back to a previous clip, if necessary. If the speaker does not recognize an object in a clip you can explain what it is.
  \item It is crucial that you get a description of the event depicted in the clip that includes a verb which roughly corresponds to the English verb in the clip label. If that does not happen you might have to probe for the intended verb. 
\end{enumerate}


  For example, it is conceivable that a speaker describes a scene in which a man is ``lying'' on the ground as ``There is a man on the ground''. Similarly, if a speaker gives a description of possible intentions the agent might have, like ``He's cleaning up' (for \textit{wash plate}), or ``He wants the man to come to him'' (for \textit{pull person}), or a very general description of the scene, you should immediately probe for the intended verb. If a speaker uses a serial verb construction make sure this is the most basic way of encoding the event.

\paragraph{Further probing and elicitation}

While carrying out the procedure outlined above opportunities for further probing might come up. This does not have to be done with every single speaker. 

  In some cases, you might want to probe further whether the indexing patterns of a verb change when the animacy value of the object/undergoer changes. It might for example be possible to use some of the verbs of spatial configuration, such as `stand' and `lie', with inanimates (as in English). Or you might want to find out what happens to the indexing patterns, if a child falls instead of a coconut?

  Another point for further probing is following up on any alternative verbs which a speaker might have used in the description of a particular event. What is the exact meaning of the verb? What are its indexing patterns?

  It might be worth enquiring further into what happens to indexation when the volitionality of the Agent (e.g., Agent does something inadvertently) or the telicity of the event (e.g., `eating bananas' vs. `eat a banana up') change. It'll probably turn out quite quickly whether something is going on there.

  Some events might be described with a serial verb construction. When this happens, make sure that this is the most basic way of encoding the event. 

  Finally, for the clips where it makes sense, you could ask the speakers to imagine that they themselves \textit{did} what was shown in the clip or that it \textit{happened to them}. Ask them to imagine that they went home to their family and told them about it. This would yield a 1\textsuperscript{st} person singular participant (in the agent or patient role) and will be helpful in finding out about or excluding person effects. Again, it will not be necessary to do this with all speakers and it might well turn out that it only works with some.

\paragraph{List of video clips}

Below is a list of all video clips for the task. Each row provides information on the combination of factors which define a given cell in the possibility space. For each cell, there are two clips. The verb describing the main event in each clip is given numbered from 01-42 (which is the original numbering). There is a short description of the event depicted in each clip. Finally, the name of the clip file is given. Core set video clips appear in boldface. The full set of video clips can be downloaded from \url{http://www.smg.surrey.ac.uk/projects/alor-pantar/pronominal-marking-video-stimuli/}.\enlargethispage{4em}



\noindent
\footnotesize
\begin{tabularx}{\textwidth}{llllllQl}
\lsptoprule 
\\[.4cm] %set off rotated boxes from header
\begin{rotate}{30}participants\end{rotate} 		& \begin{rotate}{30}volitional\end{rotate} 		& \begin{rotate}{30}telic\end{rotate} 		& \begin{rotate}{30}animate\end{rotate} 		& \begin{rotate}{30}stative\end{rotate} 		&  event 	&Description 									&Clip file name \\

\midrule
\textbf{1} 	& \textbf{+} 	& \textbf{+} 	& \textbf{+} 	& \textbf{--} 	& \textbf{1} 	\textbf{\textit{sit down}}& \textbf{Person} \textbf{sitting} \textbf{down.} 	& \textbf{C14\_sit.down\_01} \\
1 		& + 		& + 		& + 		& -- 		& 2 		\textit{stand up}& Person standing up. 						& P21\_stand.up\_02  \\
1 		& + 		& -- 		& + 		& + 		& 3 		\textit{stand}&Person standing.				& P17\_stand\_03  \\
\textbf{1} 	& \textbf{+} 	& \textbf{--} & \textbf{+} 	& \textbf{+} & \textbf{4} 	\textbf{\textit{lie}}& \textbf{Person} \textbf{lying} \textbf{on} \textbf{the} \textbf{ground.} & \textbf{C10\_lie\_04}  \\
\textbf{1} 	& \textbf{+} 	& \textbf{--} & \textbf{+} 	& \textbf{--} 	& \textbf{5} 	\textbf{\textit{dance}}&\textbf{People} \textbf{dancing.} & \textbf{C03\_dance\_05} \\
1 		& + 		& -- 		& + 		& -- 		& 6 		\textit{run}& Person running across the frame.					&  P20\_run\_06  \\
1 		& -- 	& + 		& + 		& -- 		& 7 		\textit{wake up}& Person waking up suddenly.					&  P04\_wake.up\_07  \\
\textbf{1} 	& \textbf{--}& \textbf{+} & \textbf{+} 	& \textbf{--} 	& \textbf{8} 	\textbf{\textit{fall} \textbf{asleep}}& \textbf{Person} \textbf{sitting,} \textbf{falling} \textbf{asleep.}&  \textbf{C06\_fall.asleep\_08}  \\
\textbf{1} 	& \textbf{--}& \textbf{+} & \textbf{--}& \textbf{--} 	& \textbf{9} 	\textbf{\textit{fill} \textbf{up}}& \textbf{Glass} \textbf{being} \textbf{filled} \textbf{from} \textbf{bottle.} & \textbf{C09\_fill.up\_09} \\
1 		& -- 	& + 		& --		& -- 		& 10 		\textit{go out}& Flame goes out. 						& P03\_go.out\_10  \\
\textbf{1} 	& \textbf{--}& \textbf{--} & \textbf{+} & \textbf{+} & \textbf{11} 	\textbf{\textit{sleep}}& \textbf{Person} \textbf{sleeping.} 			& \textbf{C05\_sleep\_11} \\
1 		& -- 	& -- 		& + 		& + 		& 12 		\textit{be tall}& Two people, one tall and one short. 				& P05\_be.tall\_12 \\
\textbf{1} 	& \textbf{--}& \textbf{--} & \textbf{+} & \textbf{--} 	& \textbf{13} 	\textbf{\textit{laugh}}& \textbf{Person} \textbf{laughing.}			 & \textbf{C07\_laugh\_13} \\
1 		& -- 	& -- 		& + 		& -- 		& 14 		\textit{fall}& Person slipping and falling. 					& P09\_person\_fall\_14 \\
1 		& -- 	& -- 		& -- 		& + 		& 15 		\textit{be big}& One big and two small stones.					&  P18\_be.big\_15  \\
\textbf{1} 	& \textbf{--}& \textbf{--} & \textbf{--}& \textbf{+}& \textbf{16} 	\textbf{\textit{be} \textbf{long}}& \textbf{One} \textbf{long} \textbf{and} \textbf{three} \textbf{short} \textbf{logs.} & \textbf{C17\_be.long\_16} \\
\textbf{1} 	& \textbf{--}& \textbf{--} & \textbf{--}& \textbf{--} & \textbf{17} 	\textbf{\textit{fall}}& \textbf{Coconut} \textbf{falling.}			&  \textbf{C15\_fall\_17}  \\
1 		& -- 	& -- 		& -- 		& -- 		& 18 		\textit{burn}& Burning house. 					& P10\_burn\_18 \\
2 		& + 		& + 		& + 		& -- 		& 19 		\textit{wake s.o. up}& Person waking another person up.				&  P07\_wake.up.person\_19  \\
\textbf{2} 	& \textbf{+} 	& \textbf{+} 	& \textbf{+} 	& \textbf{--} 	& \textbf{20} 	\textbf{\textit{run} \textbf{to} \textbf{s.o.}}& \textbf{Child} \textbf{running} \textbf{to} \textbf{parent.} & \textbf{C12\_run.to.person\_20} \\
\textbf{2} 	& \textbf{+} 	& \textbf{+} 	& \textbf{--} & \textbf{--} 	& \textbf{21} \textbf{\textit{eat} \textbf{sth.}} &\textbf{Person} \textbf{eating} \textbf{a} \textbf{banana.} & \textbf{C11\_eat.banana\_21} \\
2 		& + 		& + 		& -- 		& -- 		& 22 		\textit{wash sth.} &Person washing plate.					&   P16\_wash.plate\_22 \\
\textbf{2} 	& \textbf{+} 	& \textbf{--} & \textbf{+} 	& \textbf{+} & \textbf{23} 	\textbf{\textit{lean} \textbf{on} \textbf{s.o.}} & \textbf{Child} \textbf{leaning} \textbf{on} \textbf{parent.} & \textbf{C02\_lean.on.person\_23}  \\
\lspbottomrule
\end{tabularx}

\noindent
\begin{tabularx}{\textwidth}{llllllQl}
\lsptoprule
\\[.4cm] %set off rotated boxes from header
\begin{rotate}{30}participants\end{rotate} 		& \begin{rotate}{30}volitional\end{rotate} 		& \begin{rotate}{30}telic\end{rotate} 		& \begin{rotate}{30}animate\end{rotate} 		& \begin{rotate}{30}stative\end{rotate} 		&  event 	&Description 									&Clip file name \\

\midrule
2 		& + 		& -- 		& + 		& + 		& 24 		\textit{hold s.o.} &Person holding child. 					& P15\_hold.person\_24 \\
\textbf{2} 	& \textbf{+} 	& \textbf{--} & \textbf{+} 	& \textbf{--} 	& \textbf{25} 	\textbf{\textit{pull} \textbf{s.o.}} &\textbf{A} \textbf{pulling} \textbf{B.} &  \textbf{C01\_pull.person\_25}  \\
2 		& + 		& -- 		& + 		& -- 		& 26 		\textit{smell s.o.} &A sniffing at B, disgusted face. 				& P01\_smell.person\_26 \\
\textbf{2} 	& \textbf{+} 	& \textbf{--} & \textbf{--} & \textbf{+} & \textbf{27} 	\textbf{\textit{lean} \textbf{on} \textbf{sth.}}& \textbf{Person} \textbf{leaning} \textbf{on} \textbf{house.} & \textbf{C21\_lean.on.house\_27} \\
2 		& + 		& -- 		& -- 		& + 		& 28 		\textit{hold sth.} &Person hugging a tree.					 & P13\_hold.tree\_28 \\
\textbf{2} 	& \textbf{+} 	& \textbf{--} & \textbf{--}& \textbf{--} 	& \textbf{29} 	\textbf{\textit{pull} \textbf{sth.}} &\textbf{Child} \textbf{pulling} \textbf{a} \textbf{log.} & \textbf{C18\_pull.log\_29} \\
2 		& + 		& -- 		& -- 		& -- 		& 30 		\textit{smell sth.} &Person sniffing at food, disgusted face. & P02\_smell.food\_30 \\
2 		& -- 	& + 		& + 		& -- 		& 31 		\textit{fall onto s.o.} &Banana drops on person's stomach.			 & P19\_fall.onto.person\_31  \\
\textbf{2} 	& \textbf{--}& \textbf{+}& \textbf{+} 	& \textbf{--} 	& \textbf{32} 	\textbf{\textit{step} \textbf{on} \textbf{s.o.}}& \textbf{Child} \textbf{stepping} \textbf{on} \textbf{lying} \textbf{person.} & \textbf{C04\_step.on.person\_32}  \\
\textbf{2} 	& \textbf{--}& \textbf{+}& \textbf{--}& \textbf{--} 	& \textbf{33} 	\textbf{\textit{step} \textbf{on} \textbf{sth.}}& \textbf{Person} \textbf{stepping} \textbf{on} \textbf{a} \textbf{banana.} & \textbf{C20\_step.on.banana\_33} \\
2 		& -- 	& + 		& -- 		& -- 		& 34 		\textit{fall onto sth.}& Banana falling onto log.				&  P11\_fall.onto.log\_34  \\
\textbf{2} 	& \textbf{--}& \textbf{--}& \textbf{+} 	& \textbf{+} & \textbf{35} 	\textbf{\textit{be} \textbf{afraid} \textbf{of} \textbf{s.o.}} &\textbf{Child} \textbf{afraid} \textbf{of} \textbf{snake.} & \textbf{C08\_be.afraid.of.snake\_35} \\
2 		& -- 	& -- 		& + 		& + 		& 36 		\textit{bend person}& Rock bending someone's back. 				& P08\_bend.person\_36  \\
2 		& -- 	& -- 		& + 		& -- 		& 37 		\textit{hear s.o.}& A hears B calling out and turns head. 			& P12\_hear.person\_37  \\
\textbf{2} 	& \textbf{--}& \textbf{--}& \textbf{+} 	& \textbf{--} 	& \textbf{38} 	\textbf{\textit{bump} \textbf{into} \textbf{s.o.}} &\textbf{A} \textbf{bumping} \textbf{into} \textbf{B.} & \textbf{C13\_bump.into.person\_38} \\
2 		& -- 	& -- 		& -- 		& + 		& 39 		\textit{bend sth.} &Log lying on a plank bending it.				&  P14\_bend.plank\_39  \\
\textbf{2} 	& \textbf{--}& \textbf{--}& \textbf{--}& \textbf{+} & \textbf{40} \textbf{\textit{be} \textbf{afraid} \textbf{of} \textbf{sth.}}	& \textbf{Person} \textbf{afraid} \textbf{of} \textbf{axe.}&  \textbf{C19\_be.afraid.of.axe\_40}  \\
2 		& -- 	& -- 		& -- 		& -- 		& 41 		\textit{hear sth.} &A hears noise and turns head. & P06\_hear.noise\_41 \\
\textbf{2} 	& \textbf{--}& \textbf{--}& \textbf{--}& \textbf{--} 	& \textbf{42} 	\textbf{\textit{bump} \textbf{into} \textbf{sth.}} &\textbf{Person} \textbf{walking} \textbf{into} \textbf{a} \textbf{tree.} & \textbf{C16\_bump.into\_tree\_42} \\
\lspbottomrule
\end{tabularx}
\normalsize

\section*{Acknowledgments}
This chapter presents an overview of the results reported in two earlier papers, ``Conditions on pronominal marking in the Alor-Pantar languages'' by Sebastian Fedden, Dunstan Brown, Greville G. Corbett, Marian Klamer, Gary Holton, Laura C. Robinson and Antoinette Schapper, published in \textit{Linguistics} 51(1): 33-74 in March 2013, and ``Variation in pronominal indexing: lexical stipulation vs. referential properties in the Alor-Pantar languages'' by Sebastian Fedden, Dunstan Brown, Franti\v{s}ek Kratochv\'il, Laura C. Robinson and Antoinette Schapper published in \textit{Studies in Language} 38(1): 44-79 in May 2014. This chapter also has an additional section on the video elicitation to be used by others for further comparative work. We would like to thank the editor and two anonymous reviewers. The work reported here was supported under the European Science Foundation's EuroBABEL program (project ``Alor-Pantar languages: origin and theoretical impact''). At the time of this write-up, Fedden and Brown were funded by the Arts and Humanities Research Council (UK) under grant AH/K003194/1. We thank this funding body for its support. \textit{Correspondence address:} Sebastian Fedden, Surrey Morphology Group, School of English and Languages, University of Surrey, Guildford GU2 7XH, United Kingdom. E-mail: s.fedden@surrey.ac.uk.\largerpage[3]

\section*{Abbreviations}
 
\begin{tabular}{>{\sc}ll}
1 & 1st person   \\
2 & 2nd person \\
3 & 3rd person \\ 
act & actor \\
agt & agent \\
al & alienable \\
an & animate \\
ast & assistive \\
aux & auxiliary \\
ben & benefactive \\
compl & completive \\
contr\_foc & contrastive focus \\
dir & directional \\
dur & durative \\
excl & exclusive \\
exclam & exclamative \\
gen & genitive \\
\end{tabular} 
\begin{tabular}{>{\sc}ll}
goal & goal \\
incl & inclusive \\
ipfv & imperfective \\
loc & locative \\ 
pat & patient \\
pfv & perfective \\
pl & plural \\
pn & personal name \\
poss & possessive \\
real & realis \\
rec & recipient \\ 
seq & sequential \\
sg & singular \\
sim & simultaneous \\ 
spec & specific \\
top & topic \\
v & verb \\
\end{tabular} 