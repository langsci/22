%9
% examples done
% tables done
% bib done
% crossrefs

\documentclass[output=paper]{LSP/langsci}

% \setcitation{Klamer, Marian, Antoinette Schapper \& Greville Corbett}{Plural number words in the Alor-Pantar languages}{375--412}
% \renewcommand{\lsCollectionPaperCitationText}{\footnotesize\bottomcitation}

\title{Plural number words in the Alor-Pantar languages}
\author{Marian Klamer\and Antoinette Schapper \lastand Greville Corbett}
\abstract{In this chapter, we investigate the variation in form, syntax and semantics of the plural words found across the Alor-Pantar languages. We study five AP languages: Western Pantar, Teiwa, Abui, Kamang and Wersing. We show that plural words in Alor-Pantar family are diachronically instable: although proto-Alor-Pantar had a plural number word *non, many AP languages have innovated new plural words. Plural words in these languages exhibit not only a wide variety of different syntactic properties but also variable semantics, thus likening them more to the range exhibited by affixal plural number than previously recognized.}

\ChapterDOI{10.5281/zenodo.569397}

\maketitle
\lehead{Marian Klamer, Antoinette Schapper \& Greville Corbett}

\begin{document}
%9  
% examples done
% tables done
% bib done
% crossrefs
 
\section{Introduction}\label{sec:9:1}%1
 
\hypertarget{Toc376962648}{}
The majority of the world's languages express nominal plurality\is{plural (number) word} by affixation. After affixation, the use of independent plural\is{plural (number) word} words is the most widespread strategy: it is used in 16\% of Dryer's (2011) sample of 1066 languages.\nocite{Dryer2011} Yet, `plural\is{plural (number) word} words' have received remarkably little attention since their preliminary treatment in \citet{Dryer1989}. In this chapter, we build on  \citet{SchapperEtAl2011plural} in furthering the investigation of plural\is{plural (number) word} words using data from the Alor-Pantar (AP) languages, which are of great typological interest.

A plural\is{plural (number) word} word is ``a morpheme whose meaning and function is similar to that of plural\is{plural (number) word} affixes in other languages, but which is a separate word'' (\citealt[865]{Dryer1989}; \citealt[166]{Dryer2007}). Plural\is{plural (number) word} words are the most common example of a more general category, that of grammatical number words -- a number of languages employ singular or dual words as well as plural\is{plural (number) word} words. For Dryer, to be a plural\is{plural (number) word} word a lexeme must be the prime indicator of plurality\is{plural (number) word}: ``I do not treat a word as a plural\is{plural (number) word} word if it co-occurs with an inflectional indication of plural\is{plural (number) word} on the noun'' (1989: 867). Dryer further makes a distinction between `pure' number words and other kinds of number expressions: ``We can [...] distinguish `pure' plural\is{plural (number) word} words, which only code plurality\is{plural (number) word}, from articles that code number in addition to other semantic or grammatical features of the noun phrase, in which these articles are the sole indicator of number in noun phrases''. Thus the bar is set quite high: plural\is{plural (number) word} words are the prime indicator of plurality\is{plural (number) word}, and in the
pure case they have this as their unique function.

 Plural\is{plural (number) word} words in Alor-Pantar languages carry also a range of additional semantic connotations beyond simple plurality\is{plural (number) word}, including completeness, abundance, individuation, and partitivity. These are interrelated to the other options the individual languages have for marking plurality\is{plural (number) word}. This means that our discussion of plural\is{plural (number) word} words in Alor-Pantar languages necessarily also touches on other plurality\is{plural (number) word} expressing strategies available in the languages. We will see that the form, syntax and semantics of plural\is{plural (number) word} words across the Alor-Pantar languages display a high degree of diversity.

This paper is structured as follows. {\S} \ref{sec:9:2} introduces the lexical forms of the plural\is{plural (number) word} words of the languages and the sources of the data discussed in this paper. {\S} \ref{sec:9:3} discusses their syntax, while {\S} \ref{sec:9:4} looks in detail at the semantics of the plural\is{plural (number) word} words. {\S} \ref{sec:9:5} places AP plural\is{plural (number) word} words in a wider typological context, and {\S} \ref{sec:9:6} presents our conclusions.

\section{Plural\ist{plural (number) word} number words across Alor-Pantar}  %2
\label{sec:9:2}
Plural\is{plural (number) word} words are found across the Alor-Pantar languages, as shown in Table \ref{tab:9:1}. Cognate forms attested in Teiwa\il{Teiwa} (West Pantar), Klon\il{Klon} (West Alor) and Kamang\il{Kamang} (Central-East Alor) indicate that a plural\is{plural (number) word} word *non can be reconstructed for proto-Alor-Pantar (pAP)\il{proto-Alor-Pantar}. Western Pantar\il{Western Pantar}, Abui\il{Abui}, Wersing\il{Wersing}, Kula\il{Kula} and Sawila\il{Sawila} do not reflect this item, and instead appear to have innovated\is{innovation} new lexemes for plural\is{plural (number) word} words. Several AP languages in our sample (Klon\il{Klon}, Abui\il{Abui}, Wersing\il{Wersing}, Kula\il{Kula} and Sawila\il{Sawila}) have two plural\is{plural (number) word} words encoding different kinds of plurality\is{plural (number) word}, though the other languages do have a range of plural-marking\is{plural (number) word} strategies in addition to their plural\is{plural (number) word} word. There are also Alor-Pantar languages for which no plural\is{plural (number) word} word has been attested; an example is Kaera\il{Kaera} (North-East Pantar; \citealt{Klamertakaera}).



\begin{table}[t]

\begin{tabularx}{\textwidth}{p{1.5cm}p{1.5cm}L{2.3cm}Q}
\lsptoprule
\textbf{Language}  &\textbf{Reflecting} \textbf{*non} &\textbf{Not reflecting} \textbf{*non} &\textbf{Source}\\
\midrule
Western Pantar\ilt{Western Pantar} & &\textit{maru(ng)} &\citet{HoltonEtAl2008}; \citet{Holton2012,Holtontawesternpantar}, p.c.\\
\tablevspace
Teiwa\ilt{Teiwa} &\textit{non} & &\citet{Klamer2010grammar}, Teiwa\ilt{Teiwa} corpus; Schapper and \citet{Klamer2011}\\
\tablevspace
Adang\ilt{Adang} &\textit{nun} & &\citet{RobinsonEtAltaadang}\\
\tablevspace
Klon\ilt{Klon} &\textit{(o)non} &\textit{maang} &\citet{Baird2008}, Klon\ilt{Klon} corpus, p.c.\\
\tablevspace
Abui\ilt{Abui} & &\textit{loku,}\textit{we} &Schapper fieldnotes; \citet{Kratochvil2007}, Abui\ilt{Abui} corpus\\
\tablevspace
Kamang\ilt{Kamang} &\textit{nung} & &Schapper Kamang\ilt{Kamang} corpus, \citeyear{Schapperta}, fieldnotes;  \citet{SchapperEtAl2011kamus};  \citet{Stokhof1978,Stokhof1982}\\
\tablevspace
Wersing\ilt{Wersing} & &\textit{deing}, \textit{naing} & \citet{SchapperEtAltawersing}, fieldnotes,  Wersing\ilt{Wersing} corpus; \citet{Malikosand}\\
\tablevspace
Kula\ilt{Kula} & &\textit{du(a)}, \textit{araman} &Nicholas Williams p.c.\\
\tablevspace
Sawila\ilt{Sawila} & &\textit{do}, \textit{maarang} &Franti\v{s}ek Kratochv\'il p.c.\\
\lspbottomrule
\end{tabularx}
\caption{Cognate and non-cognate plural\ist{plural (number) word} words in Alor-Pantar languages}
\label{tab:9:1}
\end{table}

In all Alor-Pantar languages, nouns are uninflected for number, and a noun phrase without a plural\is{plural (number) word} word can refer to any number of individuals. For instance, Teiwa\il{Teiwa} \textit{qavif} `goat' in (\ref{bkm:Ref334184518}a) can be interpreted as either singular or plural\is{plural (number) word}, depending on the context. Those Alor-Pantar languages that have a plural\is{plural (number) word} word use it to express plurality\is{plural (number) word}: `more than one'. Illustrations are Teiwa\il{Teiwa} \textit{non} in (\ref{bkm:Ref334184518}b), and Klon\il{Klon} \textit{onon} in (\ref{bkm:Ref354060976}b-c). The plural\is{plural (number) word} word pluralizes\is{plural (number) word} the preceding nominal expression. In none of the AP languages we investigated is the plural\is{plural (number) word} word obligatory when plural\is{plural (number) word} reference is intended.

\ea
\label{bkm:Ref334184518}
\langinfo{Teiwa}{}{Klamer, Teiwa corpus} \\
\ea
\gll  Qavif ita{{\textglotstop}}{a} ma gi? \\
 goat where \textsc{obl} go \\
\glt `Where did the goat(s) go?'
\ex
\gll Qavif non ita{{\textglotstop}}a  ma gi? \\
 goat \textsc{pl} where \textsc{obl} go \\
\glt `Where did the (several) goats go?';  *`Where did the goat go?'
\z
\z

\newpage
\ea\label{bkm:Ref354060976}
\langinfo{Klon}{}{Baird, Klon corpus, p.c.} \\
  \ea
  \gll Ge-ebeng go-thook. \\
  3.\textsc{gen}-friend 3-meet \\
\glt `(He) met his friend(s).'\footnote{Compare \textit{Iniq ge-ebeng go-thook} `They met his friend(s)', where the non-singular pronoun\is{pronoun} \textit{iniq} encodes the subject (Baird, p.c. ).}
  \ex
  \gll Ge-ebeng onon go-thook. \\
  3.\textsc{gen}-friend \textsc{pl} 3-meet \\
\glt `His friends met him'/`(He) met his friends.'\\
  (*`(He) met his friend.'; *`(They) met their friend.')
  \ex
  \gll Ininok onon ge-ebeng go-thook. \\
  person  \textsc{pl} 3.\textsc{gen}-friend 3-meet \\
\glt `The people met their friend.'
  \z
\z

While plural\is{plural (number) word} words only occur with third person referents, none of the languages seems to have semantic restrictions on which referents can be marked plural\is{plural (number) word}. For instance, in all the languages we examined, both animate\is{animacy} and inanimate\is{animacy} entities can be pluralized\is{plural (number) word}. There does not seem to be a preference to use a plural\is{plural (number) word} word more often with animate\is{animacy} than with inanimate nouns, or vice versa. In Wersing\il{Wersing}, for example, the plural\is{plural (number) word} word can be used to signal the plurality\is{plural (number) word} of a human \REF{ex:9:3}, animal \REF{ex:9:4} or inanimate\is{animacy} referent \REF{ex:9:5}. There is similarly no difference in the plural\is{plural (number) word} marking of large versus small referents, as illustrated for Western Pantar\il{Western Pantar} \textit{raya} `chief' \REF{ex:9:6} and \textit{bal} `ball' \REF{ex:9:7}. \textit{Bal} \textit{marung} `ball \textsc{pl'} in \REF{ex:9:7} refers to an unspecified number of balls. This can be a small number of balls, say two or three; it does not have to be a large number of balls.




\ea%3
\label{ex:9:3}
\langinfo{Wersing}{}{\citealt[469]{SchapperEtAltawersing}} \\
\gll  {\dots}, saku deing bias ol tamu poko dein=a ge-pai ge-tai...\\
  {\dots} adult \textsc{pl} usually child grandchild small \textsc{pl=art} \textsc{3-}make 3-sleep\\
\glt `{\dots}, the adults would usually [do it] to make the children and grandchildren sleep{\dots}'
\z








\ea%4
\label{ex:9:4}
\langinfo{Wersing}{}{\citealt[469]{SchapperEtAltawersing}} \\
\gll Ne-karbau wari ne-wai deing=na yeta le-gadar.{\upshapefootnotemark}\\
  \textsc{1sg}\textsc{-}buffalo and \textsc{1sg-}goat \textsc{pl}=\textsc{foc} \textsc{2pl.agt} \textsc{appl-}guard\\
\glt `You watch out for my buffaloes and my goats.' 
\z







\ea%5
\label{ex:9:5}
\langinfo{Wersing}{}{\citealt[469]{SchapperEtAltawersing}} \\
\gll Kiki deing aso ge-mira susa. \\
  flower \textsc{pl} also 3-inside suffer  \\
\glt `The flowers were also suffering.'
 \z








\ea%6
\label{ex:9:6}
\langinfo{Western Pantar}{}{\citealt{Holton2012}} \\
\gll  Raya marung wang hundar. \\
   chief \textsc{pl} exist amazed  \\
\glt `The chiefs were amazed.' (*`The chief is amazed.')
\z







\ea%7
\label{ex:9:7}
\langinfo{Western Pantar}{}{\citealt{Holton2012}} \\
\gll Bal  marung mea tang pering.  \\
  ball \textsc{pl} table on pour   \\
\glt `A bunch of balls are spread out on the table.'
\z


\footnotetext{Here the plural\is{plural (number) word} word must have scope over both nouns, such that this example cannot be read to mean ``my buffalo and my goats''.}





Where the plural\is{plural (number) word} words do differ from plural\is{plural (number) word} affixes in other languages is in their shape and distribution: they are for the most part free word forms, and they need not occur next to the noun they pluralize\is{plural (number) word}. This is illustrated in \REF{ex:9:8}, where Teiwa\il{Teiwa} \textit{non} occurs next to the adjective \textit{sib} `clean' while pluralising\is{plural (number) word} \textit{gakon} `his shirt'. Similarly in \REF{ex:9:9} we see Adang\il{Adang} \textit{nun} follows the verb \textit{mat}\textit{{\textepsilon}} `large' modifying the head noun \textit{ti} `tree'.

\xbox{\textwidth}{
\ea%8
\langinfo{Teiwa}{}{Klamer, Teiwa corpus} \\
\label{ex:9:8}
\gll  Uy masar ga-kon  sib non ga{{\textglotstop}}{an,} ma {tona}{{\textglotstop}}{.}\\
   person male \textsc{3sg.poss\ist{possession}}-shirt clean \textsc{pl} \textsc{dem} come collect\\
\glt `Those clean shirts of that man, collect them.'
\z
}






\ea%9
\langinfo{Adang}{}{\citealt[252]{RobinsonEtAltaadang}} \\
\label{ex:9:9}
\gll Pen ti {mat}{{\textepsilon}} nun {\textglotstop}a-b{{\textopeno}}{{\textglotstop}}{{\textopeno}}{i.} \\
   Pen tree large \textsc{pl} \textsc{3incl.obj}-cut \\
\glt `Pen cut some large trees.'
\z






Plural\is{plural (number) word} words in AP languages cannot co-occur with a numeral in a single NP. For instance, in Teiwa\il{Teiwa}, a noun can be pluralized\is{plural (number) word} with either a plural\is{plural (number) word} word or with a numeral (plus optional classifier\is{numeral classifier}) (\ref{ex:9:10}a-b), but not with both at the same time (\ref{ex:9:10}c). Adang\il{Adang} shows the same restriction; the plural\is{plural (number) word} word \textit{nun} cannot co-occur with a numeral, compare (\ref{ex:9:11}a-b).\footnote{A combination of a mass noun and a numeral is also ungrammatical: *\textit{s}\textit{{\textepsilon}}\textit{i} \textit{ut} `water four' \citep[296]{Haan2001}.}


\ea%bkm:Ref354063073
\label{ex:9:10}
\langinfo{Teiwa}{}{Klamer, Teiwa corpus} \\
\ea
\gll  war non\\
   rock \textsc{pl}  \\
\glt `(several/many) rocks'
\ex
\gll war (bag) {haraq}\\
 rock \textsc{clf} two  \\
\glt   `two rocks'
\ex
\gll *war (bag) haraq non \\
 rock \textsc{clf} two \textsc{pl} \\
 \glt Intended: `two rocks'
\z
\z



\ea%11
\label{ex:9:11}
\langinfo{Adang}{}{\citealt[253]{RobinsonEtAltaadang}} \\
\ea
\gll {s}{\textepsilon}{i} nun ho {\textglotstop}uhu{\textltailn} {\textepsilon}  b{\textepsilon}{\ng} tanib \\
   water \textsc{pl} \textsc{def} pour and other draw.water.from.well  \\
\glt `Pour out that little bit of water and get some more from the well.'
\ex
\gll *s{\textepsilon}{i} nun al{\textopeno} ho {\textglotstop}uhu{\textltailn} {\textepsilon} b{\textepsilon}{\ng} tanib \\
   water \textsc{pl} two \textsc{def} pour and other draw.water.from.well  \\
\glt  Intended: `Pour out the two bits of water and get some more from the well.'
\z
\z

In sum, proto-Alor-Pantar\il{proto-Alor-Pantar} had a plural\is{plural (number) word} word of the shape *non. Some Alor-Pantar languages inherited both form and function, others innovated\is{innovation} a plural\is{plural (number) word} word. The languages under investigation do not show restrictions on which referents can be marked plural\is{plural (number) word}, and in none of the languages does the plural\is{plural (number) word} word co-occur with a numeral in an NP.

\section{Syntax of plural\ist{plural (number) word} words in Alor-Pantar}  %3

\label{sec:9:3}
The plural\is{plural (number) word} words investigated in \citet{Dryer1989} are very heterogeneous in their categorial properties. They belong to one of the following classes: (i) articles; (ii) numerals; (iii) grammatical number words like singular, dual, trial; (iv) a closed class of noun modifiers; and (v) a class of their own. Dryer concludes that ``there is little basis for using the term [plural\is{plural (number) word} word] as a syntactic category'' (1989: 879).

In this section, we investigate the syntax of plural\is{plural (number) word} words in Western Pantar\il{Western Pantar} ({\S} \ref{sec:9:3.1}), Teiwa\il{Teiwa} ({\S} \ref{sec:9:3.2}), Kamang\il{Kamang} ({\S} \ref{sec:9:3.3}), Abui\il{Abui} ({\S} \ref{sec:9:3.4}) and Wersing\il{Wersing} (3.5). For each language, we describe the template of the NP as well as the position and combinatorial properties of the plural\is{plural (number) word} word. We confirm Dryer's observation that there is little syntactic unity in plural\is{plural (number) word} words across languages. Our description focuses on the following issues:

\begin{enumerate}
\item Does the plural\ist{plural (number) word} word occur in the NP?
\item  How does the plural\ist{plural (number) word} word behave in respect to quantifiers in the NP?
\item   Can the plural\ist{plural (number) word} word alone form an NP?
\end{enumerate}

The languages under discussion differentiate the plural\is{plural (number) word} word from other syntactic classes. We will see that significant variation exists in terms of which syntactic class the plural\is{plural (number) word} word class resembles most. In Wersing\il{Wersing}, the plural\is{plural (number) word} word shares many properties with nouns, while in Kamang\il{Kamang} the plural\is{plural (number) word} word is most similar to pronouns\is{pronoun}. In Western Pantar\il{Western Pantar} and Teiwa\il{Teiwa}, the plural\is{plural (number) word} words are comparable with numerals and quantifiers.

\subsection{Western Pantar\ilt{Western Pantar}}  %3.1
\label{sec:9:3.1}
The template of the Western Pantar\il{Western Pantar} NP is presented in \REF{ex:9:12} \citep{Holtontawesternpantar}. The NP is maximally composed of a head noun (N) followed by an adjective in the attribute slot (\textsc{Attr),} followed by numeral phrases with an optional classifier\is{numeral classifier} (\textsc{(Clf)} \textsc{Num)} or a plural\is{plural (number) word} word (\textsc{Pl),} a demonstrative\is{demonstrative} \textsc{(Dem)} and an article \textsc{(Art)}. 

\ea\label{ex:9:12}
\upshape
Template of the Western Pantar\il{Western Pantar} NP\footnote{Western Pantar\ilt{Western Pantar} does not have relative clauses.}

 [\textsc{N  Attr\{(Clf) Num / Pl\}Dem  Art]}\textsc{\textsubscript{\upshape NP}}
 
\z


Western Pantar\il{Western Pantar} has no dedicated slot for (non-numeral) quantifiers, as these behave like adjectives or like nouns: adjectival quantifiers go in the A\textsc{ttr} slot \REF{ex:9:13}, while nominal quantifiers occur in apposition to the NP, to the right of the article \REF{ex:9:14}.



\ea%13
\label{ex:9:13}
\langinfo{Western Pantar}{}{\citealt{Holtontawesternpantar}} \\
\gll  Wakke-wakke haweri wang Tubbe birang kalalang. \\
   child{\Tilde}\textsc{rdp} many exist T. speak know  \\
\glt `Most/many children can speak the Tubbe language.'
\z







\ea%14
\label{ex:9:14}
\langinfo{Western Pantar}{}{\citealt{Holtontawesternpantar}} \\
\gll {\ob}{\ob}{Hai} bloppa sing{\cb}\textsubscript{\upshape NP} {der}{\cb}\textsubscript{\upshape NP} ga-r diakang.  \\
  \textsc{2sg.poss\ist{possession}} weapon \textsc{art} some 3\textsc{sg-}with descend  \\
\glt `Bring down some of your weapons.' [publia152]
\z






Nominal plurality\is{plural (number) word} is expressed by the plural\is{plural (number) word} word \textit{maru(ng)}, \REF{ex:9:15}.\footnote{\textit{Marung} has cognate forms in three AP languages: Klon\il{Klon} \textit{maang}, Kula\il{Kula} \textit{araman} (with liquid nasal metathesis) and Sawila\il{Sawila} \textit{maarang} \citep{SchapperEtAlms}} The use of numerals is illustrated in \REF{bkm:Ref334530759}, and \REF{ex:9:17}-\REF{ex:9:18} show that numeral and plural\is{plural (number) word} word do not co-occur in a single NP.


\ea%15
\label{ex:9:15}
\langinfo{Western Pantar}{}{\citealt{Holton2012}} \\
\gll  Bal  marung mea tang pering. \\
 ball \textsc{pl} table on pour   \\
\glt `A bunch of balls are spread out on the table.'
\z







\ea %16
\label{bkm:Ref334530759}
\langinfo{Western Pantar}{}{\citealt{Holton2012}} \\
\gll Bal ara atiga, kalla yasing, mea tang {ti}{{\textglotstop}}{ang.} \\
 ball large three small five table on  set \\
\glt `Three large balls and five small balls are sitting on the table.'
\z



\ea%17
\label{ex:9:17}
\langinfo{Western Pantar}{}{\citealt{Holton2012}} \\
\ea
\gll *ke{\textglotstop}e kealaku maru \\
   fish  twenty \textsc{pl} \\
\glt Intended: \textsc{`}twenty fish'
\ex
\gll *ke{\textglotstop}e bina maru \\
 fish \textsc{clf} \textsc{pl} \\
\glt Intended: `twenty fish'
\z
\z


\ea%18
\label{ex:9:18}
\langinfo{Western Pantar}{}{\citealt{Holton2012}} \\
\ea
\gll *ke{\textglotstop}e  maru kealaku \\
   fish  \textsc{pl} twenty \\
\glt Intended: `twenty fish'
\ex 
\gll *ke{\textglotstop}e maru  bina \\
 fish \textsc{pl} \textsc{clf} \\
 \glt Intended: `twenty fish'
\z
\z

\textit{Maru(ng)} cannot substitute for a whole NP and function independently as a verbal argument, compare (\ref{ex:9:19}a) and (\ref{ex:9:19}b).


\ea%19
\label{ex:9:19}
\langinfo{Western Pantar}{}{\citealt{Holton2012}} \\
\ea
\gll  Raya marung lama ta. \\
   chief \textsc{pl} walk \textsc{ipfv}  \\
\glt `The chiefs walk.'
\ex
\gll *Marung lama ta. \\
  \textsc{pl} walk \textsc{ipfv}  \\
\glt Intended: `They walk.'
\z
\z


 In sum, Western Pantar\il{Western Pantar} \textit{marung} can only be used as a nominal attribute within an NP. It is in complementary distribution with adjectival quantifiers and numerical expressions and lacks nominal properties.

\subsection{Teiwa\ilt{Teiwa}} %3.2
\label{sec:9:3.2}
The template of the Teiwa\il{Teiwa} NP is presented in \REF{ex:9:20}. The NP is maximally composed of a head noun (N) followed by an attributive (\textsc{Attr)} noun, derived nominal or adjective\textsc{,} followed by a numeral phrase (indicated by \{\})consisting of either a numeral with an optional classifier\is{numeral classifier} (\textsc{(Clf)} \textsc{Num)} or a plural\is{plural (number) word} word with an optional quantifier (\textsc{Pl} \textsc{(Q)),} a demonstrative\is{demonstrative} \textsc{(Dem)} and a demonstrative\is{demonstrative} particle in the article (\textsc{Art)} slot.

\xbox{\textwidth}{
\ea%20
\label{ex:9:20}
\upshape
 Template of the Teiwa\il{Teiwa} NP\footnote{Teiwa\ilt{Teiwa} has no relative clauses; nominal referents are focused with the focus particle \textit{la} \citep{Klamer2010grammar}.}

 [\textsc{N  Attr\{(Clf) Num / Pl (Q)\}  Dem Art ]}\textsc{\textsubscript{\upshape NP}}
\z
}

In the \textsc{Dem} slot, we often find \textit{ga}\textit{{\textglotstop}}\textit{an} (glossed as `that.\textsc{knwn}'), a 3\textsc{sg} object pronoun\is{pronoun} that also functions as a demonstrative\is{demonstrative} modifier of nouns. In the \textsc{Art} slot are the demonstrative\is{demonstrative} particles \textit{u} `\textsc{distal'} and \textit{a} `\textsc{proximate'}. These particles occupy the NP-final position, marking definiteness and/or the location of NP referent with respect to the speaker.

The plural\is{plural (number) word} word has its own slot within the NP. It cannot combine with numeral constituents as those in (\ref{ex:9:21}a); compare (\ref{ex:9:21}b) with (\ref{ex:9:22}a-c). However, \textit{non} can be combined with a quantifier in an NP, as shown in \REF{ex:9:23} and \REF{ex:9:24}. Note that \textit{dum} `many/much' is used contrastively here.


\ea%bkm:Ref354653126
\label{ex:9:21}
\langinfo{Teiwa}{}{Klamer, Teiwa corpus} \\
\ea
\gll \textit{war} \textit{(bag)} \textit{haraq}  \\
  rock \textsc{clf} two \\
\glt `two rocks'
\ex
\gll \textit{war} \textit{non}\\
  rock \textsc{pl}\\
\glt `rocks'
\z
\z


\ea%bkm:Ref335059934
\label{ex:9:22}
\langinfo{Teiwa}{}{Klamer, Teiwa corpus} \\
\ea
\gll  *war haraq non \\
  rock two \textsc{pl}   \\
\glt Intended: `two rocks'
\ex
\gll *war bag haraq non \\
  rock \textsc{clf} two \textsc{pl}   \\
\glt  Intended: `two rocks'
\ex
\gll *war bag non \\
   rock \textsc{clf} \textsc{pl}  \\
\glt Intended: `two rocks'
\z
\z





\ea%bkm:Ref354653269
\label{ex:9:23}
\langinfo{Teiwa}{}{Klamer, Teiwa corpus} \\
\gll Hala {\ob}{qavif} non {dum}{\cb}\textsubscript{\upshape NP} pin aria{{\textglotstop}}{?} \\
  someone goat \textsc{pl} many hold arrive   \\
\glt `Were many [rather than few] goats brought here?'
\z







\ea%24
\label{ex:9:24}
\langinfo{Teiwa}{}{Klamer, Teiwa corpus} \\
\gll  {\ob}{Wat} non {dum}{\cb}\textsubscript{\upshape NP} usan ma! \\
  coconut \textsc{pl} many pick.up come  \\
\glt `Pick up the many coconuts.' [situation: there are many coconuts in a pile of various kinds of fruits, and the order is to pick up these, not the rest]
\z






\textit{Non} does not substitute for an NP and cannot function independently as a verbal argument, either with or without the distal demonstrative\is{demonstrative} particle \textit{u} that functions as an (grammatically optional) article in (\ref{ex:9:25}b-c). It must always remain part of the NP, as shown by the ungrammaticality of (\ref{ex:9:25}d).


\ea%25
\label{ex:9:25}
\langinfo{Teiwa}{}{Klamer, Teiwa corpus} \\
\ea
\gll {\ob}{G-oqai}   non u{\cb}\textsubscript{\upshape NP} min-an tau. \\
   \textsc{3sg}-child \textsc{pl} \textsc{dist} die-\textsc{real} \textsc{pfv} \\
\glt `Her children (lit. those her children) have died.'
\ex
\gll *{\ob}Non u{\cb}\textsubscript{\upshape NP} min-an tau. \\
   \textsc{pl} \textsc{dist} die-\textsc{real} \textsc{pfv} \\
\glt  Intended: `They have died.'
\ex
\gll *{\ob}{Non}{\cb}\textsubscript{\upshape NP} min-an tau. \\
    \textsc{pl} die-\textsc{real} \textsc{pfv} \\
\glt  Intended: `They have died.'
\ex
\gll *{\ob}{G-oqai}   {u}{\cb}\textsubscript{\upshape NP} non min-an tau. \\
  \textsc{3sg}-child \textsc{dist} \textsc{pl} die-\textsc{real} \textsc{pfv}  \\
\glt  Intended: `Her children (they) have died.'
\z
\z






Just as Western Pantar\il{Western Pantar} \textit{maru(ng)}, Teiwa\il{Teiwa} \textit{non} can occur in an NP that stands in apposition with a pronoun\is{pronoun} \REF{ex:9:26}:


\ea%26
\label{ex:9:26}
\langinfo{Teiwa}{}{Klamer, Teiwa corpus} \\
\gll {\ob}Kemi  {non}{\cb}\textsubscript{\upshape NP} iman xap gu-uyan mat... \\
  ancestor \textsc{pl} they bride  3.\textsc{obj-}search take  \\
\glt `(Our) ancestors (they) searched for brides...'
\z






It is possible for an NP with \textit{non} to be part of the subject of numeral predication if the numeral predicate also contains a classifier\is{numeral classifier}, as illustrated in \REF{ex:9:27}, where \textit{bag} is the generic numeral classifier\is{numeral classifier} \citep{Klamer2014history} and combines with \textit{tiaam} `six'. The plural\is{plural (number) word} word \textit{non} is part of the subject NP, and is grammatically optional. Subjects pluralized\is{plural (number) word} with \textit{non} can thus occur with a numeral predicate.

However, an NP with \textit{non} cannot be the subject of a quantifier predication with \textit{dum} `many/much', compare (\ref{ex:9:28}a-b). This is because the Teiwa\il{Teiwa} plural\is{plural (number) word} word \textit{non} often has the connotation of `many' and `plenty' (see {\S} \ref{sec:9:4.2}). A subject NP like the one in \REF{ex:9:28} already implies that there are `many/plenty goats', so that combining it with a predicate `be many' in (\ref{ex:9:28}b) is semantically redundant.


\ea%27
\label{ex:9:27}
\langinfo{Teiwa}{}{Klamer, Teiwa corpus} \\
\gll {\ob}Ga-qavif (non){\cb}\textsubscript{\upshape NP} {\ob}{un} bag {tiaam}{\cb}\textsubscript{\upshape Pred} \\
    \textsc{3sg}-goat \textsc{pl} \textsc{cont} \textsc{clf} six \\
\glt `His goats are six.'
\z







\ea%28
\label{ex:9:28}
\langinfo{Teiwa}{}{Klamer, Teiwa corpus} \\
\ea
\gll {\ob}{Ga-qavif}{\cb}\textsubscript{\upshape NP} {\ob}un dum{\cb}\textsubscript{\upshape Pred}  \\
 \textsc{3sg}-goat \textsc{cont} many   \\
\glt `His goats are many.'
\ex
\gll *{\ob}Ga-qavif non{\cb}\textsubscript{\upshape NP} {\ob}un dum{\cb}\textsubscript{\upshape Pred}  \\
    \textsc{3sg}-goat \textsc{pl} \textsc{cont} many\\
\glt Intended: `His many/plenty goats are many.'
\z
\z






The fact that \textit{non} does not combine with a numeral in a single NP suggests that it patterns with the numeral word class. However, unlike numerals, \textit{non} cannot combine with a classifier\is{numeral classifier}. On the other hand, \textit{non} can combine with the quantifier \textit{dum} `much/many' in a single NP, which a numeral cannot do. However, at the same time, \textit{non} does not pattern with the class of quantifiers for two reasons. First, such quantifiers can occur as predicates, while \textit{non} cannot, (\ref{ex:9:29}a-b); and second, non-numeral quantifiers can occur both inside the NP (\ref{ex:9:30}a) as well as outside of it, adjacent to the verb (\ref{ex:9:30}b), while \textit{non} must remain within the NP. In (\ref{ex:9:30}c) the NP contains \textit{non}, and the ungrammaticality of (\ref{ex:9:30}d) shows that \textit{non} cannot occur in the position adjacent to the verb.


\ea%bkm:Ref334184526
\label{ex:9:29}
\langinfo{Teiwa}{}{Klamer, Teiwa corpus} \\
\ea
\gll Masar {\ob}un dum{\cb}\textsubscript{\upshape Pred} \\
    male \textsc{cont} many \\
\glt  `There are many men.' (Lit. `Males are [being] many.')
\ex
\gll *Masar {\ob}un non{\cb}. \\
   male \textsc{cont} \textsc{pl}  \\
\glt  Intended: `There are many/several males.'
\z
\z







\ea%bkm:Ref334184556
\label{ex:9:30}
\langinfo{Teiwa}{}{Klamer, Teiwa corpus} \\
\ea
\gll  {\ob}{Qavif} dum {ga}{{\textglotstop}}{an}{\cb}\textsubscript{\upshape NP} hala tatax. \\
   goat many that.\textsc{knwn} someone chop  \\
 \glt `Many (known) goats were chopped up.'
\ex
\gll {\ob}Qavif ga{{\textglotstop}}{an}{\cb}\textsubscript{\upshape NP} hala dum tatax. \\
   goat that.\textsc{knwn} someone many chop  \\
\glt `Many of these (known) goats were chopped up.'
\ex
\gll {\ob}{Qavif} non {ga}{{\textglotstop}}{an}{\cb}\textsubscript{\upshape NP} hala tatax. \\
   goat \textsc{pl} that.\textsc{knwn} someone chop  \\
\glt  `These (known) goats were chopped up by someone.'
\ex
\gll *{\ob}Qavif ga{{\textglotstop}}{an}{\cb}\textsubscript{\upshape NP} hala non tatax \\
   goat that.\textsc{knwn} someone \textsc{pl} chop  \\
\glt  Intended: `These (known) goats were chopped up.'
\z
\z


In sum, Teiwa\il{Teiwa} \textit{non} does not have any nominal properties, shares some of the distributional properties of numerals and quantifiers, and constitutes its own syntactic class.\footnote{In addition to the plural\is{plural (number) word} word, Teiwa\il{Teiwa} has four dedicated pronoun\is{pronoun} series for referents of different quantificational types: (i) the dual paradigm (\textit{we} \textit{two}, etc.), 
(ii) the ``X and they'' paradigm (\textit{you (sg/pl) and they}, 
\textit{s/he/they and they};
\textit{I/we (incl/excl) and they}), 
(iii) the ``X alone'' paradigm (\textit{I} \textit{alone},\textit{you} \textit{alone}, etc.) and (iv) the ``X as a group of ...'' paradigm (\textit{we/you/they as a group of x numbers}) \citep[82-85]{Klamer2010grammar}. The plural\is{plural (number) word} word cannot co-occur with these pronouns\is{pronoun}. Teiwa\il{Teiwa} has no associative plural\is{plural (number) word} word. To express associative plural\is{plural (number) word} notions, a form from the special pronoun\is{pronoun} series ``X and they'' is used, e.g., \textit{Rini} \textbf{\textit{i-qap}} \textit{a-kawan aria' wad } `Rini \textbf{3-and.they} 3-friend
arrive today',  `\textit{Today} \textit{Rini arrived with her friends}'.}

\subsection{Kamang\ilt{Kamang}} %3.3
\label{sec:9:3.3}
The template of the Kamang\ilt{Kamang} noun phrase (NP) is presented in \REF{ex:9:31}. The NP is maximally composed of a head noun (\textsc{N}) followed by its attribute (\textsc{Attr),} a numeral phrase \textsc{(Num)}, a relative clause (\textsc{Rc}), a demonstrative\is{demonstrative} \textsc{(Dem)} and an article \textsc{(Art)}. The article marks the right edge of an NP and is used to nominalize\is{nominalization} (i.e., create NPs from) clauses and other non-nominal phrases in the language. In addition, a Kamang\il{Kamang} NP can occur with a range of items co-referential with it in a slot outside the NP, called here the NP-appositional (\textsc{Appos)} slot (discussed further below). The apposition between an NP and an item in the NP-appositional slot is syntactically tight: there is no intonational break or pause between NP and appositional item, and no item may intervene between them. For more details on the status of the \textsc{Appos} slot or for discussion of the other NP slots, see \citet{Schapperta}.

\ea%31
\label{ex:9:31}
\upshape
Template of the Kamang\il{Kamang} NP  \citep{Schapperta}\\	

 [\textsc{N}\textsc{\textsubscript{\upshape HEAD}}\textsc{ Attr  NumP  Rc  Dem  Art]}\textsc{\textsubscript{\upshape NP}} \textsc{\textsubscript{\upshape Appos}}

\z


The Kamang\il{Kamang} plural\is{plural (number) word} word \textit{nung} is conspicuously absent from the template in \REF{ex:9:31}. In Kamang\il{Kamang} \textit{nung} does not occur within the NP, but directly follows it. That is, it occurs to the right of the NP article, where one is expressed. For example, in \REF{ex:9:32} and \REF{ex:9:33} \textit{nung} follows the specific (`\textsc{spec}') and definite (`\textsc{def}') articles respectively. The alternative order with the article following \textit{nung} is not grammatical: *\textit{nung=a} `\textsc{pl=spec}' and *\textit{nung=ak} `\textsc{pl=def'}. In short, \textit{nung} only occurs in the NP-appositional slot.


\ea%32
\label{ex:9:32}
\langinfo{Kamang}{}{Schapper, fieldnotes} \\
\gll  Almakang laising-laung=a nung {ye}{{\textglotstop}}{-baa} sue. \\
  people youthful=\textsc{spec} \textsc{pl} \textsc{3.sben}-say arrive  \\
\glt `Go tell the young people to come.'
\z







\ea%33
\label{ex:9:33}
\langinfo{Kamang}{}{Schapper, fieldnotes} \\
\gll  Muut=ak nung iduka. \\
  citrus=\textsc{def} \textsc{pl} sweet   \\
\glt `The citrus fruits are sweet.'
\z






By contrast, other Kamang\il{Kamang} quantifiers can occur within the NP, i.e., to the left of the NP-defining article. Non-numeral quantifiers such as \textit{adu} `many/much' occupy the \textsc{Attr} slot within the NP and cannot float out of it, as seen in \REF{ex:9:34}.


\ea%34
\label{ex:9:34}
\langinfo{Kamang}{}{Schapper, fieldnotes} \\
\ea
\gll sibe adu=a \\
  chicken many=\textsc{spec} \\
\glt `the many chickens' 
\ex
\gll *{sibe=a} {adu} \\
chicken=\textsc{spec} many \\
\glt Intended: `the many chickens'
\z
\z



Kamang\il{Kamang} does not have a syntactic class of non-numeral quantifiers; items denoting \textit{many}, \textit{few}, \textit{a} \textit{little}, etc. are adjectives and occur in the \textsc{Attr} slot of the NP. Numeral quantifiers occur with a classifier\is{numeral classifier} in the \textsc{NumP}. The unmarked position for the \textsc{NumP} is within the NP to the left of the article (\ref{ex:9:35}a), and the marked position is post-posed into the NP-appositional slot outside the NP (\ref{ex:9:35}b). The latter position is less frequent and pragmatically marked, functioning to topicalize the enumeration of the NP referent.

\xbox{\textwidth}{
\ea%35
\label{ex:9:35}
\langinfo{Kamang}{}{Schapper, fieldnotes} \\
\ea
\gll  sibe {\ob}{uh}   {su}{\cb}\textsc{\textsubscript{\upshape NumP}}{=a}\\
  chicken \textsc{clf } three=\textsc{spec} \\
 \glt `the three chickens' 
\ex
\gll sibe=a {\ob}uh su{\cb}\textsc{\textsubscript{\upshape NumP}} \\
 chicken=\textsc{spec} \textsc{clf} three  \\
\glt `the chickens, the three ones'
\z
\z
}

The plural\is{plural (number) word} word shares distributional properties in common not only with a \textsc{NumP} but also with a pronoun\is{pronoun}, since the NP-appositional position can also host a pronoun\is{pronoun}. In \REF{ex:9:36} we see that a pronoun\is{pronoun} (\ref{ex:9:36}a) and a plural\is{plural (number) word} word (\ref{ex:9:36}b) respectively can both occur in the slot following an NP. In these examples, the parts of the free translations in curly brackets are the semantics contributed by the items in the appositional slot.


\ea%36
\label{ex:9:36}
\langinfo{Kamang}{}{Schapper, fieldnotes} \\
\ea
\gll almakang=ak gera  \\
 people=\textsc{def} \textsc{3.contr}   \\
\glt `the \{specific group of\} people \{not some other group\}'
\ex
\gll almakang=ak nung \\
  people=\textsc{def} \textsc{pl}   \\
\glt  `the \{multiple\} people'
\z
\z






The Kamang\il{Kamang} plural\is{plural (number) word} word has a distribution similar to that of an NP in two respects. Firstly, \textit{nung} can substitute for a whole NP, where reference is sufficiently clear. For instance, in \REF{ex:9:37} \textit{nung} is the sole element representing the S of the verb \textit{sue} `come'. Secondly, like an NP, a plural\is{plural (number) word} word can itself occur with a pronoun\is{pronoun} in the NP appositional slot where no NP is expressed, as in \REF{ex:9:38}.


\ea%37
\label{ex:9:37}
\langinfo{Kamang}{}{Schapper, fieldnotes} \\
\gll  {\ob}Nung{\cb}\textsubscript{\upshape NP} sue. \\
   \textsc{pl} arrive  \\
\glt `\{Multiple\} (people) arrived.'
\z







\ea%38
\label{ex:9:38}
\langinfo{Kamang}{}{Schapper, fieldnotes} \\
\gll  {\ob}{Nung}{\cb}\textsubscript{\upshape \textsc{NP}} {gera}\textsubscript{\upshape \textsc{appos}} sue. \\
   \textsc{pl} \textsc{3.contr} arrive  \\
\glt `\{Multiple other\} (people) arrived.'
\z






\textit{Nung} is not compatible with any other quantificational items. That is, despite its occurring outside the NP, marking an NP with \textit{nung} means that other quantificational items cannot occur in the NP. This is seen in the examples in \REF{ex:9:39} where \textit{nung} cannot grammatically co-occur with the numeral quantifier \textit{su} `three' (\ref{ex:9:39}a) and with the non-numeral quantifier \textit{adu} `many' (\ref{ex:9:39}b).

\newpage
\ea%39
\label{ex:9:39}
\langinfo{Kamang}{}{Schapper, fieldnotes} \\
\ea
\gll  *{sibe} uh su {nung}\\
   chicken \textsc{clf} three \textsc{pl} \\
\glt Intended: `three chickens'
\ex
\gll
*\textit{sibe} \textit{adu} \textit{nung}\\
    chicken many   \textsc{pl}\\
\glt  Intended: `many chickens'
 \z
 \z




In addition to the plural\is{plural (number) word} word, Kamang\il{Kamang} has multiple dedicated quantificational pronoun\is{pronoun} series to signal different quantificational types.\footnote{There are four ``quantifying'' pronominal\is{pronoun} paradigms in Kamang\il{Kamang}: (i) the ``alone'' paradigm (\textit{I alone/on my own}, \textit{we alone/on our own}, etc.), (ii) the dual paradigm (\textit{we} \textit{two}, etc.- only in non-singular numbers), (iii) the ``all'' paradigm (\textit{we} \textit{all}, etc.- only in non-singular numbers), and (iv) the ``group'' paradigm (\textit{we together in a group}, etc.- only in non-singular numbers). See \citet{Schapperta} for full set of Kamang\il{Kamang} pronominal\is{pronoun} paradigms.} For instance, we see the third person pronouns\is{pronoun} forms for group plurality\is{plural (number) word} and universal quantification in \REF{ex:9:40} and \REF{ex:9:41} respectively. The plural\is{plural (number) word} word cannot co-occur with these pronouns\is{pronoun}.


\ea%40
\langinfo{Kamang}{}{Schapper, fieldnotes} \\
\label{ex:9:40}
\gll  Geifu   loo maa.  \\
  3.\textsc{group} walk go   \\
\glt `They go together (as a group).'
\z






\ea%41
\label{ex:9:41}
\langinfo{Kamang}{}{Schapper, fieldnotes} \\
\gll Gaima bisa wo-ra=bo pilan. \\
  3.\textsc{all} can   3.\textsc{loc}-wear=\textsc{lnk} lego-lego  \\
\glt `They all can wear (them) and dance in a lego-lego.'
\z






The use of quantificational pronouns\is{pronoun} with NPs is illustrated in \REF{ex:9:42} and \REF{ex:9:43}. We see in these examples that the quantificational pronouns\is{pronoun} fill the appositional slot in the same manner as the plural\is{plural (number) word} word \textit{nung} and signal the plurality\is{plural (number) word} of the referents of the preceding NP.


\ea%42
\label{ex:9:42}
\langinfo{Kamang}{}{Schapper, fieldnotes} \\
\gll   {\ob}Mane ang{\cb}\textsubscript{\upshape \textsc{NP}} {geifu}\textsubscript{\upshape \textsc{appos}}   {mauu.}\\
  village \textsc{dem} 3.\textsc{group} war  \\
\glt `Those villages make war together (against another village).'
\z






\ea%43
\label{ex:9:43}
\langinfo{Kamang}{}{Schapper, fieldnotes} \\
\gll  {\ob}Arita pang{\cb}\textsubscript{\upshape NP} {gaima}\textsubscript{\upshape \textsc{appos}} luaa-ra lai-ma{.}\\
    leaf \textsc{dem}   3.\textsc{all} whither-\textsc{aux} finished-\textsc{pfv}\\
\glt `All the leaves have withered completely.'
\z






Finally, Kamang\il{Kamang} has a suffix marking associative plurality\is{plural (number) word}, \textit{-lee} `\textsc{assoc}'. This suffix can occur on kin terms or proper names, as in \REF{ex:9:44} and \REF{ex:9:45} respectively. Nouns marked by \textit{-lee} cannot be modified by any other NP elements. The plural\is{plural (number) word} word \textit{nung} does not occur in such contexts.


\ea%44
\label{ex:9:44}
\langinfo{Kamang}{}{Schapper, fieldnotes} \\
\gll  {\dots},   ge-dum-lee see silanta malii \\
  {\dots} 3.\textsc{gen}-child-\textsc{assoc} arrive mourn mourn \\
\glt `{\dots}, her children and their associates come to mourn.'
\z







\ea%45
\label{ex:9:45}
\langinfo{Kamang}{}{Schapper, fieldnotes} \\
\gll  Marten-lee n-at tak.  \\
  Marten-\textsc{assoc} 1\textsc{sg}-from run   \\
\glt `Marten and his associates run away from me.'
\z






So, the Kamang\il{Kamang} plural\is{plural (number) word} word occurs outside the NP and shares distributional properties of pronouns\is{pronoun}. The semantics of the plural\is{plural (number) word} word also intersects with pronouns\is{pronoun}, in particular, the quantificational pronouns\is{pronoun} whose functions are to denote different number features.

\subsection{Abui\ilt{Abui}} %3.4
\label{sec:9:3.4}
The template of the Abui\il{Abui} NP is presented in \REF{ex:9:46}.\footnote{The morphosyntactic analysis and glossing of Abui\il{Abui} presented here is that of Schapper, and differs from that presented in \citet{Kratochvil2007}. Examples are individually marked as to source.} The NP is composed of a head noun (N) followed by its attribute (\textsc{Attr).} The Abui\il{Abui} plural\is{plural (number) word} word \textit{loku} is not etymologically related to the plural\is{plural (number) word} word that is reconstructable for pAP\il{proto-Alor-Pantar}. It has a variable position with respect to the relative clause (\textsc{Rc}), being able to either precede or follow the plural\is{plural (number) word} word. The plural\is{plural (number) word} word occurs inside the NP and thus always occurs to left to the determiner (\textsc{Det).}

\ea%46
\label{ex:9:46}
\upshape 
 Template of the Abui\ilt{Abui} NP

[\textsc{N  Attr  \{Pl  Rc  /  Rc  Pl \} Det}]\textsubscript{\upshape NP}
\z

The variable position of \textit{loku} in relation to the relative clause is illustrated in \REF{ex:9:47} and \REF{ex:9:48}. In \REF{ex:9:47} \textit{loku} appears after the relative clause but before the demonstrative\is{demonstrative} \textit{yo}. In \REF{ex:9:48} \textit{loku} precedes both the relative clause and the article \textit{nu}. The two plural\is{plural (number) word} word positions are mere variants of one another; extensive elicitation and the examination of corpus data have revealed no difference in the scope or semantics correlating with the plural\is{plural (number) word} word's position, although corpus frequency and speaker judgments point to the position preceding the relative clause as being preferred.


\ea%47
\label{ex:9:47}
\langinfo{Abui}{}{Kratochv\'il, Abui corpus} \\
\gll  {\ob}...oto he-amakaang {\ob}{ba} h-omi {mia}{\cb}\textsubscript{\upshape RC} loku yo{\cb}\textsubscript{\upshape NP}  mi pak mahoi-ni \\
  car \textsc{3.gen}-person \textsc{rel} \textsc{3.gen}-inside in \textsc{pl} \textsc{dem}   take cliff gather-\textsc{pfv}  \\
\glt `...those people who were inside the car were taken over the [edge of the] cliff.'
\z
 

\ea%48
\label{ex:9:48}
\langinfo{Abui}{}{Kratochv\'il, Abui corpus} \\
\gll  {\ob}{Sieng} loku {\ob}{ba} uti {mia}{\cb}\textsubscript{\upshape RC} {nu}{\cb}\textsubscript{\upshape NP} sik bakon-i   mi melang sei. \\
   rice \textsc{pl} \textsc{rel} garden in \textsc{art} pluck rip.off.\textsc{pfv-pfv}   take village come.down \\
\glt `Pluck off [all] the rice that is in the garden [and] take it down to the village.'
\z

 




\textit{Loku} cannot co-occur in an NP together with any quantifiers; numeral (\ref{ex:9:49}a) or non-numeral (\ref{ex:9:49}b). However, it is possible for an NP with \textit{loku} to be the subject of both numeral and non-numeral quantifier predications (\ref{ex:9:50}a-b). This indicates that, whilst double marking of quantification/plurality\is{plural (number) word} is not permitted within the NP, there is no semantic redundancy in the quantificational values of the Abui\il{Abui} plural\is{plural (number) word} word and other quantifiers. In this respect, Abui\il{Abui} \textit{loku} differs from Teiwa\il{Teiwa} \textit{non} ({\S} \ref{sec:9:3.2}).


\ea%49
\label{ex:9:49}
\langinfo{Abui}{}{Schapper, fieldnotes} \\
\ea
\gll *He-wiil taama loku nu mon-i.\\
   \textsc{3.gen}-child six \textsc{pl} \textsc{art} die.\textsc{pfv}-\textsc{pfv} \\
\glt Intended: `His six children died.'
\ex
\gll *He-wiil faring loku nu mon-i.\\
 \textsc{3.gen}-child many \textsc{pl} \textsc{art} die.\textsc{pfv}-\textsc{pfv}   \\
\glt  Intended: `His many children died.'
\z
\z





\ea%50
\label{ex:9:50}
\langinfo{Abui}{}{Schapper, fieldnotes} \\
\ea
\gll  He-wiil loku nu {taama.}\\
  \textsc{3.gen}-child \textsc{pl} \textsc{art} six  \\
\glt `His children were six.' i.e., `He had six children.'
\ex
\gll He-wiil loku nu faring. \\
 \textsc{3.gen}-child \textsc{pl} \textsc{art} many   \\
\glt  `His children were many.' i.e., `He had many children.'
\z
\z






\textit{Loku} can be used to modify a third person pronoun\is{pronoun}, as in \REF{ex:9:51} and \REF{ex:9:52}. Abui\il{Abui} has no number distinction in the third person of its pronominal\is{pronoun} series. By using \textit{loku} the plural\is{plural (number) word} reference can be made explicit.


\ea%51
\label{ex:9:51}
\langinfo{Abui}{}{Kratochv\'il, Abui corpus} \\
\gll  Hel loku abui yaa ut {teak.}\\
  3 \textsc{pl}   mountain go garden watch  \\
\glt `They went to the mountains to check the garden.'
\z







\ea%52
\label{ex:9:52}
\langinfo{Abui}{}{Kratochv\'il, Abui corpus} \\
\gll  Hel loku he-sepatu {he-tawida}. \\
  3 \textsc{pl}   \textsc{3.gen}-shoe \textsc{3.gen}-be.alike   \\
\glt `They have the same shoes.'
\z






\textit{Loku} must co-occur with a noun or with the third person pronoun\is{pronoun} \textit{hel}. It cannot stand alone in an NP.

In addition to the general plural\is{plural (number) word} word \textit{loku}, Abui\il{Abui} has an associative plural\is{plural (number) word} word, \textit{we} `\textsc{assoc'}. This item only appears marking proper names for humans and has the meaning `[name] and people associated with [name]' and occurs directly after the noun it modifies, as in (\ref{ex:9:53}a). When \textit{loku} is used in the same context (\ref{ex:9:53}b), the reading is not one of associative plurality\is{plural (number) word}, but of individualized plurality\is{plural (number) word}. \textit{Loku} and \textit{we} can co-occur, and either can precede the other, as shown in (\ref{ex:9:53}c).




\ea%53
\label{ex:9:53}
\ea
\langinfo{Abui}{}{Schapper, fieldnotes} \\
\gll Benny w{e} ut yaa. \\
   Benny \textsc{assoc} garden go.to  \\
\glt `Benny and his associates go to the garden.'
\ex
\gll Benny loku ut yaa. \\
   Benny \textsc{pl} garden go.to  \\
\glt `Different individuals called Benny go to the garden.'
\ex
\gll Benny loku we / Benny  we loku ut yaa.\\
    Benny \textsc{pl}  \textsc{assoc} / Benny \textsc{assoc} \textsc{pl} garden go.to\\
\glt  `Two or more people called Benny go to the garden.'
\z
\z






Connected to its individualising semantics, \textit{loku} may be used with verbs to make expressions for collections of people. Examples are given in \REF{ex:9:54}.


\ea%54
\label{ex:9:54}
\ea
\langinfo{Abui}{}{\citealt[155]{Kratochvil2007}} \\
\gll pe loku  \\
   near \textsc{pl}  \\
\glt lit. `the near ones'; i.e. `neighbours'
\ex
\gll firai loku \\
  run \textsc{pl}   \\
 \glt lit. `the running ones'; i.e. `runners'
\glt
\ex
\gll walangra loku \\
   fresh \textsc{pl}  \\
\glt  lit. `the new ones'; i.e. `the newcomers, the Malays'
\z
\z






Abui\il{Abui} differs from the more western languages (such as Western Pantar\il{Western Pantar} and Teiwa\il{Teiwa}) in that it has two plural\is{plural (number) word} words marking different kinds of plurality\is{plural (number) word}.

\subsection{Wersing\ilt{Wersing}} %3.5
\label{sec:9:3.5}
The template for the Wersing\il{Wersing} noun phrase (NP) is given in \REF{ex:9:55}. Modifiers follow the head noun of the NP (\textsc{N}\textsubscript{\upshape \MakeUppercase{head}}). They are an attribute (\textsc{Attr}), a numeral (\textsc{Num}) or the plural\is{plural (number) word} word\textsc{(Pl),} and a relative clause (\textsc{Rc).} \textsc{R}ight-most in the NP is a determiner (\textsc{Det}). See \citet{SchapperEtAltawersing} for details and full illustration of the Wersing\il{Wersing} NP.

\ea%55
\label{ex:9:55}
\upshape 
 Template of the Wersing\ilt{Wersing} NP 

  [\textsc{N}\textsubscript{\upshape \MakeUppercase{head } }\textsc{Attr Num/Pl  Rc  Det]}\textsc{\textsubscript{\upshape NP}}
\z


The Wersing\il{Wersing} plural\is{plural (number) word} word is \textit{deing}. As is clear from template \REF{ex:9:55}, it occurs in the NP in the same slot as a numeral. It cannot be used in combination with a numeral or any non-numeral quantifier (which are typically simple intransitive verbs that appear in the \textsc{Attr} slot), as illustrated in \REF{ex:9:56}.


\ea%56
\label{ex:9:56}
\langinfo{Wersing}{}{Schapper and Hendery, Wersing corpus} \\
\ea
\gll *aning    weting deing \\
    person five \textsc{pl} \\
 \glt Intended: `five people'
\ex
\gll *aning bal {deing}\\
   person many   \textsc{pl} \\
\glt  Intended: `many people'
\z
\z






\textit{Deing\il{Deing}} need not occur with an overt noun in the NP, but can stand alone so long as the referent can be retrieved from the discourse context. So, for instance, the head noun \textit{gis} in (\ref{ex:9:57}a) can be elided, as in the following examples (\ref{ex:9:57}b-d). What is more, the NP can be reduced to the plural\is{plural (number) word} word (57d) where there is neither noun head nor article.


\ea%57
\label{ex:9:57}
\langinfo{Wersing}{}{Schapper and Hendery, Wersing corpus} \\
\ea
\gll {g}{-is}    kebai dein=a\\
3-content young \textsc{pl=art}  \\
\glt `their (coconut) young flesh'
\ex
\gll kebai dein=a   \\
   young \textsc{pl=art}  \\
\glt `the young (flesh)' 
\ex
\gll dein=a   \\
   \textsc{pl=art}  \\
\glt  `the (young flesh)'
\ex
\gll {deing}\\
    \textsc{pl}\\
\glt  `the (young flesh)'
\z
\z




Like Kamang\il{Kamang} and the other eastern Alor languages, and Teiwa\il{Teiwa} on Pantar, Wersing\il{Wersing} has multiple pronominal\is{pronoun} paradigms dedicated to denoting particular quantities of referents, for instance, universal quantification (`\textsc{all')} \REF{ex:9:58} and group plurality\is{plural (number) word} (`\textsc{group}') \REF{ex:9:59}.\footnote{There are five ``quantifying'' pronominal\is{pronoun} paradigms in Wersing\il{Wersing}: (i) the ``alone'' paradigm (\textit{I alone (no one else)}, etc.), (ii) the ``independent'' paradigm (\textit{I on my own without help}, etc.), (iii) the dual paradigm (\textit{we} \textit{two}, etc.- only in non-singular numbers), (iv) the ``all'' paradigm (\textit{we} \textit{all}, etc.- only in non-singular numbers), and (v) the ``group'' paradigm (\textit{we together in a group}, etc.- only in non-singular numbers) \citep{SchapperEtAltawersing}.} Such quantificational pronouns\is{pronoun} also play an important role in marking plurality\is{plural (number) word} of NP referents in Wersing\il{Wersing}. In \REF{ex:9:60} we see, for instance, the 3\textsuperscript{rd} person pronoun\is{pronoun} \textit{genaing} being
used to signal the plurality\is{plural (number) word} of the referents of the preceding NP.\footnote{A pronoun\is{pronoun} of this paradigm can also be marked with \textit{-le}, as in: \textit{Aning ge-naingle kamar ming=te nanal te-mekeng} (\textsc{3.all-pl} room be.in=\textsc{conj} thing \textsc{recp}-exchange) `All of those who are in the room exchange things'. The -\textit{le} suffix does not appear on nouns or any other pronominal\is{pronoun} series in Wersing\il{Wersing}; it is likely cognate with the Kamang\il{Kamang} associative plural\is{plural (number) word} marker \textit{-lee} (see {\S} \ref{sec:9:3.3}).}


\ea%58
\label{ex:9:58}
\langinfo{Wersing}{}{\citealt[463]{SchapperEtAltawersing}} \\
\gll  Tanaing dra bo! \\
   \textsc{1pl.incl.all} sing   \textsc{emph}  \\
\glt `Let's sing.'
\z







\ea%59
\label{ex:9:59}
\langinfo{Wersing}{}{\citealt[479]{SchapperEtAltawersing}} \\
\gll  Nyawi nyi-mit o! \\
   \textsc{1pl.excl.}\textsc{group} \textsc{1pl.excl}-sit \textsc{exclam}  \\
\glt `Let's sit together!'
\z







\ea%60
\label{ex:9:60}
\langinfo{Wersing}{}{\citealt[495]{SchapperEtAltawersing}} \\
\gll  Ge-siriping genaing beteng ge-dai. \\
 \textsc{3-}root \textsc{3.all}   pull 3-come.up    \\
\glt `All its roots were pulled right up.'
\z





Wersing\il{Wersing} \textit{d}\textit{eing} can nevertheless mark plurality\is{plural (number) word} for non-singular numbers of topic pronouns\is{pronoun}, as in \REF{ex:9:61} and \REF{ex:9:62}. In this respect, then, the Wersing\il{Wersing} plural\is{plural (number) word} word is not like a pronoun\is{pronoun} as in Kamang\il{Kamang}, but a distinct item which can modify any NP head, nominal or pronominal\is{pronoun}.


\ea%61
\label{ex:9:61}
\langinfo{Wersing}{}{Schapper  and Hendery, Wersing corpus} \\
\gll  Gai dein=a mona min-a. \\
   \textsc{3.top} \textsc{pl}=\textsc{art} \textsc{across} be.at-\textsc{real}  \\
\glt `They are all over there.'
\z







\ea%62
\label{ex:9:62}
\langinfo{Wersing}{}{Schapper  and Hendery, Wersing corpus} \\
\gll  Nyai deing o-min-a. \\
  \textsc{1pl.excl.top} \textsc{pl}   \textsc{here}-be.at-\textsc{real}   \\
\glt `We are all here.'
\z






Wersing\il{Wersing} has a further plural\is{plural (number) word} word, \textit{naing}, which marks associative plurality\is{plural (number) word}. This form has been observed only marking personal names, as in \REF{ex:9:63} and \REF{ex:9:64}. As an associative plural\is{plural (number) word} word, it doesn't have the ability to stand in for a NP. Like \textit{deing}, \textit{naing} cannot occur with other quantifiers, numeral and non-numeral.


\ea%63
\label{ex:9:63}
\langinfo{Wersing}{}{\citealt{Malikosand}} \\
\gll  Petrus naing g-aumeng ga-pang ge-pai. \\
  Peter \textsc{assoc} 3-fear 3-dead 3-make   \\
\glt `Peter and the others were afraid to die.'
\z







\ea%64
\label{ex:9:64}
\langinfo{Wersing}{}{\citealt{Malikosand}} \\
\gll  Yesus naing lailol gewai Kapernaum {taing.}\\
  Jesus \textsc{assoc} walk 3-go Kapernaum reach  \\
\glt `Jesus and the others walked onto Kapernaum.'
\z






\subsection{Summary}  %3.6
\label{sec:9:3.6}




Most Alor-Pantar languages have inherited a plural\is{plural (number) word} word, but they show much variation in the syntactic properties of this word. Table \ref{tab:9:2} presents a summary of the variable syntax discussed in the previous sections. 
\begin{table}[h]

\begin{tabularx}{\textwidth}{>{\small}Qccccc}
\lsptoprule
 & {Teiwa\ilt{Teiwa}}  &{W Pantar\il{Western Pantar}} &{Kamang\ilt{Kamang}}  &{Abui\ilt{Abui}}  &{Wersing\ilt{Wersing}}\\
\midrule
Is plural\ist{plural (number) word} word part of NP? &yes &yes &no &yes &yes\\
Can plural\ist{plural (number) word} word stand alone in NP? &no &no &-- &no &yes\\
Can the plural\ist{plural (number) word} word and non-numeral quantifier co-occur? &yes &no &no &no &no\\
\parbox{3.5cm}{Can plural\ist{plural (number) word} word and\\\mbox{numeral} co-occur?} &no &no &no &no &no\\
\lspbottomrule
\end{tabularx}
\caption{Variable syntax of five Alor-Pantar plural\ist{plural (number) word} words}
\label{tab:9:2}
\end{table}

The table reveals the gradient differences between plural\is{plural (number) word} words in Alor-Pantar languages. Kamang\il{Kamang} stands out from the other four languages for the fact that the plural\is{plural (number) word} word is not part of the NP. Of the languages that do have their plural\is{plural (number) word} word in the NP, the plural\is{plural (number) word} word cannot typically stand alone in the NP, but requires another, nominal, element be present. In Wersing\il{Wersing}, however, this is only the case for the associative plural\is{plural (number) word} word \textit{deing}; its plural\is{plural (number) word} number word can form independent NPs. Alor-Pantar plural\is{plural (number) word} words are prohibited from co-occurring with quantifiers. No language allows co-occurrence with a numeral quantifier and only Teiwa\il{Teiwa} permits co-occurrence with a non-numeral quantifier.

 These different properties mean that in all five languages, plural\is{plural (number) word} word(s) constitutes a word class of its own, with only partial overlap with other morpho-syntatic classes of words. In Western Pantar\il{Western Pantar}, the plural\is{plural (number) word} word shares much with adjectival quantifiers and numerical expressions. In Teiwa\il{Teiwa}, the plural\is{plural (number) word} word patterns mostly with non-numeral quantifiers. In Kamang\il{Kamang} and Wersing\il{Wersing}, plural\is{plural (number) word} words pattern similarly to quantificational pronouns\is{pronoun} in denoting the number of a preceding noun. However, Wersing\il{Wersing} \textit{deing} behaves much more like a nominal element. Nominal properties are also visible in the Abui\il{Abui} word \textit{loku}, particularly in its frequent use with verbs to form expressions for collections of people.

 In short, Alor-Pantar plural\is{plural (number) word} words are a morpho-syntactically diverse group of items that are seemingly united only by their semantic commonalities. Yet, as we will see in the following sections, even the semantics of plurality\is{plural (number) word} reveal more variability than might have been expected.

\section{Semantics of plural\ist{plural (number) word} words in Alor-Pantar}  %4
\label{sec:9:4}
In all five languages, the plural\is{plural (number) word} words code plurality\is{plural (number) word} alongside other notions. In this section, we review three additional connotations of the plural\is{plural (number) word} word.

\subsection{Completeness} %4.1
\label{sec:9:4.1}
The Western Pantar\il{Western Pantar} plural\is{plural (number) word} word \textit{maru(ng)} typically imparts a sense of entirety, completeness, and comprehensiveness, as in \REF{ex:9:65}:


\ea%65
\label{ex:9:65}
Western Pantar\ilt{Western Pantar} (Holton, Western Pantar\ilt{Western Pantar} corpus)\\
\gll  Ping pi mappu maiyang, lokke maiyang saiga si, gai ke{{\textglotstop}}{e} maru si aname ging haggi kanna. \\
    \textsc{1pl.incl}  \textsc{1pl.incl.poss\ist{possession}} fishpond place fishtrap place \textsc{dem} \textsc{art}  \textsc{3.poss\ist{possession}} fish \textsc{pl} \textsc{art} person \textsc{3pl.act} take already \\
\glt `We placed our fishponds, placed our fish traps,  and then people took all the fish.'
\z


Its sense of comprehensiveness and entirety explains why NPs pluralized\is{plural (number) word} with \textit{maru(ng)} can be the subject of the nominal predicate \textit{gaterannang} `all' expressing universal quantity, as in \REF{ex:9:66}, while combinations of \textit{marung} and mid-range quantifiers such as \textit{haweri} `many' are absent in the Western Pantar\il{Western Pantar} corpus. It also explains why \textit{marung} is not compatible with a numeral predicate, as in \REF{ex:9:67}, as these indicate a quantity of a certain number rather than universal quantity.


\ea%66
\label{ex:9:66}
\langinfo{Western Pantar}{}{\citealt{Holton2012}} \\
\gll  {\ob}{Aname } {marung}{\cb} ging gaterannang dia wang pidding. \\
   people \textsc{pl} they all  go exist spread  \\
\glt `All the people spread out [to look for them]' \citep{Holton2012}
(Lit. `All people they were all going spreading ...')
\z








\ea%67
\label{ex:9:67}
\langinfo{Western Pantar}{}{\citealt{Holton2012}} \\
\gll  *{\ob}{Aname} {marung}{\cb} ging kealaku dia wang pidding. \\
    people \textsc{pl} they twenty go exist spread \\
 \glt Intended: `All people they were twenty going spreading...'
\z






Finally, \textit{marung} is used with count nouns, and cannot combine with mass nouns such as \textit{halia} `water', \REF{ex:9:68}. In this respect, \textit{marung} contrasts with the plural\is{plural (number) word} words in Abui\il{Abui}, Wersing\il{Wersing}, Kamang\il{Kamang} and Teiwa\il{Teiwa}, which can combine with mass nouns (sections 4.2,  4.3.1).


\ea%68
\label{ex:9:68}
\langinfo{Western Pantar}{}{Holton, p.c.} \\
\gll  *halia marung \\
   water \textsc{pl}  \\
\glt Intended: `several containers of water';  `multiple waters'
\z






The connotation of comprehensiveness is also found in Abui\il{Abui} \textit{loku}. That is, the inclusion of \textit{loku} signals that the whole mass of saliva was subject to the swarming of the birds in \REF{ex:9:69} and that all the available corn had to be stowed away \REF{ex:9:70} in an orderly fashion, so as to use the maximum capacity of the basket.


\ea%69
\label{ex:9:69}
\langinfo{Abui}{}{Kratochv\'il, Abui corpus} \\
\gll  ... {kuya} do sila nahang oro  he-ya he-puyung loku do he-afai. \\
   ...  bird \textsc{dem} much everywhere \textsc{level}   \textsc{3.gen}-mother \textsc{3.gen-}saliva \textsc{pl} \textsc{dem} \textsc{3.gen-}swarm \\
\glt `Those birds were everywhere there, swarming over the saliva of his mother.'
\z













\ea%70
\label{ex:9:70}
\langinfo{Abui}{}{Kratochv\'il, fieldnotes} \\
\gll  Fat loku mi ba buot he-rei \\
    corn \textsc{pl} take \textsc{conj} back.basket \textsc{3.gen-}stow \\
\glt `Stow all the corn in the basket.'
\z






The sense of comprehensive quantity expressed by \textit{loku} (`all') is relative to the situation at hand (`all that is there'). As a result, \textit{loku} can occur with the universal quantifier \textit{tafuda} `all', as in \REF{ex:9:71}.



\ea%71
\label{ex:9:71}
\langinfo{Abui}{}{Kratochv\'il, Abui corpus} \\
\gll  Ama {\ob}ne-mea loku{\cb} tafuda takaf-i  do n-omi he-ukda \\
    person \textsc{1sg.gen-}mango \textsc{pl} all steal-\textsc{pfv}  \textsc{dem} \textsc{1sg.gen-}inside 3.\textsc{gen-}shock \\
\glt `All my mangos got stolen,  it really shocked me.' %(Kratochv\'il, Abui corpus)
\z












In Wersing\il{Wersing}, the sense of comprehensiveness is found when the plural\is{plural (number) word} word is used together with an already plural\is{plural (number) word} topic pronoun\is{pronoun}. For instance, in \REF{ex:9:72} the use of \textit{deing} implies that the whole set of those who were expected are present.


\ea%72
\label{ex:9:72}
\langinfo{Wersing}{}{Schapper, fieldnotes} \\
\gll Tai deing o-min-a.  \\
   \textsc{1pl.incl.top} \textsc{pl}   \textsc{here}-be.at-\textsc{real}  \\
\glt `We are all here.'
\z






\subsection{Abundance} %4.2
\label{sec:9:4.2}
In Teiwa\il{Teiwa} and Wersing\il{Wersing}, using the plural\is{plural (number) word} word can add the sense that the referent occurs in particular abundance.

While the core semantics of Teiwa\il{Teiwa} \textit{non} is plural\is{plural (number) word} `more than one' or `several', it often has the connotation of `many, plenty', as in \REF{ex:9:73}. This is not true for all plural\is{plural (number) word} words in AP languages.


\ea%bkm:Ref335061853
\label{ex:9:73}
\langinfo{Teiwa}{}{Klamer, Teiwa corpus} \\
\ea
\gll  in  non\\
   it.thing  \textsc{pl} \\
   \glt `plenty of things' 
\ex
\gll in  bun non \\
it.thing bamboo \textsc{pl} \\
\glt `plenty of pieces of bamboo'
\ex 
\gll wou  non\\
\textsc{pl} mango\\
\glt  `plenty of mangos'
\z
\z

Especially when combining with nouns referring to utensils or consumables, the plurality\is{plural (number) word} of \textit{non} often has the connotation `plenty'. A similar reading is imposed when \textit{non} combines with small objects such as flowers or insects. As these come in sets of conventionally large numbers, the use of \textit{non} implies that their set is larger than expected. For instance, \textit{haliwai} \textit{non} in \REF{ex:9:74} refers to black ants as crawling into the sarong in unexpected numbers.

\ea%74
\label{ex:9:74}\langinfo{Teiwa\ilt{Teiwa}}{}{Klamer, Teiwa\ilt{Teiwa} corpus}\\
\gll ...a mis-an haliwai non daa nuan gom ma yiri  u si,... \\
  \textsc{3sg} sit-\textsc{real} black.ant \textsc{pl} ascend cloth inside come crawl \textsc{dist} \textsc{sim}  \\
\glt `...(while) he sat (unexpectedly many) black ants came crawling into his sarong,...'
\z






There are other specific readings that \textit{non} may get, varying according to the type of nominal referent and the pragmatics of the situation. For example, when \textit{non} combines with objects such as seeds, chairs, or rocks, it may imply that they occur in a set that has an unusual configuration which is more disorderly than the conventional one, such as when seeds are spilled across the floor rather than in a bag or a pile, or when chairs are scattered around the room instead of organized around a table. Finally, \textit{non} may also code that the set is non-homogeneous, e.g., \textit{war} \textit{non} may refer to `several rocks', but also to `rocks of various kinds and sizes'.

Wersing\il{Wersing} also reflects this sense, when referring to inanimates\is{animacy}, especially where they have little individuation. In \REF{ex:9:75} the use of \textit{deing} to modify \textit{wor} `rock' and \textit{inipak} `sand' suggests that an abundance of these items are swept up by the wind. Without the plural\is{plural (number) word} word, there would be no indication of the amount of rock and sand moved by the wind.


\ea%75
\label{ex:9:75}
\langinfo{Wersing}{}{Schapper, fieldnotes} \\
\gll  Tumur lapong gai ge-tati=sa, wor anta inipak lang=mi dein=a ge-poing ge-dai medi aruku le-ge-ti.   \\
  east.wind wind 3.\textsc{a} 3-stand=\textsc{conj} rock or sand beach=\textsc{loc}   \textsc{pl=art} \textsc{3-}hit 3-go.up take dry.land \textsc{appl-3-}lie \\
\glt `When the east wind blows, a mass of rocks and sand from the beach is lifted up and deposited on dry land (beyond the beach).'
\z










Such senses of abundance have not been observed with the plural\is{plural (number) word} word in Western Pantar\il{Western Pantar} or Kamang\il{Kamang}.

\subsection{Individuation} %4.3
\label{sec:9:4.3}
The use of a plural\is{plural (number) word} word often imposes an individuated reading of a referent, that is, that the referent is not an undifferentiated mass but rather is composed of an internally cohesive set of individuals of the same type. For instance, consider the contrast between the \textit{we} and the \textit{loku} plural\is{plural (number) word} in Abui\il{Abui} in (\ref{ex:9:76}a-b), repeated from example \REF{ex:9:53} in {\S} \ref{sec:9:3.4}. The associative plural\is{plural (number) word} \textit{we} gives a reading of a closely-knit group of individuals centred on one prominent individual, Benny. By contrast, the \textit{loku} plural\is{plural (number) word}, when it is used in the same context, imposes a referentially heterogeneous or individualized reading whereby multiple distinct people of the same name are being referred to. This difference is also characteristic of the Wersing\il{Wersing} plural\is{plural (number) word} words \textit{deing} `\textsc{pl}' and \textit{naing} `\textsc{assoc}'.


\ea%76
\label{ex:9:76}
\langinfo{Abui}{}{Schapper, fieldnotes} \\
\ea
\gll Benny we ut yaa. \\
   Benny \textsc{assoc} garden go.to \\
\glt `Benny and his associates go to the garden.'
\ex
\gll Benny loku ut yaa.\\
  Benny \textsc{pl} garden go.to  \\
\glt  `Different individuals called Benny go to the garden.'
\z
\z





There are two contexts in which we find a particular tendency of plural\is{plural (number) word} words in AP to impose individualized readings on the nouns they modify. These are discussed in the following subsections.

\subsection{Individuation of mass to count}
While they are typically used with count nouns, plural\is{plural (number) word} words may combine with mass nouns, provided these are recategorized. Combining a plural\is{plural (number) word} word with a mass noun indicates that it is interpreted as a count noun. For instance, Teiwa\il{Teiwa} \textit{yir} `water' is interpreted as a mass in (\ref{ex:9:77}a), but gets an individuated reading in (\ref{ex:9:77}b) when it combines with \textit{non}. In Kamang\il{Kamang} (\ref{ex:9:78}a) the noun \textit{ili} `water' combined with \textit{nung} is individuated just like when it combines with the numeral \textit{nok} `one' (\ref{ex:9:78}b).\footnote{As we saw in {\S} \ref{sec:9:4.1}, Western Pantar\il{Western Pantar} \textit{maru(ng)} does not have this individuating function due to the sense of comprehensiveness and completeness of the word.}

\newpage
\ea%77
\label{ex:9:77}
\langinfo{Teiwa}{}{Klamer, Teiwa corpus} \\
\ea
\gll  Na yir ma gelas {mia}{{\textglotstop}}{.} \\
    \textsc{1sg} water \textsc{obl} glass fill \\
\glt `I fill the glass with water.'
\ex
\gll Na yir non ma drom {mia}{{\textglotstop}}{.} \\
   \textsc{1sg} water \textsc{pl} \textsc{obl} drum fill  \\
\glt  `I fill the drum with several containers of water.'
\z
\z







\ea%78
\label{ex:9:78}
\langinfo{Kamang}{}{\citealt{SchapperEtAl2011plural}} \\
\ea
\gll\textit{ili} \textit{nung} \\
   water \textsc{pl} \\
 \glt `\{multiple individual\} waters' 
 \ex 
 \gll \textit{ili} \textit{nok}\\
  water one\\
\glt `a water'
\z
\z

The plural\is{plural (number) word} words in Abui\il{Abui} and Wersing\il{Wersing} also occur together with mass nouns with readings of abundance, as discussed already in {\S} \ref{sec:9:4.2}. Western Pantar\il{Western Pantar} \textit{marung} cannot combine with mass nouns.

\subsection{Clan or place name to members}
When Abui\il{Abui} \textit{loku} is combined with the name of a clan or a place name, the expression refers to the members belonging to that clan \REF{ex:9:79} or issuing from that place \REF{ex:9:80}, a use that can be extended to the question word \textit{te} `where' \REF{ex:9:81}.


\ea%79
\label{ex:9:79}
\langinfo{Abui}{}{\citealt[165]{Kratochvil2007}} \\
\gll  {Afui Ata} loku \\
   clan.name \textsc{pl}  \\
\glt `people of the Afui Ata clan'
\z







\ea%80
\label{ex:9:80}
\langinfo{Abui}{}{\citealt[166]{Kratochvil2007}} \\
\gll  Kafola  loku \\
    Kabola  \textsc{pl} \\
\glt `people from Kabola'
\z







\ea%81
\label{ex:9:81}
\langinfo{Abui}{}{Kratochvil, fieldnotes} \\
\gll  Edo te loku, {naana?}\\
   \textsc{2sg.foc} where \textsc{pl} older.sibling \\
\glt `Where are you from, bro?'
\z






A similar use is attested for Teiwa\il{Teiwa} \textit{non} when it is used to make an ethnonym from a clan name \REF{ex:9:82}. However, Teiwa\il{Teiwa} \textit{non} cannot combine with place names.


\ea%82
\label{ex:9:82}
\langinfo{Teiwa}{}{Klamer, Teiwa corpus} \\
\gll  Teiwa non ga{{\textglotstop}}{an} {ita}{{\textglotstop}}{a} ma gi? \\
   clan.name \textsc{pl} that.\textsc{knwn} where \textsc{obl} go  \\
\glt `Where did that group of Teiwa [people] go to?'
\z






This function of the plural\is{plural (number) word} word is not known to occur in Western Pantar\il{Western Pantar}, Kamang\il{Kamang} or Wersing\il{Wersing}. In Kamang\il{Kamang} this kind of plurality\is{plural (number) word} is encoded by the combination of a place name with a group plural\is{plural (number) word} pronoun\is{pronoun}, as in \REF{ex:9:83}.


\ea%83
\label{ex:9:83}
\langinfo{Kamang}{}{Schapper, fieldnotes} \\
\gll  Ga wo-suk-si=bo gafaa Takailubui geifu {mauu-h=a},...\\
   3.\textsc{agt} \textsc{3.loc-}think\textsc{-ipfv=lnk} 3.\textsc{alone} Takailubui 3.\textsc{grp} war-\textsc{purp}=\textsc{spec} \\
\glt `They think that if they alone make war against the people of Takailubui,...'
\z






\subsection{Partitive} %4.4
\label{sec:9:4.4}
Plural\is{plural (number) word} words also occur in contexts of partitive plural\is{plural (number) word} reference. This means that the plural\is{plural (number) word} can be used to pick out a part or group of referents from a larger set.

The Kamang\il{Kamang} plural\is{plural (number) word} word \textit{nung} can be used for partitive plural\is{plural (number) word} reference, often with contrast between different subsets of referents. For instance, in \REF{ex:9:84} \textit{nung} is used twice to divide the set of citruses into the multitude that are sweet and the multitude that are sour. Similarly, in \REF{ex:9:86} \textit{nung} is used twice to contrast the sub-set of people who went to Molpui with the sub-set that went to the nearby village.


\ea%84
\label{ex:9:84}
\langinfo{Kamang}{}{\citealt[40]{Stokhof1982}} \\
\gll  Muut=ak nung iduka, ah=a nung alesei. \\
   citrus=\textsc{pl} \textsc{pl} sweet \textsc{cnct=spec} \textsc{pl} sour  \\
\glt `Some of these citrus fruits, others are sour.'
\z







\ea%85
\label{ex:9:85}
\langinfo{Kamang}{}{\citealt[57]{Stokhof1978}} \\
\gll  Nung gera ye-iyaa ai Molpui wo-oi ye-te,  nung gera yeeisol ye-iyaa ai  mane wo-oi ye-wete. \\
    \textsc{pl} \textsc{3.contr} \textsc{3.gen}-return take M. 3.\textsc{loc}-to \textsc{3.gen}-go.up  \textsc{pl} \textsc{3.contr} straight \textsc{3.gen}-return take  village 3.\textsc{loc}-towards \textsc{3.gen}-go.up.across \\
\glt `Some of them went home going up to Molpui, others went straight home going up across to the village.'
\z




The Wersing\il{Wersing} plural\is{plural (number) word} word can be used also in partitive plural\is{plural (number) word} reference, but does not typically make explicit contrasts between subsets using the plural\is{plural (number) word} word over multiple NPs. For instance, in \REF{ex:9:87} \textit{deing} refers to a subset of candle nuts that have not yet been crushed, the other set is not explicitly mentioned but must simply be inferred from the discourse context. In \REF{ex:9:88}, the other member of the whole (namely the speaker himself) is singular and so is not marked with the plural\is{plural (number) word} word, but he is contrasted with the set of others who are teaching other languages. This second plural\is{plural (number) word} set is accordingly marked with \textit{deing}.


\ea%86
\label{ex:9:86}
\langinfo{Wersing}{}{Schapper, fieldnotes} \\
\gll Deing de naung. \\
 \textsc{pl} \textsc{ipfv} \textsc{neg}   \\
\glt `Some are still not done.'
\z







\ea%87
\label{ex:9:87}
\langinfo{Wersing}{}{Schapper, fieldnotes} \\
\gll  Naida Abui\ilt{Abui} ge-lomu ong ge-tenara,  pang=sa te-nong aumang dein=a  Pantara ge-lomu ong ge-tenara war Sawila ge-lomu.\\
  \textsc{1pl.excl.top} Abui\ilt{Abui} 3-language use 3-teach  \textsc{dem=conj} \textsc{1pl.incl}-friend other \textsc{pl=art}   Pantar 3-language use 3-teach and Sawila\ilt{Sawila} 3-language\\
\glt `I will teach them Abui\ilt{Abui} and other friends of ours  will teach them Pantar and Sawila\ilt{Sawila}.'
\z

Such a contrastive use of the plural\is{plural (number) word} word has not been attested in Western Pantar\il{Western Pantar} and Abui\il{Abui}, but may be a sense present in Teiwa\il{Teiwa} \textit{non}, see \REF{ex:9:22} and \REF{ex:9:23} ({\S} \ref{sec:9:3.2}).

\subsection{Vocative} %4.5
\label{sec:9:4.5}
A term of address, relation or kin can be also marked with a plural\is{plural (number) word} to express a plural\is{plural (number) word} vocative. Western Pantar\il{Western Pantar} \textit{marung} has a vocative use in \REF{ex:9:89}. Teiwa\il{Teiwa} \textit{non} can be used in vocatives with kin\is{kinship} terms, for instance, when starting a speech \REF{ex:9:90} or in a hortative \REF{ex:9:91}.


\ea%88
\label{ex:9:88}
\langinfo{Western Pantar}{}{Holton, Western Pantar corpus} \\
\gll  Wenang  marung hing yadda mising, nang {na-ti}{{\textglotstop}}{ang.} \\
   Mr  \textsc{pl} \textsc{pl} \textsc{not.yet} sit I \textsc{1sg}-sleep  \\
\glt `You all keep sitting, I'm going to sleep.'
\z







\ea%89
\label{ex:9:89}
\langinfo{Teiwa}{}{Klamer, Teiwa corpus} \\
\gll Na-{rat qai} non oh!  \\
  1\textsc{sg}.\textsc{poss\ist{possession}}-grandchild \textsc{pl} \textsc{excl}   \\
\glt `Oh my grandchildren!'
\z







\ea%90
\label{ex:9:90}
\langinfo{Teiwa}{}{Klamer, Teiwa corpus} \\
\gll  Na-{gas qai} non, tup pi gi ina. \\
   \textsc{1sg-}female.younger.sibling \textsc{pl} get.up \textsc{1pl.incl} go eat  \\
\glt `My (female) friends, let's get up to eat.'
\z






Abui\il{Abui} \textit{loku} also can be present in vocative contexts with relational nouns \REF{ex:9:91} or kin terms \REF{ex:9:92}.


\ea%91
\label{ex:9:91}
\langinfo{Abui}{}{Schapper, fieldnotes} \\
\gll  Ne-feela loku, yaa fat ho-aneek.\\
   \textsc{1sg.gen-}friend \textsc{pl} go corn \textsc{3.loc}-weed \\
\glt `My friends, go weed the corn.'
\z







\ea%92
\label{ex:9:92}
\langinfo{Abui}{}{Schapper, fieldnotes} \\
\gll  Ne-fing loku, me! \\
    \textsc{1sg.gen-}elder.sibling \textsc{pl} come \\
\glt `My siblings, come on already.'
\z






There is no reason to expect that plural\is{plural (number) word} words should not be usable in vocatives. Yet, the plural\is{plural (number) word} word is not found in Kamang\il{Kamang} or Wersing\il{Wersing} vocatives. In Kamang\il{Kamang}, there are a range of special vocatives for calling (a) child(ren) or (b) friend(s). A Kamang\il{Kamang} vocative suffix, when used, means that a noun cannot be further modified, for instance, with the plural\is{plural (number) word} word.

\subsection{Summary}  %4.6
\label{sec:9:4.6}
Plural\is{plural (number) word} words code more than plurality\is{plural (number) word}; they have additional connotations and usages which vary across the languages as summarized in Table \ref{tab:9:3}.

\begin{table}[h]
 
\begin{tabularx}{\textwidth}{l@{}ccccc}
\lsptoprule
 & {Teiwa\ilt{Teiwa}}  & {Abui\ilt{Abui}}  & {Wersing\ilt{Wersing}} & {W Pantar\ilt{Western Pantar}} & {Kamang\ilt{Kamang}} \\
\midrule
Completeness  &no &yes &yes \dag &yes &no\\
Vocative &yes &yes &no &yes &no\\
Individuation: mass>count &yes &yes &no &no &yes\\
Individuation: name>members  &yes &yes &no &no &no\\
Abundance &yes &no &yes \ddag &no &no\\
Partitive &no &no &yes &no &yes\\

\lspbottomrule
\end{tabularx}

{\dag} On topic pronouns\is{pronoun} only. \ddag With inanimates\is{animacy} only.
\caption{Semantics of Alor-Pantar plural\ist{plural (number) word} words}
\label{tab:9:3}
\end{table}

\section{Typological perspectives on plural\ist{plural (number) word} words in AP languages} %5
\label{sec:9:5}
We saw in {\S} \ref{sec:9:1} that a good deal of what was known of the typology of plural\is{plural (number) word} words is due to Matthew Dryer's work, in particular \citet{Dryer1989,Dryer2011} and to a lesser extent \citet{Dryer2007}. \citet{Dryer2011} documents the use of plural\is{plural (number) word} words in the coding of nominal plurality\is{plural (number) word}. In doing so, Dryer wanted to prove the existence of a phenomenon that was not generally recognized, and his definitions reflect that. As mentioned in {\S} \ref{sec:9:1}, for Dryer, to be a plural\is{plural (number) word} word an item must be the prime indicator of plurality\is{plural (number) word}, and in the pure case they have this as their unique function. Based on this constrained characterization, Dryer shows that plural\is{plural (number) word} words nevertheless show considerable diversity.

 First, while being by definition non-affixal, they vary according to their degree of phonological independence. Second, they show great variety in the word class to which they belong; they may be integrated (to a greater or lesser degree) into another class, or form a unique class. The examples from the Alor-Pantar languages show vividly the variety of plural\is{plural (number) word} words in this regard: in all of them, plural\is{plural (number) word} words form a unique class on their own, which is however integrated into another class - but which class is variable across the languages. For instance, in Teiwa\il{Teiwa}, the plural\is{plural (number) word} word is part of the noun phrase and behaves largely like a nominal quantifier, while in Kamang\il{Kamang}, rather than actually being part of the noun phrase, the plural\is{plural (number) word} word distributes as a noun phrase itself.

 Third, plural\is{plural (number) word} words may have different values. In this respect they are perhaps poorly named. Dryer (1989: 869) suggests that ``grammatical number words'' would be a better term, since he gives instances of singular words and dual words. This is an area where Alor-Pantar languages indicate how the typology can be taken forward. When we look at the full range of ``ordinary'' number values, those associated with affixal morphology, we distinguish `determinate' and `indeterminate' number values \citep[39-41]{Corbett2000}. Determinate number values are those where only one form is appropriate, given the speaker's knowledge of the real world. If a language has an obligatory dual, for instance, this would be a determinate number value since to refer to two distinct entities this would be the required choice. However, values such as paucal or greater plural\is{plural (number) word} are not like this; there is an additional element to the choice. We find this same distinction in the Alor-Pantar number words: for instance, Teiwa\il{Teiwa} \textit{non
}signals not just plurality\is{plural (number) word} but has the connotation of abundance (like the greater plural\is{plural (number) word}).

 Fourth, a key part of the typology of number systems is the items to which the values can apply. Two systems may be alike in their values (say both have singular and plural\is{plural (number) word}) but may differ dramatically in that in one language almost all nominals have singular and plural\is{plural (number) word} available, while in the other plurality\is{plural (number) word} may be restricted to a small (top) segment of the Animacy\is{animacy} Hierarchy. The data from Alor-Pantar languages are important in showing how this type of differentiation applies also with number words. With affixal number, we find instances of recategorization; these are found particularly where a mass noun is recategorized as a count noun, and then has singular and plural\is{plural (number) word} available. We see this equally in Alor-Pantar languages such as Kamang\il{Kamang} where \textit{nung} is used with \textit{ili} `water', when recategorized as a count noun.

 Furthermore, number words are not restricted to appearing with nouns. In Abui\il{Abui}, plural\is{plural (number) word} \textit{loku} can occur with a third person pronoun\is{pronoun}; \textit{hel} is the third singular pronoun\is{pronoun}, which can be pluralized\is{plural (number) word} by \textit{loku}. While this is of great interest, other languages go further. A fine example is Miskitu\il{Miskitu}, a Misumalpan language of Nicaragua and Honduras. Number is marked by number words (\citealt{Greenms} Andrew Koontz-Garboden, p.c.), singular (\textit{kum}) and plural\is{plural (number) word} (\textit{nani}). Pronouns\is{pronoun} take the plural\is{plural (number) word} word, rather like nouns:


\ea%113
\label{ex:9:113}
\langinfo{Miskitu}{}{\citealt{Greenms}, Andrew Koontz-Garboden, p.c.} \\
\gll   Yang nani {kauhw-ri.}\\
  1 \textsc{pl} fall-1.\textsc{pst.indf}  \\
\glt `We (exclusive) fell.'
\z






This example, like all those cited above from Alor-Pantar languages, helps to extend the typology of number words; as we gather a fuller picture, the typology of number words becomes increasingly like that of affixal number.


\section{Conclusions} %6
\label{sec:9:6}
Proto-Alor-Pantar\il{proto-Alor-Pantar} had a plural\is{plural (number) word} word of the shape *{non}. Some daughter languages inherited this form, others innovated\is{innovation} one or more plural\is{plural (number) word} words. In none of the five AP languages investigated here do restrictions apply on the type of referents that can be pluralized\is{plural (number) word} with the plural\is{plural (number) word} word, and all of them prohibit a combination of the plural\is{plural (number) word} word and a numeral in a single constituent.

 The syntax of the plural\is{plural (number) word} word varies. In each language investigated here the word constitutes a class of its own. In Western Pantar\il{Western Pantar}, the plural\is{plural (number) word} word shares much with adjectival quantifiers and numerical expressions, in Teiwa\il{Teiwa} it patterns mostly with non-numeral quantifiers, and in Kamang\il{Kamang}, Abui\il{Abui} and Wersing\il{Wersing} plural\is{plural (number) word} words function very much like nouns. The plural\is{plural (number) word} words in the five languages behave differently, so that it is not possible to establish a category of plural\is{plural (number) word} word that is cross-linguistically uniform.

 The plural\is{plural (number) word} words all code plurality\is{plural (number) word}, but in all five languages they have additional connotations, such as expressing a sense of completeness or abundance. A plural\is{plural (number) word} word may also function to impose an individuated reading of a referent, or to pick out a part or group of referents from a larger set. Plural\is{plural (number) word} words are used to express plural\is{plural (number) word} vocatives. None of the additional senses and functions of the plural\is{plural (number) word} words is shared across all of the five languages.

 What our study shows is that, even amongst five typologically similar and genetically closely related languages whose ancestor had a plural\is{plural (number) word} word, the original plural\is{plural (number) word} word has drifted in different syntactic directions and developed additional semantic dimensions, showing a degree of variation that is higher than any other inherited word.


\section*{Acknowledgments}
We are grateful to the following colleagues for answering our questions and gracefully sharing their data: Gary Holton for Western Pantar\il{Western Pantar}, Franti\v{s}ek Kratochv\'il and Benny Delpada for Abui\il{Abui}, and Louise Baird for Klon\il{Klon}. We are also very grateful to Mary Darlymple and Martin Haspelmath for their comments on an earlier version of this paper.

\clearpage\section*{Abbreviations}
\begin{multicols}{2}
\begin{tabbing}
\scshape W \rm (Pantar) \= exclamation\kill
\scshape = \> clitic boundary\\
\scshape {\Tilde} \> reduplication\is{reduplication}\\
\scshape 1 \> 1st person\\
\scshape 2 \> 2nd person\\
\scshape 3 \> 3rd person\\
\scshape act \> actor\\
\scshape agt \> agent\\
\scshape all \> all\\
\scshape AP \> Alor-Pantar\\
\scshape appl \> applicative\\
\scshape appos \> apposition\\
\scshape art \> article\\
\scshape assoc \> associative\\
\scshape attr \> attribute slot\\
\scshape aux \> auxiliary\\
\scshape clf \> classifier\is{numeral classifier}\\
\scshape cnct \> connector\\
\scshape conj \> conjunction\is{conjunction}\\
\scshape cont \> continuous\\
\scshape contr \> contrastive focus\\
\scshape def \> definite\\
\scshape dem \> demonstrative\is{demonstrative}\\
\scshape det \> determiner\\
\scshape dist \> distal\\
\scshape emph \> emphasis\\
\scshape excl \> exclusive\\
\scshape exclam \> exclamation\\
\scshape foc \> focus\\
\scshape gen \> genitive\\
\scshape incl \> inclusive\\
\scshape indf \> indefinite\\
\scshape ipfv \> imperfective\\
\scshape knwn \> known\\
\scshape lnk \> linker\\
\scshape loc \> locative\\
\scshape N \> noun\\
\scshape NP \> noun phrase\\
\scshape Num(P) \> numeral (phrase)\\
\scshape obj \> object\\
\scshape obl \> oblique\\
\scshape \rm pAP \> proto-Alor-Pantar\il{proto-Alor-Pantar}\\
\scshape pfv \> perfective\\
\scshape pl \> plural\is{plural (number) word}\\
\scshape poss \> possessive\is{possession}\\ 
\scshape pst \> past\\
\scshape purp \> purposive\\
\scshape rc \> relative clause\\
\scshape rdp \> reduplication\is{reduplication}\\
\scshape real \> realis\\
\scshape sben \> self-benefactive\\
\scshape sg \> singular\\
\scshape spec \> specific\\
\scshape top \> topic\\
\scshape W \rm (Pantar) \> Western Pantar\il{Western Pantar} \\
\end{tabbing}
\end{multicols}












 
 
\printbibliography[heading=subbibliography,notkeyword=this]

\end{document} 
